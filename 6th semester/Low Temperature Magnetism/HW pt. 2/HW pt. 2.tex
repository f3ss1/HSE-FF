%
%Не забыть:
%--------------------------------------
%Вставить колонтитулы, поменять название на титульнике



%--------------------------------------

\documentclass[a4paper, 12pt]{article} 

%--------------------------------------
%Russian-specific packages
%--------------------------------------
%\usepackage[warn]{mathtext}
\usepackage[T2A]{fontenc}
\usepackage[utf8]{inputenc}
\usepackage[english,russian]{babel}
\usepackage[intlimits]{amsmath}
\usepackage{esint}
%--------------------------------------
%Hyphenation rules
%--------------------------------------
\usepackage{hyphenat}
\hyphenation{ма-те-ма-ти-ка вос-ста-нав-ли-вать}
%--------------------------------------
%Packages
%--------------------------------------
\usepackage{amsmath}
\usepackage{amssymb}
\usepackage{amsfonts}
\usepackage{amsthm}
\usepackage{latexsym}
\usepackage{mathtools}
\usepackage{etoolbox}%Булевые операторы
\usepackage{extsizes}%Выставление произвольного шрифта в \documentclass
\usepackage{geometry}%Разметка листа
\usepackage{indentfirst}
\usepackage{wrapfig}%Создание обтекаемых текстом объектов
\usepackage{fancyhdr}%Создание колонтитулов
\usepackage{setspace}%Настройка интерлиньяжа
\usepackage{lastpage}%Вывод номера последней страницы в документе, \lastpage
\usepackage{soul}%Изменение параметров начертания
\usepackage{hyperref}%Две строчки с настройкой гиперссылок внутри получаеммого
\usepackage[usenames,dvipsnames,svgnames,table,rgb]{xcolor}% pdf-документа
\usepackage{multicol}%Позволяет писать текст в несколько колонок
\usepackage{cite}%Работа с библиографией
\usepackage{subfigure}% Человеческая вставка нескольких картинок
\usepackage{tikz}%Рисование рисунков
\usepackage{float}% Возможность ставить H в положениях картинки
\usepackage{braket}% Бра-кет вектора
% Для картинок Моти
\usepackage{misccorr}
\usepackage{lscape}
\usepackage{cmap}



\usepackage{graphicx,xcolor}
\graphicspath{{Pictures/}}
\DeclareGraphicsExtensions{.pdf,.png,.jpg}

%----------------------------------------
%Список окружений
%----------------------------------------
\newenvironment {theor}[2]
{\smallskip \par \textbf{#1.} \textit{#2}  \par $\blacktriangleleft$}
{\flushright{$\blacktriangleright$} \medskip \par} %лемма/теорема с доказательством
\newenvironment {proofn}
{\par $\blacktriangleleft$}
{$\blacktriangleright$ \par} %доказательство
%----------------------------------------
%Список команд
%----------------------------------------
\newcommand{\grad}
{\mathop{\mathrm{grad}}\nolimits\,} %градиент

\newcommand{\diver}
{\mathop{\mathrm{div}}\nolimits\,} %дивергенция

\newcommand{\rot}
{\ensuremath{\mathrm{rot}}\,}

\newcommand{\Def}[1]
{\underline{\textbf{#1}}} %определение

\newcommand{\RN}[1]
{\MakeUppercase{\romannumeral #1}} %римские цифры

\newcommand {\theornp}[2]
{\textbf{#1.} \textit{ #2} \par} %Написание леммы/теоремы без доказательства

\newcommand{\qrq}
{\ensuremath{\quad \Rightarrow \quad}} %Человеческий знак следствия

\newcommand{\qlrq}
{\ensuremath{\quad \Leftrightarrow \quad}} %Человеческий знак равносильности

\renewcommand{\phi}{\varphi} %Нормальный знак фи

\newcommand{\me}
{\ensuremath{\mathbb{E}}}

\newcommand{\md}
{\ensuremath{\mathbb{D}}}

\newcommand{\med}[1]
{\ensuremath{\langle#1\rangle}}



%\renewcommand{\vec}{\overline}




%----------------------------------------
%Разметка листа
%----------------------------------------
\geometry{top = 3cm}
\geometry{bottom = 2cm}
\geometry{left = 1.5cm}
\geometry{right = 1.5cm}
%----------------------------------------
%Колонтитулы
%----------------------------------------
\pagestyle{fancy}%Создание колонтитулов
\fancyhead{}
%\fancyfoot{}
%----------------------------------------
%Интерлиньяж (расстояния между строчками)
%----------------------------------------
%\onehalfspacing -- интерлиньяж 1.5
%\doublespacing -- интерлиньяж 2
%----------------------------------------
%Настройка гиперссылок
%----------------------------------------
\hypersetup{				% Гиперссылки
	unicode=true,           % русские буквы в раздела PDF
	pdftitle={Заголовок},   % Заголовок
	pdfauthor={Автор},      % Автор
	pdfsubject={Тема},      % Тема
	pdfcreator={Создатель}, % Создатель
	pdfproducer={Производитель}, % Производитель
	pdfkeywords={keyword1} {key2} {key3}, % Ключевые слова
	colorlinks=true,       	% false: ссылки в рамках; true: цветные ссылки
	linkcolor=blue,          % внутренние ссылки
	citecolor=blue,        % на библиографию
	filecolor=magenta,      % на файлы
	urlcolor=cyan           % на URL
}
%----------------------------------------
%Работа с библиографией (как бич)
%----------------------------------------
\renewcommand{\refname}{Список литературы}%Изменение названия списка литературы для article
%\renewcommand{\bibname}{Список литературы}%Изменение названия списка литературы для book и report
%----------------------------------------
\begin{document}
	\begin{titlepage}
		\begin{center}
			$$$$
			$$$$
			$$$$
			$$$$
			{\Large{НАЦИОНАЛЬНЫЙ ИССЛЕДОВАТЕЛЬСКИЙ УНИВЕРСИТЕТ}}\\
			\vspace{0.1cm}
			{\Large{ВЫСШАЯ ШКОЛА ЭКОНОМИКИ}}\\
			\vspace{0.25cm}
			{\large{Факультет физики}}\\
			\vspace{5.5cm}
			{\Huge\textbf{{Домашние работы по второй части курса}}}\\%Общее название
			\vspace{2cm}
			{Работу выполнил студент 3 курса}\\
			{Захаров Сергей Дмитриевич}\\
			\vfill
			\includegraphics[width = 0.2\textwidth]{HSElogo}\\
			\vfill
			Москва\\
			2021
		\end{center}
	\end{titlepage}
	
\tableofcontents


\newpage

\section{\textit{Изучается рассеянный под углом 90$^\circ$ луч оптического диапазона. Измеряется смещение частоты $\Delta \omega$, возникающее из-за поглощения или излучения квантов упругих колебаний. Найти скорость звука $v_{\text{зв}}$ по измеренной величине $\Delta \omega / \omega$}}

Запишем сразу разность частот:

\begin{equation}
	\omega_{fall} - \omega_{dis} = \Delta \omega
\end{equation}

Энергия, которая передается одному фонону с импульсом $p$:

\begin{equation}
	\hbar (\omega_{dis} - \omega_{fall}) = u p
	\label{eq:8_onep}
\end{equation}

Запишем закон сохранения импульса:

\begin{equation}
	\vec{p_1} = \vec{p_2} + \vec{p}, \quad \vec{p} = \vec{p_1} - \vec{p_2} \qrq p_y = -p_2, \quad p_x = p_1
\end{equation}

Для значений импульса $p_1, p_2$ можем записать:

\begin{equation}
	p_1 = \frac{\hbar \omega_{fall}}{c}, \quad p_2 = \frac{\hbar \omega_{dis}}{c} \qrq p_x = \frac{\hbar \omega}{c}, \quad p_y = -\frac{\hbar (\omega - \Delta \omega)}{c}
\end{equation}

Теперь, подставляя в \ref{eq:8_onep}:

\begin{align*}
	\hbar\Delta \omega = u \sqrt{p_x^2 + p_y^2} = u \sqrt{\left(\frac{\hbar\omega}{c}\right)^2 + \left[\frac{\hbar}{c}(\omega - \Delta \omega)\right]^2} \qrq \\
	\Rightarrow \quad \Delta \omega = u \frac{\omega}{c} \sqrt{1 + \left(1 - \frac{\Delta\omega}{\omega}\right)^2} \qrq u = \frac{c \cdot \Delta\omega / \omega}{\sqrt{1 + \left(1 - \dfrac{\Delta\omega}{\omega}\right)^2}}
\end{align*}

\newpage

\section{\textit{Найти теплоемкость для ферромагнетика с $J = 1000$~К, при $T\sim100$~К, $\theta \sim 50$~К, сравнить с решеточной теплоемкостью в нулевом магнитном поле}}

\begin{align*}
	E = V \int\limits_0^{k_{max}} \frac{d k \; k^2 \hbar \omega}{2\pi^2 \left[\exp\left(\dfrac{\hbar\omega}{k T} - 1\right)\right]} = [\omega = \alpha k^2 \Rightarrow k \sqrt{\omega/\alpha}, dk = d\omega / (2\sqrt{\alpha\omega})] = \\
	= V\int\limits_0^{\omega_{max}} d\omega\;  \frac{\hbar \omega^2}{\alpha \cdot 2\sqrt{\alpha \omega}\cdot 2\pi^2 \left[\exp\left(\dfrac{\hbar\omega}{k T}\right) - 1\right]} = [x = \hbar\omega / (k T)] = V \cdot \frac{\hbar}{4 \pi^2 \alpha\sqrt{\alpha}} \int\limits_0^{\theta/T} dx \; \left(\frac{k T}{\hbar}\right)^{5/2} \frac{x^{3/2}}{(e^x - 1)} = \\
	= \frac{V \hbar (k T)^{5/2}}{4 \pi^2 \alpha\sqrt{\alpha} \hbar^2 \sqrt{\hbar}} \int\limits_0^{1/2} dx \; \frac{x^{3/2}}{e^x - 1} \approx \frac{V (k T)^{5/2}}{4 \pi^2 (\alpha \hbar)^{3/2}} \int\limits_0^{1/2} dx \; \sqrt{x} = \frac{3 V (k T)^{5/2}}{16\sqrt{2} \pi^2 (\alpha \hbar)^{3/2}} = [\alpha = \frac{2 J S a^2}{\hbar}] = \\
	 = \frac{3 V(k T)^{5/2}}{64 \pi^2 (J S)^{3/2} a^3}
\end{align*}

Тогда для теплоемкости можем записать:

\begin{equation}
	c_V = \left(\frac{\partial E}{\partial T}\right)_V = \frac{15 V (k T)^{3/2}}{128 \pi^2 (JS)^{3/2}a^3}
\end{equation}

Если сюда подставить все числа ($s = 1/2$, $a \approx 1$\AA, $V \approx 1\text{ см}^3$ и то, что в условии дано), то получится что-то около (в единицах постоянной Больцмана): 

\begin{equation}
	c_V \approx 10^{21} k
\end{equation}

В то же время для решеточной:

\begin{equation}
	c \approx 3R \frac{V}{a^3 N_a} = 3 \frac{V}{a^3} k \approx 3 \cdot 10^{24} k 
\end{equation}

Итого теплоемкость от магнонов в порядка 1000 раз меньше решеточной.

\end{document}