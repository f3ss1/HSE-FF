%
%Не забыть:
%--------------------------------------
%Вставить колонтитулы, поменять название на титульнике



%--------------------------------------

\documentclass[a4paper, 12pt]{article} 

%--------------------------------------
%Russian-specific packages
%--------------------------------------
%\usepackage[warn]{mathtext}
\usepackage[T2A]{fontenc}
\usepackage[utf8]{inputenc}
\usepackage[english,russian]{babel}
\usepackage[intlimits]{amsmath}
\usepackage{esint}
%--------------------------------------
%Hyphenation rules
%--------------------------------------
\usepackage{hyphenat}
\hyphenation{ма-те-ма-ти-ка вос-ста-нав-ли-вать}
%--------------------------------------
%Packages
%--------------------------------------
\usepackage{amsmath}
\usepackage{amssymb}
\usepackage{amsfonts}
\usepackage{amsthm}
\usepackage{latexsym}
\usepackage{mathtools}
\usepackage{etoolbox}%Булевые операторы
\usepackage{extsizes}%Выставление произвольного шрифта в \documentclass
\usepackage{geometry}%Разметка листа
\usepackage{indentfirst}
\usepackage{wrapfig}%Создание обтекаемых текстом объектов
\usepackage{fancyhdr}%Создание колонтитулов
\usepackage{setspace}%Настройка интерлиньяжа
\usepackage{lastpage}%Вывод номера последней страницы в документе, \lastpage
\usepackage{soul}%Изменение параметров начертания
\usepackage{hyperref}%Две строчки с настройкой гиперссылок внутри получаеммого
\usepackage[usenames,dvipsnames,svgnames,table,rgb]{xcolor}% pdf-документа
\usepackage{multicol}%Позволяет писать текст в несколько колонок
\usepackage{cite}%Работа с библиографией
\usepackage{subfigure}% Человеческая вставка нескольких картинок
\usepackage{tikz}%Рисование рисунков
\usepackage{float}% Возможность ставить H в положениях картинки
% Для картинок Моти
\usepackage{misccorr}
\usepackage{lscape}
\usepackage{cmap}



\usepackage{graphicx,xcolor}
\graphicspath{{Pictures/}}
\DeclareGraphicsExtensions{.pdf,.png,.jpg}

%----------------------------------------
%Список окружений
%----------------------------------------
\newenvironment {theor}[2]
{\smallskip \par \textbf{#1.} \textit{#2}  \par $\blacktriangleleft$}
{\flushright{$\blacktriangleright$} \medskip \par} %лемма/теорема с доказательством
\newenvironment {proofn}
{\par $\blacktriangleleft$}
{$\blacktriangleright$ \par} %доказательство
%----------------------------------------
%Список команд
%----------------------------------------
\newcommand{\grad}
{\mathop{\mathrm{grad}}\nolimits\,} %градиент

\newcommand{\diver}
{\mathop{\mathrm{div}}\nolimits\,} %дивергенция

\newcommand{\rot}
{\ensuremath{\mathrm{rot}}\,}

\newcommand{\Def}[1]
{\underline{\textbf{#1}}} %определение

\newcommand{\RN}[1]
{\MakeUppercase{\romannumeral #1}} %римские цифры

\newcommand {\theornp}[2]
{\textbf{#1.} \textit{ #2} \par} %Написание леммы/теоремы без доказательства

\newcommand{\qrq}
{\ensuremath{\quad \Rightarrow \quad}} %Человеческий знак следствия

\newcommand{\qlrq}
{\ensuremath{\quad \Leftrightarrow \quad}} %Человеческий знак равносильности

\renewcommand{\phi}{\varphi} %Нормальный знак фи

\newcommand{\me}
{\ensuremath{\mathbb{E}}}

\newcommand{\md}
{\ensuremath{\mathbb{D}}}



%\renewcommand{\vec}{\overline}




%----------------------------------------
%Разметка листа
%----------------------------------------
\geometry{top = 3cm}
\geometry{bottom = 2cm}
\geometry{left = 1.5cm}
\geometry{right = 1.5cm}
%----------------------------------------
%Колонтитулы
%----------------------------------------
\pagestyle{fancy}%Создание колонтитулов
\fancyhead{}
%\fancyfoot{}
%----------------------------------------
%Интерлиньяж (расстояния между строчками)
%----------------------------------------
%\onehalfspacing -- интерлиньяж 1.5
%\doublespacing -- интерлиньяж 2
%----------------------------------------
%Настройка гиперссылок
%----------------------------------------
\hypersetup{				% Гиперссылки
	unicode=true,           % русские буквы в раздела PDF
	pdftitle={Заголовок},   % Заголовок
	pdfauthor={Автор},      % Автор
	pdfsubject={Тема},      % Тема
	pdfcreator={Создатель}, % Создатель
	pdfproducer={Производитель}, % Производитель
	pdfkeywords={keyword1} {key2} {key3}, % Ключевые слова
	colorlinks=true,       	% false: ссылки в рамках; true: цветные ссылки
	linkcolor=blue,          % внутренние ссылки
	citecolor=blue,        % на библиографию
	filecolor=magenta,      % на файлы
	urlcolor=cyan           % на URL
}
%----------------------------------------
%Работа с библиографией (как бич)
%----------------------------------------
\renewcommand{\refname}{Список литературы}%Изменение названия списка литературы для article
%\renewcommand{\bibname}{Список литературы}%Изменение названия списка литературы для book и report
%----------------------------------------
\begin{document}
	\begin{titlepage}
		\begin{center}
			$$$$
			$$$$
			$$$$
			$$$$
			{\Large{НАЦИОНАЛЬНЫЙ ИССЛЕДОВАТЕЛЬСКИЙ УНИВЕРСИТЕТ}}\\
			\vspace{0.1cm}
			{\Large{ВЫСШАЯ ШКОЛА ЭКОНОМИКИ}}\\
			\vspace{0.25cm}
			{\large{Факультет физики}}\\
			\vspace{5.5cm}
			{\Huge\textbf{{Лабораторные работы по спектроскопии и дифракции}}}\\%Общее название
			\vspace{1cm}
			{Работу выполнил студент 3 курса}\\
			{Захаров Сергей Дмитриевич}
			\vfill
			\includegraphics[width = 0.2\textwidth]{HSElogo}\\
			\vfill
			Москва\\
			2021
		\end{center}
	\end{titlepage}
	
	\tableofcontents
	
	\newpage
	
	\section{Постановка цели}
	
	\subsection{Оже-спектроскопия}
	
	В качестве исследуемого вещества нам был предложен неизвестный сыпучий образец, спрессованный в форму ломкой цилиндрической таблетки. Предлагалось подготовить ее для внесения в вакуумную камеру, после чего внести в нее и получить Оже-спектр, после чего путем его анализа постараться определить, из каких элементов образец состоит.
	
	\subsection{Сканирующая туннельная микроскопия}
	
	Было решено на уже внесенном в вакуум графите получить обзорный кадр и определить высоту ступеньки. Забегая вперед, в ходе эксперимента проявился т.н. муаровый узор, возникающий при наложении двух периодических сетчатых рисунков, например, слоев решетки графита, поэтому было также предложено по нему определить взаимное расположение слоев, дающих рисунок.
	
	\subsection{Сканирующая туннельная спектроскопия}
	
	Также на уже внесенном в вакуум графите было предложено получить вольт-амперную характеристику (ВАХ) туннельного тока от напряжения.
	
	\section{Оже-спектроскопия}
	
	\subsection{Принцип метода}
	
	Мы пользуемся тем фактом, что энергия связи электронов глубоких оболочек атома чувствительна к природе элемента, что позволяет, измеряя кинетическую энергию эмитированных с поверхности под действием фотонной или электронной бомбардировки, получать информацию об \textbf{элементном} составе поверхности. Мы также пользуемся тем, что электроны с кинетической энергией 15-1000~эВ обладают очень маленькими длинами свободного пробега в веществе, что позволяет получать информацию о поверхности.
	
	При бомбардировке образца электронами с энергией порядка 3000~эВ происходит несколько параллельных процессов. Во-первых, упругое рассеяние электронов на электронных оболочках атомов. Эти электроны покидают образец без изменения энергии. Во-вторых, неупругое рассеяние электронов на электронных оболочках атомов, в частности нас интересует рассеяние на электронах внутренних оболочек атомов.
	
	Мы рассматриваем оже-пики, которые появляются вследствие т.н. Оже-процесса. Схема процесса представлена на рисунке \ref{fig:1_diag} и состоит из трех этапов. Сперва первичный электрон с энергией порядка 2-3~кЭв выбивает электрон с оболочки атома (этот электрон называется вторичным), образуя тем самым вакансию (а). После этого происходит релаксация за счет внутреннего перехода электрона с более высокого уровня на получившуюся вакансию (б). Наконец, испускается Оже-электрон, который мы детектируем и кинетическую энергию которого мы измеряем (в).
	
	\begin{figure}[H]
		\centering
		\includegraphics[width=0.7\linewidth]{1_diag}
		\caption{Схематическая иллюстрация оже-процесса из \cite{Auge_diag}.}
		\label{fig:1_diag}
	\end{figure}
	
	\subsection{Подготовка образца}
	
	Первоначально предлагалось закрепить образец на держателе с помощью танталовой нити с использованием контактной сварки, однако спустя несколько попыток было решено, что данный способ при отсутствии должной практики крайне сложен в практическом исполнении, принимая во внимание тот факт, что образец был цилиндрической формы. По этой причине было решено <<накрыть>> образец танталовой пластиной, предварительно просверлив в ней отверстие достаточного диаметра для получения Оже-спектра и сделав <<ножки>>, с помощью которых образец бы держался между пластинами.
	
	%\begin{figure}[H]
	%	\centering
	%	\includegraphics[width=0.7\linewidth]{1_fastening}
	%	\caption{Крепление образца с помощью танталовой пластины.}
	%	\label{fig:1_fastening}
	%\end{figure}
	
	\subsection{Анализ Оже-спектра}
	
	Изначально был получен Оже-спектр при бомбардировке образца электронами с энергией 3000~эВ с помощью установки, описанной в \cite{Auger_spectr}. С учетом возможности наличия т.н. пиков потерь, которые находятся в той же области, что и оже-пики, было решено проверить их присутствие с помощью увеличения энергии бомбардирующих электронов на 500~эВ. В таком случае Оже-пики, которые являются характеристикой вещества, должны были бы остаться на месте, а пики потерь --- сместиться. Из рисунка \ref{fig:1_Auge_double} видно, что смещения ни одного из пиков не наблюдается, что свидетельствует о том, что все пики являются оже-пиками.
	
	В силу совпадения оже-спектров анализ проводился только спектра 3000~эВ. Однако в более <<далекой>> по энергиям части спектра для того, чтобы лучше разрешить пики, оказался полезен и спектр 3500~эВ. Понятно, что в идеале хотелось бы иметь спектр 5000~эВ, пики с которого соответствовали бы, как и пики с 3000~эВ, каталожным, однако это оказалось невозможно по независящим от нас техническим причинам. Проанализированный спектр 3000~эВ представлен на рисунке \ref{fig:1_Auge_3000}.
	
	Как было сказано, спектр 3500~эВ оказался полезен в <<дальней>> части спектра: с его помощью был уточнен пик, предположительно, тулия. Кроме того, благодаря тому, что это сканирование мы запустили с меньшего нижнего порога по энергиям, из него также можно было вытащить один дополнительный пик в начале, который оказался натриевым. Эти два ненайденных с помощью спектра 3000~эВ пика указаны на спектре 3500~эВ на рисунке \ref{fig:1_Auge_3500}. 
	
	В качестве каталожных спектров были взяты спектры из \cite{Auger}.
	
	\section{Сканирующая туннельная микроскопия}
	
	\subsection{Принцип метода}
	
	Сканирование осуществляется с помощью специальной очень острой металлической иглы, в идеале на крайней точке которого сидит один единственный атом. Если достаточно близко приблизить образец к игле и подать напряжение, то потечет туннельный ток, направление которого может меняться (с образца на иглу, или наоборот) в зависимости от полярности напряжения. По зависимости величины тока от напряжения можно получить информацию о расстоянии между зондом и атомами поверхности образца. Вероятность туннельного эффекта зависит экспоненциально от расстояния, что и обеспечивает высокое разрешение данного метода.
	
	Более полное описание микроскопа GPI-300, на котором проводилось сканирование, представлено в \cite{Eltsov}. Для сканирования, как было сказано выше, был выбран графитовый образец.
	
	\subsection{Измерение высоты ступеньки и получение обзорного СТМ-кадра}
	
	С помощью последовательного сканирования различных областей, была найдена область с несколькими ступенями. Полученное изображение представлено на рисунке \ref{fig:2_step}. График перепада высот на исследованной ступеньке указан на рисунке \ref{fig:2_step_anal}. Из последнего видно, что высота ступени определяется как $3.5$~\AA.
	
	\subsection{Наблюдение атомной структуры}
	
	
	
	\subsection{Зависимость СТМ-изображения атомной структуры от напряжения}
	
	На одном и том же участке было проведено несколько последовательных сканирований при различных напряжениях с целью найти какую-то зависимость качества изображения от этого напряжения. В результате было определено, что явной зависимости нет, а параметры необходимо подбирать индивидуально. Полученные кадры представлены на рисунке \ref{fig:2_different_volt}, а профили одного из рядов на них --- на рисунке \ref{fig:2_different_volt_profiles}. Из последнего видно, что с увеличением напряжения ''глубина'' профиля становится все меньше.
	
	\subsection{Наблюдение муара}
	
	Увиденный нами муар представлен на рисунке \ref{fig:2_muar}. Наблюдаются атомные модуляции, а также модуляция сверхструктуры. На Фурье-образе также видно две структуры: атомный гексагон и гексагон сверхструктуры. Было вычислено расстояние между соседними элементами сверхструктуры, которое оказалось равным $7.64\text{ нм}/2 = 3.82$~нм. Реально было измерено расстояние не между двумя соседними элементами, а между элементами, находящимися через один, чтобы накопить большую статистику.
	
	Было высказано предположение, что муар составлен двумя ''слоями'' графита (''слои'' здесь подразумевают не атомные слои, см. пункт про атомную структуру). С помощью моделирования было установлена, что ''слои'' должны быть развернуты друг относительно друга на 6.2$^\circ$. Смоделированный муар представлен на рисунке \ref{fig:2_muar_model}.
	
	\section{Сканирующая туннельная спектроскопия}
	
	\subsection{Принцип метода}
	
	Согласно \cite{Mironov} скажем, что выражения для туннельного тока может быть представлено в следующем виде:
	
	\begin{equation}
		dI = A \cdot D(E) \rho_P (E) f_P(E) \rho_S(E)(1 - f_S(E)) dE
	\end{equation}
	
	Здесь $A$ --- некоторая постоянная, $D(E)$ --- прозрачность барьера, $\rho_P(E), \rho_S(E)$ --- плотности состояний в материале зонда и исследуемого образца соответственно, $f(E)$ --- функция распределения Ферми.
	
	Предполагая, что плотность состояний вблизи уровня ферми в зонде практически постоянна, а также предполагая, что температуры низкие, мы можем записать:
	
	\begin{equation}
		I(V) \propto \int\limits_0^{eV} \rho_S(E)dE 
	\end{equation} 
	
	Тогда зависимость туннельного тока от напряжения определяется плотностью состояний в энергетическом спектре образца. Тогда для плотности состояний мы можем записать:
	
	\begin{equation}
		\rho_S(eV) \propto \frac{\partial I}{\partial V}
	\end{equation}
	
	То есть на деле, снимая ВАХ, а затем дифференцируя ее, мы можем судить о плотности состояний в исследуемом образце.
	
	\subsection{Снятие ВАХ}
	
	На уже использованном для СТМ графитовом образце было снято несколько ВАХ на одном и том же месте. После анализа результатов была выбрана наименее зашумленная серия, которая после этого была дополнительно отфильтрована. Полученная ВАХ (исходная и после фильтра) представлена на рисунке \ref{fig:3_STS}. Видно, что на ней все еще достаточно много шумов. Чтобы это улучшить, вероятно, стоит изначально подводить иглу ближе к образу (т.е. менять $U_{\text{base}}$), а также уменьшить шаг по напряжению.
	
	\section{Выводы}
	
	В результате проведения лабораторной работы:
	
	\begin{enumerate}
		\item Был расшифрован Оже-спектр неизвестного образца. Предположительно он состоит из натрия, фтора, тулия и иттрия. Принимается, что пики углерода и кислорода происходят из остатков газа в вакуумной камере.
		
		\item Была измерена высота ступеньки на графитовом образце. Ее высота оказалась равной 3.5~\AA.
		
		\item Были получены различные СТМ-кадры, в том числе и обзорные, и кадры, отображающие атомную структуру.
		
		\item Был проведен эксперимент по поиску зависимости СТМ-изображения атомной структуры от напряжения, в ходе которого было установлено, что с ростом напряжения ''глубина'' кадра уменьшается.
		
		\item При сканировании был обнаружен муар, который, по всей видимости, вызывается поворотом двух (или более) ''слоев'' решетки относительно друг друга. В результате компьютерного моделирования этого предположения было установлено, что в таком случае слои должны быть повернуты друг относительно друга на $6.2^\circ$.
		
		\item Была получена ВАХ, которая отдаленно напоминает ВАХ, например, из \cite{Article}, однако она сильно зашумлена. В качестве вариантов решения проблемы в будущем были предложены более близкое подведение иглы к образцу и уменьшение шага по напряжению.
	\end{enumerate}
	
	\newpage
	
	\begin{thebibliography}{2}
		
		\bibitem{Auger} Handbook of Auger Electron Spectroscopy. /  Lawrence E. Davis, Noel C. MacDonald, Paul W. Palmberg, Greald E. Riach, Roland E. Wever --- Physical Electrinics Industries, Inc., February 1976.
		
		\bibitem{Auger_spectr} Описание работы оже-спектрометра. / ??
		
		\bibitem{STM} Общее описание СТМ GPI-300. / ??
		
		\bibitem{Article} Scanning tunneling microscopy and spectroscopy of the electronic local density of states of graphite surfaces near monoatomic step edges. / Y. Niimi, T. Matsui, H. Kambara, K. Tagami, M. Tsukada, and Hiroshi Fukuyama --- Tokyo, Japan : Department of Physics, University of Tokyo, February 24, 2006.
		
		\bibitem{Auge_diag} Анализ поверхности методами Оже- и рентгеновской фотоэлектронной спектроскопии. / Бриггса Д., Сиха М.М.
		
		\bibitem{Eltsov} Сверхвысоковакуумный сканирующий туннельный микроскоп GPI-300. / Ельцов К.Н. , А.Н. Климов, А.Н. Косяков, О.В. Объедков, В.Ю. Юров, В.М. Шевлюга --- Москва : Труды института общей физики им. А.М. Прохорова, 2003.
		
		\bibitem{Mironov} Основы сканирующей зондовой микроскопии. / В.Л. Миронов --- Нижний Новгород : Институт физики микроструктур, 2004
		
	\end{thebibliography}
	
	
	\newpage
	
	% Оже-спектры выношу отдельно, по одному на страницу, чтобы не сидеть "а че там написано"
	
	\begin{figure}[H]
		\centering
		\includegraphics[width=1.3\linewidth, angle=-90]{1_Auge_double}
		\caption{Сравнение Оже-спектров, полученных при 3000~эВ и 3500~эВ.}
		\label{fig:1_Auge_double}
	\end{figure}
	
	\newpage
	
	\begin{figure}[H]
		\centering
		\includegraphics[width=1.3\linewidth, angle=-90]{1_Auge_3000}11
		\caption{Полученный Оже-спектр 3000~эВ с нанесенными названиями элементов, дающих пики.}
		\label{fig:1_Auge_3000}
	\end{figure}
	
	\newpage
	
	\begin{figure}[H]
		\centering
		\includegraphics[width=1.3\linewidth, angle=-90]{1_Auge_3500}
		\caption{Полученный Оже-спектр 3000~эВ с нанесенными названиями элементов, дающих не найденные на 3000~эВ пики.}
		\label{fig:1_Auge_3500}
	\end{figure}
	
	\newpage
	
	\begin{figure}[H]
		\centering
		\includegraphics[width=0.6\linewidth]{../STM_data/Step/Step final}
		\caption{Обзорный СТМ-кадр поверхности графита.}
		\label{fig:2_step}
	\end{figure}
	
	\begin{figure}[H]
		\centering
		\includegraphics[width=0.6\linewidth]{../STM_data/Step/Step final Graph_enh}
		\caption{Профиль ступеньки графита.}
		\label{fig:2_step_anal}
	\end{figure}
	
	\begin{figure}[H]
		\centering
		\subfigure[]{
			\includegraphics[width=0.36\linewidth]{../STM_data/STM Profiles new/Pictures/25_6_1}
		}	
		\subfigure[]{
			\includegraphics[width=0.36 \linewidth]{../STM_data/STM Profiles new/Pictures/64_4_1}
		}
		\subfigure[]{
			\includegraphics[width=0.36\linewidth]{../STM_data/STM Profiles new/Pictures/93_3_1}
		}	
		\subfigure[]{
			\includegraphics[width=0.36 \linewidth]{../STM_data/STM Profiles new/Pictures/150_1_1}
		}
		\subfigure[]{
			\includegraphics[width=0.36 \linewidth]{../STM_data/STM Profiles new/Pictures/200_1}
		}
		
		\caption{СТМ-изображение атомной структуры при различных напряжениях. Слева направо и сверху вниз при напряжении, соответственно: 25.6~мВ, 64.4~мВ, 93.3~мВ, 150.1~мВ, 200~мВ. Размеры всех изображений составляют $2.6\times2.6$~нм, ток --- 0.4~нА.}
		\label{fig:2_different_volt}
	\end{figure}

	\begin{figure}[H]
		\centering
		\includegraphics[width=0.9\linewidth]{../STM_data/STM Profiles new/Graphs/All graphs}
		\caption{Профили атомного изображения при одинаковых токах (0.4~нА), но при различных напряжениях: зеленый --- 25.6~мВ, красный --- 64.4~мВ, синий --- 93.3~мВ, фиолетовый --- 150.1~мВ, оранжевый --- 200~мВ.}
		\label{fig:2_different_volt_profiles}
	\end{figure}
	
	\begin{figure}[H]
		\centering
		\subfigure[]{
			\includegraphics[width=0.45\linewidth]{../STM_data/Muar/Muar final}
		}
		\subfigure[]{
			\includegraphics[width=0.45\linewidth]{../STM_data/Muar/Muar final_f}
		}
		\subfigure[]{
			\includegraphics[width=0.6\linewidth]{../STM_data/Muar/Muar final g}
		}
		\caption{СТМ-кадр графита с наблюдающимся муаром (a), его Фурье-образ (b), профиль для определения расстояния между элементами сверхструктуры, оказавшегося равным 3.82~нм (c).}
		\label{fig:2_muar}
	\end{figure}

	\begin{figure}[H]
		\centering
		\includegraphics[width=0.9\linewidth]{../STM_data/Muar/Muar model}
		\caption{Компьютерное моделирование сверхструктуры.}
		\label{fig:2_muar_model}
	\end{figure}
	
	\begin{figure}[H]
		\centering
		\subfigure[]{
			\includegraphics[width=0.6\linewidth]{3_curve}
		}
		\subfigure[]{
			\includegraphics[width=0.6\linewidth]{3_curve_smooth}
		}
		\caption{Кривая $dI/dU$, измеренная при $|U| < 400$~mV (сверху) и она же, пропущенная через фильтр (снизу).}
		\label{fig:3_STS}
	\end{figure}
	
	
	
	
	
	
	
	
	
	
	
	
	
	
	
	
	
	
	
	
	
	
	
	
	
	
	
	
	
	
	
	
	
	
\end{document}