%
%Не забыть:
%--------------------------------------
%Вставить колонтитулы, поменять название на титульнике



%--------------------------------------

\documentclass[a4paper, 12pt]{article} 

%--------------------------------------
%Russian-specific packages
%--------------------------------------
%\usepackage[warn]{mathtext}
\usepackage[T2A]{fontenc}
\usepackage[utf8]{inputenc}
\usepackage[english,russian]{babel}
\usepackage[intlimits]{amsmath}
\usepackage{esint}
%--------------------------------------
%Hyphenation rules
%--------------------------------------
\usepackage{hyphenat}
\hyphenation{ма-те-ма-ти-ка вос-ста-нав-ли-вать}
%--------------------------------------
%Packages
%--------------------------------------
\usepackage{amsmath}
\usepackage{amssymb}
\usepackage{amsfonts}
\usepackage{amsthm}
\usepackage{latexsym}
\usepackage{mathtools}
\usepackage{etoolbox}%Булевые операторы
\usepackage{extsizes}%Выставление произвольного шрифта в \documentclass
\usepackage{geometry}%Разметка листа
\usepackage{indentfirst}
\usepackage{wrapfig}%Создание обтекаемых текстом объектов
\usepackage{fancyhdr}%Создание колонтитулов
\usepackage{setspace}%Настройка интерлиньяжа
\usepackage{lastpage}%Вывод номера последней страницы в документе, \lastpage
\usepackage{soul}%Изменение параметров начертания
\usepackage{hyperref}%Две строчки с настройкой гиперссылок внутри получаеммого
\usepackage[usenames,dvipsnames,svgnames,table,rgb]{xcolor}% pdf-документа
\usepackage{multicol}%Позволяет писать текст в несколько колонок
\usepackage{cite}%Работа с библиографией
\usepackage{subfigure}% Человеческая вставка нескольких картинок
\usepackage{tikz}%Рисование рисунков
\usepackage{float}% Возможность ставить H в положениях картинки
\usepackage{mhchem}
% Для картинок Моти
\usepackage{misccorr}
\usepackage{lscape}
\usepackage{cmap}



\usepackage{graphicx,xcolor}
\graphicspath{{Pictures/}}
\DeclareGraphicsExtensions{.pdf,.png,.jpg}

%----------------------------------------
%Список окружений
%----------------------------------------
\newenvironment {theor}[2]
{\smallskip \par \textbf{#1.} \textit{#2}  \par $\blacktriangleleft$}
{\flushright{$\blacktriangleright$} \medskip \par} %лемма/теорема с доказательством
\newenvironment {proofn}
{\par $\blacktriangleleft$}
{$\blacktriangleright$ \par} %доказательство
%----------------------------------------
%Список команд
%----------------------------------------
\newcommand{\grad}
{\mathop{\mathrm{grad}}\nolimits\,} %градиент

\newcommand{\diver}
{\mathop{\mathrm{div}}\nolimits\,} %дивергенция

\newcommand{\rot}
{\ensuremath{\mathrm{rot}}\,}

\newcommand{\Def}[1]
{\underline{\textbf{#1}}} %определение

\newcommand{\RN}[1]
{\MakeUppercase{\romannumeral #1}} %римские цифры

\newcommand {\theornp}[2]
{\textbf{#1.} \textit{ #2} \par} %Написание леммы/теоремы без доказательства

\newcommand{\qrq}
{\ensuremath{\quad \Rightarrow \quad}} %Человеческий знак следствия

\newcommand{\qlrq}
{\ensuremath{\quad \Leftrightarrow \quad}} %Человеческий знак равносильности

\renewcommand{\phi}{\varphi} %Нормальный знак фи

\newcommand{\me}
{\ensuremath{\mathbb{E}}}

\newcommand{\md}
{\ensuremath{\mathbb{D}}}

\newcommand{\med}[1]
{\ensuremath{\langle#1\rangle}}



%\renewcommand{\vec}{\overline}




%----------------------------------------
%Разметка листа
%----------------------------------------
\geometry{top = 3cm}
\geometry{bottom = 2cm}
\geometry{left = 1.5cm}
\geometry{right = 1.5cm}
%----------------------------------------
%Колонтитулы
%----------------------------------------
\pagestyle{fancy}%Создание колонтитулов
\fancyhead{}
%\fancyfoot{}
%----------------------------------------
%Интерлиньяж (расстояния между строчками)
%----------------------------------------
%\onehalfspacing -- интерлиньяж 1.5
%\doublespacing -- интерлиньяж 2
%----------------------------------------
%Настройка гиперссылок
%----------------------------------------
\hypersetup{				% Гиперссылки
	unicode=true,           % русские буквы в раздела PDF
	pdftitle={Заголовок},   % Заголовок
	pdfauthor={Автор},      % Автор
	pdfsubject={Тема},      % Тема
	pdfcreator={Создатель}, % Создатель
	pdfproducer={Производитель}, % Производитель
	pdfkeywords={keyword1} {key2} {key3}, % Ключевые слова
	colorlinks=true,       	% false: ссылки в рамках; true: цветные ссылки
	linkcolor=blue,          % внутренние ссылки
	citecolor=blue,        % на библиографию
	filecolor=magenta,      % на файлы
	urlcolor=cyan           % на URL
}
%----------------------------------------
%Работа с библиографией (как бич)
%----------------------------------------
\renewcommand{\refname}{Список литературы}%Изменение названия списка литературы для article
%\renewcommand{\bibname}{Список литературы}%Изменение названия списка литературы для book и report
%----------------------------------------
\begin{document}
	\begin{titlepage}
		\begin{center}
			$$$$
			$$$$
			$$$$
			$$$$
			{\Large{НАЦИОНАЛЬНЫЙ ИССЛЕДОВАТЕЛЬСКИЙ УНИВЕРСИТЕТ}}\\
			\vspace{0.1cm}
			{\Large{ВЫСШАЯ ШКОЛА ЭКОНОМИКИ}}\\
			\vspace{0.25cm}
			{\large{Факультет физики}}\\
			\vspace{5.5cm}
			{\Huge\textbf{{Домашняя работа 2}}}\\%Общее название
			\vspace{1cm}
			{\LARGE{Лазерная спектроскопия}}\\%Точное название
			\vspace{2cm}
			{Работу выполнил студент 3 курса}\\
			{Захаров Сергей Дмитриевич}\\
			\vfill
			\includegraphics[width = 0.2\textwidth]{HSElogo}\\
			\vfill
			Москва\\
			2021
		\end{center}
	\end{titlepage}


\newpage

\section*{Задача 1}

\begin{itemize}
	\item Записать энергии колебательных уровней двухатомной молекулы с учетом ангармонизма;
	
	\item Найти энергию возбуждения из основного колебательного состояния в первое на примере молекулы \ce{HC}, выразить в см$^{-1}$, если круговая частота $2\pi\nu = 5.6 \cdot 10^{14}$ с$^{-1}$, а константа ангармонизма  $x_e = 0.02$.
\end{itemize}

Для колебательных термов мы можем записать:

\begin{equation}
	E_\nu = h\nu_0 \left(\nu + \frac{1}{2}\right) \left[1 - x_e\left( \nu + \frac{1}{2}\right)\right]
\end{equation}

Для энергии возбуждения мы должны записать разность:

\begin{equation}
	\Delta E_{0\rightarrow1} = E_1 - E_0 = h \nu_0 \cdot (1 - 2 x_e) = \hbar \omega (1 - 2x_e) \approx 2852 \text{ см}^{-1}
\end{equation}


\section*{Задача 2}

\begin{itemize}
	\item Записать выражение для вращательных уровней энергии двухатомной молекулы;
	
	\item Найти характерный масштаб расстояний между соседними уровнями на примере молекулы \ce{N2}, выразить в см$^{-1}$, если вращательная постоянная $B = 2.0$ см$^{-1}$
\end{itemize}

По всей видимости имеются в виду синглетные термы, для которых мы можем записать:

\begin{equation}
	E_r = \frac{\hbar^2}{2 I} J(J+1) = B J(J+1)
	\label{eq:E_r}
\end{equation}

Здесь $I$ --- момент инерции, $J$ --- вращательное квантовое число, $\hbar$ --- постоянная Дирака.


Так как $B$ уже в см$^{-1}$, переводить ничего никуда не надо, и мы уже будем сразу получать ответ в см$^{-1}$.

Тогда для перехода:

\begin{equation}
	\Delta E_{J\rightarrow J-1} = BJ(J+1) - B(J-1)J = BJ[J + 1 - J + 1] = 2BJ
	\label{eq:transition}
\end{equation}

В таком случае расстояние между уровнями равно $2B = 4$~см$^{-1}$ (для $J = 1$, например). 

\section*{Задача 3}

Для двухатомной молекулы с массами ядер $m_1$ и $m_2$ найти температуру, при которой средняя кинетическая энергия поступательного движения равна энергии возбуждения вращательного уровня с $J = 5$. Рассчитать ее для молекулы \ce{CO}, у которой межъядерное расстояние $d= 0.113$~нм.

Очевидно, что центр инерции будет находиться на расстояниях $l_1$ от ядра массы $m_1$ и $l_2$ от ядра массы $m_2$, где:

\begin{equation}
	\frac{l_1}{l_2} = \frac{m_2}{m_1}, \quad l_1 + l_2 = d \qrq l_1 = \frac{m_2}{m_1 + m_2}d, \quad l_2 = \frac{m_1}{m_1 + m_2} d
\end{equation}

Тогда для момента инерции относительно этой оси можем записать:

\begin{equation}
	I = \frac{m_1 m_2}{m_1 + m_2} d^2
\end{equation}

Из \ref{eq:E_r} получаем, что:

\begin{equation}
	E_r = \frac{\hbar^2}{2I} J(J+1) = E_T = \frac{3}{2}kT \qrq T = \frac{\hbar^2 J(J+1)}{3k} \cdot \frac{m_1 + m_2}{m_1 m_2 d^2} \approx 55 \text{ К}
\end{equation}

\section*{Задача 4}

С ростом $J$ энергия уровня растет квадратично. Это значит, что переходы между соседними уровнями с ростом $J$ тоже растут. Это значит, что соседние линии означают соседние переходы. Тогда из \ref{eq:transition} мы можем записать систему (с учетом $B$ в см$^{-1}$):

\begin{equation}
	\begin{cases*}
		\dfrac{1}{\lambda_1} = 2 B (J + 1)\\
		\dfrac{1}{\lambda_2} = 2 B J
	\end{cases*} \qrq  B = \frac{1}{2}\left(\frac{1}{\lambda_1} - \frac{1}{\lambda_2}\right) \approx 10.7 \text{ см$^{-1}$} %\frac{\lambda_2}{\lambda_1} = \frac{J + 1}{J} = \frac{4}{3} \qrq J_1 = 3, \quad J_2 = 2
\end{equation}

Кроме того, для переходов:

\begin{equation}
	\frac{\lambda_2}{\lambda_1} = \frac{J + 1}{J} = \frac{4}{3} \qrq J = 3
\end{equation}

Это значит, что $\lambda_1$ и $\lambda_2$ соответствуют переходам $4\rightarrow 3$ и $3\rightarrow 2$ соответственно.

\section*{Задача 5}

Через кювету с газом двухатомных молекул пропускают монохроматический лазерный пучок, длина волны которого настроена на колебательно-вращательный переход с сечением поглощения $\sigma =  10^{-18}$ см$^2$. Для молекул с колебательной частотой $\nu = 1000$ см$^{-1}$ и вращательной постоянной $B = 1.5$ см$^{-1}$ для нижнего уровня с $v = 0$, $J=15$ оценить:
\begin{itemize}
	\item Долю молекул на этом уровне при температуре 300 К;
	
	\item Коэффициент поглощения газа при давлении 20 мбар;
	
	\item Мощность лазерного излучения, прошедшего кювету длиной 15 см, при падающей мощности 50 мВт.
\end{itemize}

\subsection*{Доля молекул}

Доля молекул будет определяться распределением Гиббса. В числителе будет стоять (с учетом того, что мы рассматриваем 15-ый уровень, а $v = 0$):

\begin{equation}
	(2 \cdot 15 + 1) \exp\left(-\frac{15 B(15 + 1)}{kT}\right) \exp\left(-\frac{\nu}{2 k T}\right)
\end{equation}

В знаменателе будет стоять статистическая сумма, которую предлагается считать следующим образом:

\begin{align*}
	Z = \sum_{J, v = 0}^\infty (2 J + 1) \exp\left(-\frac{BJ(J + 1)}{kT}\right) \exp\left(-\frac{\nu (v + 1/2)}{kT}\right) =\\
	= \exp\left(-\frac{\nu}{2kT}\right) \cdot \frac{1}{1 - \exp\left(-\dfrac{\nu}{kT}\right)} \int\limits_0^\infty dJ \; (2J + 1) \exp\left(-\frac{BJ(J + 1)}{kT}\right) = \frac{kT}{2B \sinh\left(\dfrac{\nu}{2 k T}\right)}
\end{align*}

Итого доля молекул:

\begin{equation}
	\Omega = \frac{31 \exp\left(\dfrac{-240 B}{kT}\right) \exp\left(-\dfrac{\nu}{2 k T}\right)}{kT} \cdot 2 B \sinh\left(\frac{\nu}{2 k T}\right)
\end{equation}

Энергия, которую мы получим, будет в см$^{-1}$ (т.к $B$ в этих единицах), поэтому надо $kT$ перевести в те же единицы, для этого:

\begin{equation}
	kT \quad   \longrightarrow \quad \frac{kT}{hc}
\end{equation}

Тогда доля равна:

\begin{equation}
	\Omega \approx 1.43 \%
\end{equation}

\subsection*{Коэффициент поглощения}

\begin{equation}
	p = nkT \qrq n = \frac{p}{kT} \qrq n_J =  \frac{\Omega p}{kT} \qrq \alpha = n_J \sigma \approx 6.7 \cdot 10^{-3} \text{ см}^{-1}
\end{equation}

\subsection*{Мощность}

\begin{equation}
	P = P_0 \exp\left(-\alpha l\right) \approx 45.2 \text{ мВт}
\end{equation}

\section*{Задача 7}

Лазерное излучение мощностью 1 Вт периодически, в течение $10^{-2}$ с, направляется в кювету с газом длиной $l = 10$ см и объемом 50 см$^3$, с плотностью числа поглощающих молекул (сечение поглощения $\sigma = 10^{-16}$~см$^2$) $2.5\cdot 10^{14}$ см$^{-3}$. Молекулы имеют 3 поступательных и 3 вращательных степени свободы. Пренебрегая излучательными переходами, оценить амплитуды прироста давления в кювете и сигнала микрофона при чувствительности микрофона 10 мВ/Па.

На один импульс излучения :

\begin{equation}
	E_0 = P \tau
\end{equation}

Здесь $E_0$ --- энергия импульса, $P$ --- мощность, $\tau$ --- время излучения.

За единицу времени их:

\begin{equation}
	\frac{dN_{reac}}{dt} = \sigma n \frac{V}{S} \frac{d N_p}{dt} 
\end{equation}

Здесь $V$ --- объем кюветы, $S$ --- площадь сечения.

Для $N_p$ мы можем записать:

\begin{equation}
	\frac{d N_p}{dt} = \frac{P}{\hbar \omega}
\end{equation}

Для поглощенной мощности:

\begin{equation}
	P = P_0 (1 - \exp [-\sigma n l])
\end{equation}

Для поглощенной энергии:

\begin{equation}
	E = P_0 (1 - \exp [-\sigma n l]) \cdot \tau
\end{equation}

Ну и наконец для изменения давления (с учетом наличия 6 степеней свободы):

\begin{equation}
	\Delta p = \frac{2}{6} \frac{E}{V} = \frac{2}{6} \frac{P_0 (1 - \exp [-\sigma n l]) \cdot \tau}{V} \approx 14.8 \text{ Па}
\end{equation}

Для сигнала микрофона тогда:

\begin{equation}
	\Delta U = 0.148 \text{ В}
\end{equation}

\section*{Задача 8}

Непрерывное возбуждающее лазерное излучение $\lambda = 515$~нм, мощность 5 Вт (\ce{Ar}$+-$лазер) фокусируется в рассеивающий объем 1 мм$^2 \times$5~мм. Сечение комбинационного рассеяния детектируемых молекул $\sigma = 10^{-30}$~см$^2$.

Эффективность сбора сдвинутого по частоте рассеиваемого излучения на ФЭУ - 15\%, квантовая эффективность фотокатода ФЭУ - 20\%, темновой ток ФЭУ - 10 фотоэлектр/с.

Оценить минимально обнаружимую концентрацию детектируемых молекул [мол/см$^3$], положив минимальное отношение сигнал/шум равным 2.

Темновой ток 10 фотоэлектр/с, т.е. с учетом соотношения сигнал/шум на фотокатод поступает 100 фотоэлектр/с при минимальной обнаружимой концентрации.

На рассеивающий объем попадает мощность:

\begin{equation}
	P = \frac{dN_f}{dt} h\nu = \frac{dN_f}{dt} \frac{hc}{\lambda} \qrq \frac{dN_f}{dt} = \frac{\lambda P}{h c}
\end{equation}

Для рассеянных:

\begin{equation}
	\frac{dN_d}{dt} = \sigma n V j_f = \sigma n \frac{V}{S} \frac{dN_f}{dt} = \left[\frac{V}{S} = l = 5 \text{ мм}\right] = \sigma n l \frac{dN_f}{dt} = \sigma n l \frac{\lambda P}{h c}
\end{equation}

С другой стороны, согласно условия, мы можем записать:

\begin{equation}
	\frac{dN_d}{dt} = \frac{100}{15\%} \text{ фотоэлектр/с}
\end{equation}

Тогда для концентрации мы можем записать:

\begin{equation}
	n = \frac{dN_d}{dt} \frac{hc}{\sigma l \lambda P} \approx 10^{14} \text{ м}^{-3}
\end{equation}

\end{document}