%
%Не забыть:
%--------------------------------------
%Вставить колонтитулы, поменять название на титульнике



%--------------------------------------

\documentclass[a4paper, 12pt]{article} 

%--------------------------------------
%Russian-specific packages
%--------------------------------------
%\usepackage[warn]{mathtext}
\usepackage[T2A]{fontenc}
\usepackage[utf8]{inputenc}
\usepackage[english,russian]{babel}
\usepackage[intlimits]{amsmath}
\usepackage{esint}
%--------------------------------------
%Hyphenation rules
%--------------------------------------
\usepackage{hyphenat}
\hyphenation{ма-те-ма-ти-ка вос-ста-нав-ли-вать}
%--------------------------------------
%Packages
%--------------------------------------
\usepackage{amsmath}
\usepackage{amssymb}
\usepackage{amsfonts}
\usepackage{amsthm}
\usepackage{latexsym}
\usepackage{mathtools}
\usepackage{etoolbox}%Булевые операторы
\usepackage{extsizes}%Выставление произвольного шрифта в \documentclass
\usepackage{geometry}%Разметка листа
\usepackage{indentfirst}
\usepackage{wrapfig}%Создание обтекаемых текстом объектов
\usepackage{fancyhdr}%Создание колонтитулов
\usepackage{setspace}%Настройка интерлиньяжа
\usepackage{lastpage}%Вывод номера последней страницы в документе, \lastpage
\usepackage{soul}%Изменение параметров начертания
\usepackage{hyperref}%Две строчки с настройкой гиперссылок внутри получаеммого
\usepackage[usenames,dvipsnames,svgnames,table,rgb]{xcolor}% pdf-документа
\usepackage{multicol}%Позволяет писать текст в несколько колонок
\usepackage{cite}%Работа с библиографией
\usepackage{subfigure}% Человеческая вставка нескольких картинок
\usepackage{tikz}%Рисование рисунков
\usepackage{float}% Возможность ставить H в положениях картинки
% Для картинок Моти
\usepackage{misccorr}
\usepackage{lscape}
\usepackage{cmap}



\usepackage{graphicx,xcolor}
\graphicspath{{Pictures/}}
\DeclareGraphicsExtensions{.pdf,.png,.jpg}

%----------------------------------------
%Список окружений
%----------------------------------------
\newenvironment {theor}[2]
{\smallskip \par \textbf{#1.} \textit{#2}  \par $\blacktriangleleft$}
{\flushright{$\blacktriangleright$} \medskip \par} %лемма/теорема с доказательством
\newenvironment {proofn}
{\par $\blacktriangleleft$}
{$\blacktriangleright$ \par} %доказательство
%----------------------------------------
%Список команд
%----------------------------------------
\newcommand{\grad}
{\mathop{\mathrm{grad}}\nolimits\,} %градиент

\newcommand{\diver}
{\mathop{\mathrm{div}}\nolimits\,} %дивергенция

\newcommand{\rot}
{\ensuremath{\mathrm{rot}}\,}

\newcommand{\Def}[1]
{\underline{\textbf{#1}}} %определение

\newcommand{\RN}[1]
{\MakeUppercase{\romannumeral #1}} %римские цифры

\newcommand {\theornp}[2]
{\textbf{#1.} \textit{ #2} \par} %Написание леммы/теоремы без доказательства

\newcommand{\qrq}
{\ensuremath{\quad \Rightarrow \quad}} %Человеческий знак следствия

\newcommand{\qlrq}
{\ensuremath{\quad \Leftrightarrow \quad}} %Человеческий знак равносильности

\renewcommand{\phi}{\varphi} %Нормальный знак фи

\newcommand{\me}
{\ensuremath{\mathbb{E}}}

\newcommand{\md}
{\ensuremath{\mathbb{D}}}

\newcommand{\med}[1]
{\ensuremath{\langle#1\rangle}}



%\renewcommand{\vec}{\overline}




%----------------------------------------
%Разметка листа
%----------------------------------------
\geometry{top = 3cm}
\geometry{bottom = 2cm}
\geometry{left = 1.5cm}
\geometry{right = 1.5cm}
%----------------------------------------
%Колонтитулы
%----------------------------------------
\pagestyle{fancy}%Создание колонтитулов
\fancyhead{}
%\fancyfoot{}
%----------------------------------------
%Интерлиньяж (расстояния между строчками)
%----------------------------------------
%\onehalfspacing -- интерлиньяж 1.5
%\doublespacing -- интерлиньяж 2
%----------------------------------------
%Настройка гиперссылок
%----------------------------------------
\hypersetup{				% Гиперссылки
	unicode=true,           % русские буквы в раздела PDF
	pdftitle={Заголовок},   % Заголовок
	pdfauthor={Автор},      % Автор
	pdfsubject={Тема},      % Тема
	pdfcreator={Создатель}, % Создатель
	pdfproducer={Производитель}, % Производитель
	pdfkeywords={keyword1} {key2} {key3}, % Ключевые слова
	colorlinks=true,       	% false: ссылки в рамках; true: цветные ссылки
	linkcolor=blue,          % внутренние ссылки
	citecolor=blue,        % на библиографию
	filecolor=magenta,      % на файлы
	urlcolor=cyan           % на URL
}
%----------------------------------------
%Работа с библиографией (как бич)
%----------------------------------------
\renewcommand{\refname}{Список литературы}%Изменение названия списка литературы для article
%\renewcommand{\bibname}{Список литературы}%Изменение названия списка литературы для book и report
%----------------------------------------
\begin{document}
	\begin{titlepage}
		\begin{center}
			$$$$
			$$$$
			$$$$
			$$$$
			{\Large{НАЦИОНАЛЬНЫЙ ИССЛЕДОВАТЕЛЬСКИЙ УНИВЕРСИТЕТ}}\\
			\vspace{0.1cm}
			{\Large{ВЫСШАЯ ШКОЛА ЭКОНОМИКИ}}\\
			\vspace{0.25cm}
			{\large{Факультет физики}}\\
			\vspace{5.5cm}
			{\Huge\textbf{{Домашняя работа 1}}}\\%Общее название
			\vspace{1cm}
			{\LARGE{Лазерная спектроскопия}}\\%Точное название
			\vspace{2cm}
			{Работу выполнил студент 3 курса}\\
			{Захаров Сергей Дмитриевич}\\
			\vfill
			\includegraphics[width = 0.2\textwidth]{HSElogo}\\
			\vfill
			Москва\\
			2021
		\end{center}
	\end{titlepage}
	
\tableofcontents


\newpage

\section{Задача 1}

\textit{Длина волны излучения $\lambda = 400$~нм; посчитать: волновой вектор $k$~[см$^{-1}$]; частоту $\nu$~[см$^{-1}$].}

\subsection{Волновой вектор}

\begin{equation}
	k = \frac{2\pi}{\lambda} = 157080 \text{ см$^{-1}$}
\end{equation}

\subsection{Частота}

\begin{equation}
	\nu = \frac{1}{\lambda} = 25000 \text{ см$^{-1}$}
\end{equation}

\section{Задача 2}

\textit{Ширина аппаратной функции спектрографа в области $\lambda = 500$~нм составляет \mbox{$\Delta \nu = 1.6$~см$^{-1}$}; какова эта ширина в длинах волн $\Delta \lambda$~[нм]?}

Переведем длину волны в частоту:

\begin{equation}
	\nu = \frac{1}{\lambda}
\end{equation}

Найдем границы интервала в частотах:

\begin{equation}
	\nu_1 = \frac{1}{\lambda} - \frac{\Delta \nu}{2} = \frac{2 - \lambda \Delta \nu}{2 \lambda}; \quad \nu_2 = \frac{1}{\lambda} + \frac{\Delta \nu}{2} = \frac{2 + \lambda \Delta \nu}{2 \lambda}
\end{equation}

Переведем частоты обратно в длины волн:

\begin{equation}
	\lambda_1 = \frac{2 \lambda}{2 - \lambda \Delta\nu}; \quad \lambda_2 = \frac{2 \lambda}{2 + \lambda \Delta\nu}
\end{equation}

Тогда ширина:

\begin{equation}
	\Delta \lambda = \lambda_1 - \lambda_2 = \frac{4\lambda^2 \Delta\nu}{4 - \lambda^2 \Delta\nu^2} \approx \lambda^2 \Delta \nu = 0.04 \text{ см$^{-1}$}
\end{equation}

Ну как-то немного.

\section{Задача 3}

\textit{Возбужденное состояние молекулы кислорода расположено на 1.62 эВ над основным; каковы: частота излучения $\nu$ при переходе в основное состояние [Гц]; длина волны этого излучения~[нм]?}

\subsection{Частота}

\begin{equation}
	\Delta E = h c \nu \qrq \nu = \frac{\Delta E}{h c} \approx 13056.1 \text{ см$^{-1}$}
\end{equation}

\subsection{Длина волны}

\begin{equation}
	\lambda = \frac{1}{\nu} = \frac{c h}{\Delta E} \approx 765.9 \text{ нм} 
\end{equation}

\section{Задача 4}

\textit{Для спектра излучения абсолютно черного тела записать выражения для: среднего число фотонов в моде с частотой $\nu$ излучения, объемной плотности числа мод с частотой $\nu$ в интервале $d\nu$, объемной плотности энергии излучения на частоте $\nu$ в интервале $d\lambda$ при температуре $T$}.

\subsection{Среднее число фотонов в моде}

Статистическое распределение ансамбля по квантовым ячейкам (они же моды колебания) с учетом нормировки:

\begin{equation}
	W(n) = \left(1 - \exp \left[-\frac{h \nu}{k T}\right]\right) \cdot \exp \left[-\frac{n h \nu}{k T}\right]
\end{equation}

Тогда определим среднее значение количества фотонов $\med{n}$:

\begin{equation}
	\med{n} = \sum\limits_{n=0}^\infty n W(n) = \frac{1}{\exp\left[\dfrac{h \nu}{k T}\right] - 1}
\end{equation}

\subsection{Объемная плотность числа мод}

Внимательно посмотрим на то, что мы получали в лекции:

''Плотность числа мод в полости в интервале частот от $\nu$ до $\nu + d\nu$''. Ответом на указанный вопрос будет то, что было получено в лекции, домноженное на $d\nu$:

\begin{equation}
	p_\nu = \frac{1}{V} \frac{d N}{d\nu} = \frac{8 \pi \nu^2}{c_n^3} d\nu
	\label{eq:mode_dens}
\end{equation}

Здесь $c_n$ --- скорость света в веществе.

\subsection{Объемная плотность энергии излучения}

Ситуация аналогична указанной в предыдущем пункте. 

\begin{equation}
	\rho = \rho_\nu d\nu =  \frac{8 \pi \nu^2}{c^3} \frac{h\nu}{\exp\left[\dfrac{h \nu }{k T}\right] - 1} d\nu
\end{equation}

где $d\nu'$ определяется из $d\lambda$ как в задаче 2.

\section{Задача 5}

\textit{Посчитать объемную плотность числа мод излучения в максимуме спектра АЧТ с температурой 5000~К, имеющих длины волн в пределах полосы шириной $d\lambda = 10$~нм.}

Запишем закон смещения Вина (опустим индекс $_\text{max}$):

\begin{equation}
	\lambda = \frac{b}{T}
\end{equation}

Чтобы воспользуемся формулой \ref{eq:mode_dens} и не изобретать велосипед, переведем все в частоты!

\begin{equation}
	\nu_2 = \frac{c}{\lambda} = \frac{c T}{b}; \quad \nu_1 = \frac{c}{\lambda + d\lambda} = \frac{c T}{b + T d\lambda} \qrq d\nu = \frac{c T^2 d\lambda}{b^2}
\end{equation}

Подставим в \ref{eq:mode_dens} (предполагая, что $n=1$):

\begin{equation}
	p_\nu =  \frac{8 \pi \nu^2}{c^3} d\nu = \frac{8\pi T^3}{c b^3} d\lambda \approx 4.3 \cdot 10^{-9} \text{ м$^{-3} \cdot$ с}
\end{equation}

\section{Задача 6}

\textit{Записать соотношение между спектральными плотностями энергии излучения $\rho_\nu$ и $\rho_\lambda$ ($\rho_\lambda d\lambda$ – объемная плотность энергии излучения с длинами волн от $\lambda$ до $\lambda + d\lambda$).}

%В лекции мы получали выражение для $\rho_\nu$:

%\begin{equation}
%	\rho_\nu = \frac{8 \pi \nu^2}{c^3} \frac{h \nu}{\exp\left[\dfrac{h \nu}{k T}\right]}
%\end{equation}

%Аналогичным способом получим $\rho_\lambda$:

%\begin{eqnarray}
%	\rho_\lambda = p_\lambda \med{E} = \frac{1}{V} \frac{d N(\lambda)}{d \lambda} = \frac{1}{V} \left(\frac{d}{d\lambda} \left[\frac{8 \pi}{3\lambda^3} V\right]\right) = \frac{8\pi}{\lambda^4} \frac{hc /\lambda}{\exp\left[\dfrac{h c}{\lambda k T}\right] - 1}
%\end{eqnarray}

%Переводя все, что можно, в одни величины и беря отношение, получим:

%\begin{equation}
%	\frac{\rho_\nu}{\rho_\lambda} = \frac{\nu^3 \lambda^5 }{c^4} \cdot  \frac{\exp\left[\dfrac{h c}{k T \lambda}\right] - 1}{\exp\left[\dfrac{h \nu}{k T}\right] - 1} = \frac{\lambda}{\nu}
%\end{equation}

Понятно, что если интервалы по частотам и по длинам волн относятся к одному участку спектра, то доля светимости, выраженная через $\rho$, есть объективная реальность и не зависит от представления, т.е. выполняется:

\begin{equation}
	\rho_\nu d\nu = \rho_\lambda d\lambda
\end{equation}

Запишем связь частоты и длины волны:

\begin{equation}
	\lambda = \frac{c}{\nu} \qrq d\lambda = -\frac{c}{\nu^2} d\nu
\end{equation}

Минус здесь связан с тем, что для соответствия знака нужно, чтобы пределы в интегралах с участием $\rho$ были зеркальные (в смысле чтобы верхний предел в $\rho_\lambda$ соответствовал нижнему для $\rho_\nu$ и наоборот), и нам не особо интересен, так что его заметем под ковер.

Таким образом:

\begin{equation}
	\frac{\rho_\nu}{\rho_\lambda} = \frac{c}{\nu^2} = \frac{\lambda}{\nu}
\end{equation}

\section{Задача 9}

Оперировать будем в основном материалом из лекций.

\begin{equation}
	\omega^2 = \omega_0^2 \left(1 + \left[\frac{\lambda z}{\pi \omega_0^2}\right]^2\right) \qrq \left(\frac{\omega}{2}\right)^2 = \left(\frac{\omega_0}{2}\right)^2 \left(1 + \left[\frac{\lambda z}{\pi \omega_0^2}\right]^2\right)
\end{equation}

Здесь $\omega$ --- диаметр перетяжки, $z$ --- расстояние от перетяжки пучка вдоль оси.

\begin{equation}
	R = z \left(1 + \left[\frac{\pi \omega_0^2}{\lambda z}\right]^ 2\right)
\end{equation}

Здесь $R$ --- радиус кривизны.

\begin{equation}
	\frac{\partial R}{\partial z} = 1 - \frac{\pi^2 \omega_0^4}{\lambda^2 z^2} = 0 \qrq z = \pm \frac{\pi \omega^2}{\lambda} = \pm z_R
\end{equation}

Здесь $z_R$ --- 

Тогда при $z \rightarrow \infty$:

\begin{equation}
	\omega^2 \approx \omega_0^2 \left(\frac{\lambda z}{\pi \omega^2}\right)^2 \qrq \omega \approx \omega_0 \frac{\lambda z}{\pi\omega_0^2} = \frac{\lambda z}{\pi \omega_0}
\end{equation}

Тогда для радиуса кривизны:

\begin{equation}
	R = z + \frac{\pi^2\omega_0^4}{\lambda^2 z} \approx z \quad \text{ т.к. $z\rightarrow \infty$ происходит расходимость}
\end{equation}


\newpage

\begin{thebibliography}{}
	
\end{thebibliography}

\end{document}