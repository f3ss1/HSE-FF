%
%Не забыть:
%--------------------------------------
%Вставить колонтитулы, поменять название на титульнике



%--------------------------------------

\documentclass[a4paper, 12pt]{article} 

%--------------------------------------
%Russian-specific packages
%--------------------------------------
%\usepackage[warn]{mathtext}
\usepackage[T2A]{fontenc}
\usepackage[utf8]{inputenc}
\usepackage[english,russian]{babel}
\usepackage[intlimits]{amsmath}
\usepackage{esint}
%--------------------------------------
%Hyphenation rules
%--------------------------------------
\usepackage{hyphenat}
\hyphenation{ма-те-ма-ти-ка вос-ста-нав-ли-вать}
%--------------------------------------
%Packages
%--------------------------------------
\usepackage{amsmath}
\usepackage{amssymb}
\usepackage{amsfonts}
\usepackage{amsthm}
\usepackage{latexsym}
\usepackage{mathtools}
\usepackage{etoolbox}%Булевые операторы
\usepackage{extsizes}%Выставление произвольного шрифта в \documentclass
\usepackage{geometry}%Разметка листа
\usepackage{indentfirst}
\usepackage{wrapfig}%Создание обтекаемых текстом объектов
\usepackage{fancyhdr}%Создание колонтитулов
\usepackage{setspace}%Настройка интерлиньяжа
\usepackage{lastpage}%Вывод номера последней страницы в документе, \lastpage
\usepackage{soul}%Изменение параметров начертания
\usepackage{hyperref}%Две строчки с настройкой гиперссылок внутри получаеммого
\usepackage[usenames,dvipsnames,svgnames,table,rgb]{xcolor}% pdf-документа
\usepackage{multicol}%Позволяет писать текст в несколько колонок
\usepackage{cite}%Работа с библиографией
\usepackage{subfigure}% Человеческая вставка нескольких картинок
\usepackage{tikz}%Рисование рисунков
\usepackage{float}% Возможность ставить H в положениях картинки
% Для картинок Моти
\usepackage{misccorr}
\usepackage{lscape}
\usepackage{cmap}
\usepackage{empheq}
\newcommand*\widefbox[1]{\fbox{\hspace{2em}#1\hspace{2em}}}



\usepackage{graphicx,xcolor}
\graphicspath{{Pictures/}}
\DeclareGraphicsExtensions{.pdf,.png,.jpg}

%----------------------------------------
%Список окружений
%----------------------------------------
\newenvironment {theor}[2]
{\smallskip \par \textbf{#1.} \textit{#2}  \par $\blacktriangleleft$}
{\flushright{$\blacktriangleright$} \medskip \par} %лемма/теорема с доказательством
\newenvironment {proofn}
{\par $\blacktriangleleft$}
{$\blacktriangleright$ \par} %доказательство
%----------------------------------------
%Список команд
%----------------------------------------
\newcommand{\grad}
{\mathop{\mathrm{grad}}\nolimits\,} %градиент

\newcommand{\diver}
{\mathop{\mathrm{div}}\nolimits\,} %дивергенция

\newcommand{\rot}
{\ensuremath{\mathrm{rot}}\,}

\newcommand{\Def}[1]
{\underline{\textbf{#1}}} %определение

\newcommand{\RN}[1]
{\MakeUppercase{\romannumeral #1}} %римские цифры

\newcommand {\theornp}[2]
{\textbf{#1.} \textit{ #2} \par} %Написание леммы/теоремы без доказательства

\newcommand{\qrq}
{\ensuremath{\quad \Rightarrow \quad}} %Человеческий знак следствия

\newcommand{\qlrq}
{\ensuremath{\quad \Leftrightarrow \quad}} %Человеческий знак равносильности

\renewcommand{\phi}{\varphi} %Нормальный знак фи

\newcommand{\me}
{\ensuremath{\mathbb{E}}}

\newcommand{\md}
{\ensuremath{\mathbb{D}}}

\newcommand{\bra}[1]
{\ensuremath{\left\langle#1\right|}}

\newcommand{\cat}[1]
{\ensuremath{\left|#1\right\rangle}}



%\renewcommand{\vec}{\overline}




%----------------------------------------
%Разметка листа
%----------------------------------------
\geometry{top = 3cm}
\geometry{bottom = 2cm}
\geometry{left = 1.5cm}
\geometry{right = 1.5cm}
%----------------------------------------
%Колонтитулы
%----------------------------------------
\pagestyle{fancy}%Создание колонтитулов
\fancyhead{}
%\fancyfoot{}
%\fancyhead[R]{\textsc{Уравнения Лондонов. Кинетическая индуктивность сверхпроводников.}}%Вставить колонтитул сюда
%----------------------------------------
%Интерлиньяж (расстояния между строчками)
%----------------------------------------
%\onehalfspacing -- интерлиньяж 1.5
%\doublespacing -- интерлиньяж 2
%----------------------------------------
%Настройка гиперссылок
%----------------------------------------
\hypersetup{				% Гиперссылки
	unicode=true,           % русские буквы в раздела PDF
	pdftitle={Заголовок},   % Заголовок
	pdfauthor={Автор},      % Автор
	pdfsubject={Тема},      % Тема
	pdfcreator={Создатель}, % Создатель
	pdfproducer={Производитель}, % Производитель
	pdfkeywords={keyword1} {key2} {key3}, % Ключевые слова
	colorlinks=true,       	% false: ссылки в рамках; true: цветные ссылки
	linkcolor=blue,          % внутренние ссылки
	citecolor=blue,        % на библиографию
	filecolor=magenta,      % на файлы
	urlcolor=red           % на URL
}
%----------------------------------------
%Работа с библиографией (как бич)
%----------------------------------------
\renewcommand{\refname}{Список литературы}%Изменение названия списка литературы для article
%\renewcommand{\bibname}{Список литературы}%Изменение названия списка литературы для book и report
%----------------------------------------
\begin{document}
	\begin{titlepage}
		\begin{center}
			$$$$
			$$$$
			$$$$
			$$$$
			{\Large{НАЦИОНАЛЬНЫЙ ИССЛЕДОВАТЕЛЬСКИЙ УНИВЕРСИТЕТ}}\\
			\vspace{0.1cm}
			{\Large{ВЫСШАЯ ШКОЛА ЭКОНОМИКИ}}\\
			\vspace{0.25cm}
			{\large{Факультет физики}}\\
			\vspace{5.5cm}
			{\Huge\textbf{{Домашнее задание}}}\\%Общее название
			\vspace{1cm}
			{\LARGE{Квантовая механика, неделя 17}}\\%Точное название
			\vspace{2cm}
			{Задание выполнил студент 2 курса}\\
			{Захаров Сергей Дмитриевич}
			\vfill
			\includegraphics[width = 0.2\textwidth]{HSElogo}\\
			\vfill
			Москва\\
			2020
		\end{center}
	\end{titlepage}

\section*{Задача 1}

\subsection*{a)}

Спиновая часть гамильтониана представима в виде:

\begin{equation*}
	\hat{H} = - \mu \theta(t) B \hat{\sigma}_x
\end{equation*}

Здесь $\theta(t)$ --- функция Хевисайда.

Представим спиновую волновую функцию в следующем виде:

\begin{equation*}
	\Psi(t) = 
	\begin{pmatrix}
		a(t) \\
		b(t)
	\end{pmatrix}
\end{equation*}

В таком случае волновое уравнение можно переписать в следующем виде:

\begin{equation}
	i \hbar \frac{\partial}{\partial t} \Psi(t) = \hat{H} \Psi(t) \qlrq 
	\begin{cases*}
		i \hbar \dfrac{\partial}{\partial t} a(t)  = - \mu \theta(t) B b(t)\\
		i \hbar \dfrac{\partial}{\partial t} b(t)  = -\mu \theta(t) B a(t)
	\end{cases*}
\end{equation}

Если принять, что $t \ge 0$, то данная система в целом без особых проблем решается и мы получаем:

\begin{equation*}
	\begin{cases*}
	a(t) = C_1 \cdot e^{i B t \mu / \hbar} + C_2 \cdot e^{-i B t \mu / \hbar}\\
	b(t) = C_1 \cdot e^{i B t \mu / \hbar} - C_2 \cdot e^{-i B t \mu / \hbar}
	\end{cases*}
\end{equation*}

Нам задано, что в начальный момент спин направлен вдоль оси $z$. Это означает, что в начальный момент спиновая волновая функция была равна:

\begin{align*}
	\Psi(0) = 
	\begin{pmatrix}
	1 \\
	0
	\end{pmatrix} 
	\qrq 
	\begin{cases*}
		C_1 + C_2 = 1\\
		C_1 - C_2 = 0
	\end{cases*}
	\qlrq 
	\Psi(t) = 
	\begin{pmatrix}
		\left(e^{i B t \mu / \hbar} + e^{-i B t \mu / \hbar}\right)/2\\
		\left(e^{i B t \mu / \hbar} - e^{-i B t \mu / \hbar}\right)/2
	\end{pmatrix}
	\qlrq \\
	\qlrq \boxed{\Psi = 
	\begin{pmatrix}
		\cos\left(\dfrac{B\mu}{\hbar} t\right)\\
		i \sin\left(\dfrac{B\mu}{\hbar} t\right)
	\end{pmatrix}}
\end{align*}

\subsection*{б)}

Чтобы найти среднее, воспользуемся привычным нам выражением для среднего:

\begin{equation*}
	s_z(t) = \bra{\Psi}\hat{s}_z\cat{\Psi} = \frac{1}{2} 
	\begin{pmatrix}
		\cos \left(\omega t\right) & -i \sin\left(\omega t\right)
	\end{pmatrix}
	\begin{pmatrix}
		1 & 0\\
		0 & -1
	\end{pmatrix}
	\begin{pmatrix}
		\cos \left(\omega t\right)\\
		i \sin\left(\omega t\right)
	\end{pmatrix}
	= \boxed{\frac{1}{2} \cos\left(2\omega t\right) = s_z(t)}
\end{equation*}

Здесь $\omega = B \mu/\hbar$.

Аналогичным образом найдем и средние операторов $s_x$ и $s_y$:

\begin{empheq}[box=\widefbox]{align}
	s_x(t) &= 0\\
	s_y(t) &= -\frac{1}{2}\sin\left(2\omega t\right)
\end{empheq}

Получаем, что средний вектор спина прецессирует вокруг оси $\mathbf{x}$ с постоянной угловой скоростью $2\omega$.

\subsection*{в)}

Оператор эволюции имеет вид:

\begin{equation*}
	\hat{U} = \exp\left(-\frac{i}{\hbar}\hat{H} t\right)
\end{equation*}

С учетом имеющегося у нас гамильтониана, его можно переписать в виде (сразу предполагаем $t > 0$):

\begin{equation*}
	\hat{U} = \exp\left(\frac{i \mu B}{\hbar} \hat{\sigma}_x t\right) = \exp\left(i \omega t \sigma_x \right)
\end{equation*}

В таком случае, для перехода в представление Гайзенберга совершим следующее:

\begin{align*}
	\hat{s}_z(t) = \frac{1}{2} \hat{U}^\dagger \sigma_z \hat{U} = \frac{1}{2} \left[\left\{1 - i\omega t \sigma_x\ + \frac{1}{2!}(i \omega t)^2 \sigma_x^2 - ...\right\} \cdot \sigma_z \cdot \left\{1 + i\omega t \sigma_x\ + \frac{1}{2!}(i \omega t)^2 \sigma_x^2 + ...\right\} \right] = \\
	= \left[\sigma_z - i\omega t [\sigma_z,\sigma_x]\right]
\end{align*} 





\end{document}