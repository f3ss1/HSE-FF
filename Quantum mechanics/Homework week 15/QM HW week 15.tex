%
%Не забыть:
%--------------------------------------
%Вставить колонтитулы, поменять название на титульнике



%--------------------------------------

\documentclass[a4paper, 12pt]{article} 

%--------------------------------------
%Russian-specific packages
%--------------------------------------
%\usepackage[warn]{mathtext}
\usepackage[T2A]{fontenc}
\usepackage[utf8]{inputenc}
\usepackage[english,russian]{babel}
\usepackage[intlimits]{amsmath}
\usepackage{esint}
%--------------------------------------
%Hyphenation rules
%--------------------------------------
\usepackage{hyphenat}
\hyphenation{ма-те-ма-ти-ка вос-ста-нав-ли-вать}
%--------------------------------------
%Packages
%--------------------------------------
\usepackage{amsmath}
\usepackage{amssymb}
\usepackage{amsfonts}
\usepackage{amsthm}
\usepackage{latexsym}
\usepackage{mathtools}
\usepackage{etoolbox}%Булевые операторы
\usepackage{extsizes}%Выставление произвольного шрифта в \documentclass
\usepackage{geometry}%Разметка листа
\usepackage{indentfirst}
\usepackage{wrapfig}%Создание обтекаемых текстом объектов
\usepackage{fancyhdr}%Создание колонтитулов
\usepackage{setspace}%Настройка интерлиньяжа
\usepackage{lastpage}%Вывод номера последней страницы в документе, \lastpage
\usepackage{soul}%Изменение параметров начертания
\usepackage{hyperref}%Две строчки с настройкой гиперссылок внутри получаеммого
\usepackage[usenames,dvipsnames,svgnames,table,rgb]{xcolor}% pdf-документа
\usepackage{multicol}%Позволяет писать текст в несколько колонок
\usepackage{cite}%Работа с библиографией
\usepackage{subfigure}% Человеческая вставка нескольких картинок
\usepackage{tikz}%Рисование рисунков
\usepackage{float}% Возможность ставить H в положениях картинки
% Для картинок Моти
\usepackage{misccorr}
\usepackage{lscape}
\usepackage{cmap}



\usepackage{graphicx,xcolor}
\graphicspath{{Pictures/}}
\DeclareGraphicsExtensions{.pdf,.png,.jpg}

%----------------------------------------
%Список окружений
%----------------------------------------
\newenvironment {theor}[2]
{\smallskip \par \textbf{#1.} \textit{#2}  \par $\blacktriangleleft$}
{\flushright{$\blacktriangleright$} \medskip \par} %лемма/теорема с доказательством
\newenvironment {proofn}
{\par $\blacktriangleleft$}
{$\blacktriangleright$ \par} %доказательство
%----------------------------------------
%Список команд
%----------------------------------------
\newcommand{\grad}
{\mathop{\mathrm{grad}}\nolimits\,} %градиент

\newcommand{\diver}
{\mathop{\mathrm{div}}\nolimits\,} %дивергенция

\newcommand{\rot}
{\ensuremath{\mathrm{rot}}\,}

\newcommand{\Def}[1]
{\underline{\textbf{#1}}} %определение

\newcommand{\RN}[1]
{\MakeUppercase{\romannumeral #1}} %римские цифры

\newcommand {\theornp}[2]
{\textbf{#1.} \textit{ #2} \par} %Написание леммы/теоремы без доказательства

\newcommand{\qrq}
{\ensuremath{\quad \Rightarrow \quad}} %Человеческий знак следствия

\newcommand{\qlrq}
{\ensuremath{\quad \Leftrightarrow \quad}} %Человеческий знак равносильности

\renewcommand{\phi}{\varphi} %Нормальный знак фи

\newcommand{\me}
{\ensuremath{\mathbb{E}}}

\newcommand{\md}
{\ensuremath{\mathbb{D}}}



%\renewcommand{\vec}{\overline}




%----------------------------------------
%Разметка листа
%----------------------------------------
\geometry{top = 3cm}
\geometry{bottom = 2cm}
\geometry{left = 1.5cm}
\geometry{right = 1.5cm}
%----------------------------------------
%Колонтитулы
%----------------------------------------
\pagestyle{fancy}%Создание колонтитулов
\fancyhead{}
%\fancyfoot{}
\fancyhead[R]{\textsc{Уравнения Лондонов. Кинетическая индуктивность сверхпроводников.}}%Вставить колонтитул сюда
%----------------------------------------
%Интерлиньяж (расстояния между строчками)
%----------------------------------------
%\onehalfspacing -- интерлиньяж 1.5
%\doublespacing -- интерлиньяж 2
%----------------------------------------
%Настройка гиперссылок
%----------------------------------------
\hypersetup{				% Гиперссылки
	unicode=true,           % русские буквы в раздела PDF
	pdftitle={Заголовок},   % Заголовок
	pdfauthor={Автор},      % Автор
	pdfsubject={Тема},      % Тема
	pdfcreator={Создатель}, % Создатель
	pdfproducer={Производитель}, % Производитель
	pdfkeywords={keyword1} {key2} {key3}, % Ключевые слова
	colorlinks=true,       	% false: ссылки в рамках; true: цветные ссылки
	linkcolor=blue,          % внутренние ссылки
	citecolor=blue,        % на библиографию
	filecolor=magenta,      % на файлы
	urlcolor=red           % на URL
}
%----------------------------------------
%Работа с библиографией (как бич)
%----------------------------------------
\renewcommand{\refname}{Список литературы}%Изменение названия списка литературы для article
%\renewcommand{\bibname}{Список литературы}%Изменение названия списка литературы для book и report
%----------------------------------------
\begin{document}
	\begin{titlepage}
		\begin{center}
			$$$$
			$$$$
			$$$$
			$$$$
			{\Large{НАЦИОНАЛЬНЫЙ ИССЛЕДОВАТЕЛЬСКИЙ УНИВЕРСИТЕТ}}\\
			\vspace{0.1cm}
			{\Large{ВЫСШАЯ ШКОЛА ЭКОНОМИКИ}}\\
			\vspace{0.25cm}
			{\large{Факультет физики}}\\
			\vspace{5.5cm}
			{\Huge\textbf{{Домашнее задание}}}\\%Общее название
			\vspace{1cm}
			{\LARGE{Квантовая механика, неделя 15}}\\%Точное название
			\vspace{2cm}
			{Задание выполнил студент 2 курса}\\
			{Захаров Сергей Дмитриевич}
			\vfill
			\includegraphics[width = 0.2\textwidth]{HSElogo}\\
			\vfill
			Москва\\
			2020
		\end{center}
	\end{titlepage}

\section*{Задача 1}

Для основного состояния атома водорода известна волновая функция (в размерных единицах):

\begin{equation}
	\psi_0 (r) = \frac{e^{-r / a_B}}{\sqrt{\pi a^3_B}}
	\label{eq:psi_0}
\end{equation}

Здесь $a_B$ --- боровский радиус.

В таком случае:

\begin{align*}
	\langle r^2 \rangle = \frac{1}{\pi a_B^3} \int\limits_{0}^{\infty} dr \int\limits_{0}^{\pi} d\theta \int\limits_{0}^{2\pi} d\phi \; e^{-r / a_B} \cdot r^2 \cdot e^{-r / a_B} \cdot r^2 \cdot \sin \theta  = \frac{4}{a^3_B} \int\limits_0^\infty dr \; r^4 \cdot e^{-2r / a_B} = \\
	= \frac{4}{a_B^3} \cdot \frac{4! a_B^5}{2^5} = \boxed{3 a_B^2}\\
	\left \langle \frac{1}{r} \right \rangle = \frac{1}{\pi a_B^3} \int\limits_{0}^{\infty} dr \int\limits_{0}^{\pi} d\theta \int\limits_{0}^{2\pi} d\phi \; e^{-r / a_B} \cdot \frac{1}{r} \cdot e^{-r / a_B} \cdot r^2 \cdot \sin \theta = \frac{4}{a_B^3} \int\limits_0^\infty dr \; r \cdot e^{-2r / a_B} = \boxed{\frac{1}{a_B}}\\
	\langle \hat{\mathbf{p}}^2 \rangle = \frac{1}{\pi a_B^3} \int\limits_{0}^{\infty} dr \int\limits_{0}^{\pi} d\theta \int\limits_{0}^{2\pi} d\phi \; e^{-r / a_B} \cdot \left(-\hbar^2 \cdot  \frac{1}{r} \frac{\partial^2}{\partial r^2} \left[r e^{-r / a_B}\right]\right)  \cdot r^2 \cdot \sin \theta = \\
	= \frac{4 \hbar^2}{a_B^3} \int\limits_0^\infty dr \; e^{-r/a_B}\left(\frac{r e^{-r/a_B}}{a_B^2}-\frac{2 e^{-r/a_B}}{a_B}\right) \cdot r = \boxed{\frac{\hbar^2}{a_B^2у}}
\end{align*}

Последние интегралы берутся по частям, кроме того, они широко известны и посчитаны, например, на Wiki, поэтому приводить их взятие нужным не считаю.
	
\section*{Задача 2}

Вне ядра потенциал, создаваемый им, совпадает с кулоновским и равняется $\phi_o = e / r$. Внутри же ядра создается потенциал $\phi_i$, который отличается от кулоновского потенциала на величину

\begin{equation}
	\delta \phi = \phi_i - e / r
	\label{eq:delta_phi}
\end{equation}

Это отличие в свою очередь будет определять интересующее нас возмущение, которое задается формулой $V(r) = - e \delta \phi (r)$.

Согласно теории возмущений, искомый нами сдвиг по энергии будет определяться формулой:

\begin{equation*}
	\Delta E_{1s} = \int\limits_V dV \; \psi_0^\dagger(r) V(r) \psi_0(r)
\end{equation*}

С учетом того, что радиус ядра много меньше радиуса атома Бора, мы можем заменить в интеграле $\psi_0(r)$ на $\psi_0(0)$, т.к. она мало меняется. В таком случае мы получим следующее:

\begin{equation}
	\Delta E_{1s} \approx - e \psi_0^{\dagger}(0)\psi_0(0) \int \limits_V dV \; \delta \phi 
	\label{eq:delta_E}
\end{equation}

Чтобы взять последний интеграл воспользуемся фокусом: $\Delta r^2 = 6$. Тогда мы можем переписать:

\begin{equation*}
	\int\limits_V dV\; \delta \phi = \frac{1}{6} \int\limits_V dV \; \delta\phi \Delta r^2 = \frac{1}{6} \int\limits_V dV \; r^2 \Delta \delta\phi 
\end{equation*}

Переход к последнему равенству произведен по частям.

Расписав $\delta\phi$ по формуле (\ref{eq:delta_phi}) получим:

\begin{equation*}
	\Delta\delta \phi = -4\pi\rho + 4\pi e \delta^3(r) = -4\pi (\rho - \delta^3(r))
\end{equation*}

Здесь использовано уравнение Пуассона $\Delta \phi_i = -4\pi\rho$ и лапласиан функции $1/r$: \mbox{$\Delta 1/r = -4\pi\delta^3(r)$}. $\rho$ --- объемная плотность заряда, в нашем случае она равна:

\begin{equation*}
	\rho = \frac{e}{4/3 \pi r_0^3} = \frac{3e}{4\pi r_0^3}
\end{equation*}

Подставляя в интеграл:

\begin{equation*}
	-\frac{1}{6} \int\limits_V dV \; 4\pi r^2 (\rho - \delta^3(r)) = -\frac{e}{2r_0^3}\int\limits_V dV \; r^2 = -\frac{e}{2r_0^3} \int\limits_{0}^{r_0} dr \int\limits_{0}^{\pi} d\theta \int\limits_{0}^{2\pi} d\phi \; r^4 \sin\theta = - \frac{2\pi r_0^2 e}{5}
\end{equation*}

Теперь, согласно формуле (\ref{eq:delta_E}) и подставляя $\psi_0(0)$ из формулы (\ref{eq:psi_0}):

\begin{equation*}
	\Delta E_{1s} = \frac{2\pi r_0^2 e^2}{5} \cdot \frac{1}{\pi a_B^3} = \boxed{\frac{2r_0^2 e^2}{5 a_B^3}}
\end{equation*}

\section*{Задача 3}
	
\end{document}