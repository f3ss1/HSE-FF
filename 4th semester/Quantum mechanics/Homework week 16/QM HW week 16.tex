%
%Не забыть:
%--------------------------------------
%Вставить колонтитулы, поменять название на титульнике



%--------------------------------------

\documentclass[a4paper, 12pt]{article} 

%--------------------------------------
%Russian-specific packages
%--------------------------------------
%\usepackage[warn]{mathtext}
\usepackage[T2A]{fontenc}
\usepackage[utf8]{inputenc}
\usepackage[english,russian]{babel}
\usepackage[intlimits]{amsmath}
\usepackage{esint}
%--------------------------------------
%Hyphenation rules
%--------------------------------------
\usepackage{hyphenat}
\hyphenation{ма-те-ма-ти-ка вос-ста-нав-ли-вать}
%--------------------------------------
%Packages
%--------------------------------------
\usepackage{amsmath}
\usepackage{amssymb}
\usepackage{amsfonts}
\usepackage{amsthm}
\usepackage{latexsym}
\usepackage{mathtools}
\usepackage{etoolbox}%Булевые операторы
\usepackage{extsizes}%Выставление произвольного шрифта в \documentclass
\usepackage{geometry}%Разметка листа
\usepackage{indentfirst}
\usepackage{wrapfig}%Создание обтекаемых текстом объектов
\usepackage{fancyhdr}%Создание колонтитулов
\usepackage{setspace}%Настройка интерлиньяжа
\usepackage{lastpage}%Вывод номера последней страницы в документе, \lastpage
\usepackage{soul}%Изменение параметров начертания
\usepackage{hyperref}%Две строчки с настройкой гиперссылок внутри получаеммого
\usepackage[usenames,dvipsnames,svgnames,table,rgb]{xcolor}% pdf-документа
\usepackage{multicol}%Позволяет писать текст в несколько колонок
\usepackage{cite}%Работа с библиографией
\usepackage{subfigure}% Человеческая вставка нескольких картинок
\usepackage{tikz}%Рисование рисунков
\usepackage{float}% Возможность ставить H в положениях картинки
% Для картинок Моти
\usepackage{misccorr}
\usepackage{lscape}
\usepackage{cmap}
\usepackage{empheq}
\newcommand*\widefbox[1]{\fbox{\hspace{2em}#1\hspace{2em}}}



\usepackage{graphicx,xcolor}
\graphicspath{{Pictures/}}
\DeclareGraphicsExtensions{.pdf,.png,.jpg}

%----------------------------------------
%Список окружений
%----------------------------------------
\newenvironment {theor}[2]
{\smallskip \par \textbf{#1.} \textit{#2}  \par $\blacktriangleleft$}
{\flushright{$\blacktriangleright$} \medskip \par} %лемма/теорема с доказательством
\newenvironment {proofn}
{\par $\blacktriangleleft$}
{$\blacktriangleright$ \par} %доказательство
%----------------------------------------
%Список команд
%----------------------------------------
\newcommand{\grad}
{\mathop{\mathrm{grad}}\nolimits\,} %градиент

\newcommand{\diver}
{\mathop{\mathrm{div}}\nolimits\,} %дивергенция

\newcommand{\rot}
{\ensuremath{\mathrm{rot}}\,}

\newcommand{\Def}[1]
{\underline{\textbf{#1}}} %определение

\newcommand{\RN}[1]
{\MakeUppercase{\romannumeral #1}} %римские цифры

\newcommand {\theornp}[2]
{\textbf{#1.} \textit{ #2} \par} %Написание леммы/теоремы без доказательства

\newcommand{\qrq}
{\ensuremath{\quad \Rightarrow \quad}} %Человеческий знак следствия

\newcommand{\qlrq}
{\ensuremath{\quad \Leftrightarrow \quad}} %Человеческий знак равносильности

\renewcommand{\phi}{\varphi} %Нормальный знак фи

\newcommand{\me}
{\ensuremath{\mathbb{E}}}

\newcommand{\md}
{\ensuremath{\mathbb{D}}}

\newcommand{\bra}[1]
{\ensuremath{\left\langle#1\right|}}

\newcommand{\cat}[1]
{\ensuremath{\left|#1\right\rangle}}



%\renewcommand{\vec}{\overline}




%----------------------------------------
%Разметка листа
%----------------------------------------
\geometry{top = 3cm}
\geometry{bottom = 2cm}
\geometry{left = 1.5cm}
\geometry{right = 1.5cm}
%----------------------------------------
%Колонтитулы
%----------------------------------------
\pagestyle{fancy}%Создание колонтитулов
\fancyhead{}
%\fancyfoot{}
%\fancyhead[R]{\textsc{Уравнения Лондонов. Кинетическая индуктивность сверхпроводников.}}%Вставить колонтитул сюда
%----------------------------------------
%Интерлиньяж (расстояния между строчками)
%----------------------------------------
%\onehalfspacing -- интерлиньяж 1.5
%\doublespacing -- интерлиньяж 2
%----------------------------------------
%Настройка гиперссылок
%----------------------------------------
\hypersetup{				% Гиперссылки
	unicode=true,           % русские буквы в раздела PDF
	pdftitle={Заголовок},   % Заголовок
	pdfauthor={Автор},      % Автор
	pdfsubject={Тема},      % Тема
	pdfcreator={Создатель}, % Создатель
	pdfproducer={Производитель}, % Производитель
	pdfkeywords={keyword1} {key2} {key3}, % Ключевые слова
	colorlinks=true,       	% false: ссылки в рамках; true: цветные ссылки
	linkcolor=blue,          % внутренние ссылки
	citecolor=blue,        % на библиографию
	filecolor=magenta,      % на файлы
	urlcolor=red           % на URL
}
%----------------------------------------
%Работа с библиографией (как бич)
%----------------------------------------
\renewcommand{\refname}{Список литературы}%Изменение названия списка литературы для article
%\renewcommand{\bibname}{Список литературы}%Изменение названия списка литературы для book и report
%----------------------------------------
\begin{document}
	\begin{titlepage}
		\begin{center}
			$$$$
			$$$$
			$$$$
			$$$$
			{\Large{НАЦИОНАЛЬНЫЙ ИССЛЕДОВАТЕЛЬСКИЙ УНИВЕРСИТЕТ}}\\
			\vspace{0.1cm}
			{\Large{ВЫСШАЯ ШКОЛА ЭКОНОМИКИ}}\\
			\vspace{0.25cm}
			{\large{Факультет физики}}\\
			\vspace{5.5cm}
			{\Huge\textbf{{Домашнее задание}}}\\%Общее название
			\vspace{1cm}
			{\LARGE{Квантовая механика, неделя 16}}\\%Точное название
			\vspace{2cm}
			{Задание выполнил студент 2 курса}\\
			{Захаров Сергей Дмитриевич}
			\vfill
			\includegraphics[width = 0.2\textwidth]{HSElogo}\\
			\vfill
			Москва\\
			2020
		\end{center}
	\end{titlepage}

\section*{Задача 1}

Находить значение скалярного произведения будем с помощью рассмотрения $\hat{\mathbf{S}}^2$:

\begin{equation}
	\hat{\mathbf{S}}^2 = \hat{\mathbf{s}}_1^2 + \hat{\mathbf{s}}_2^2 + 2 \cdot  \hat{\mathbf{s}}_1 \cdot \hat{\mathbf{s}}_2 \qrq \hat{\mathbf{s}}_1 \cdot \hat{\mathbf{s}}_2 = \frac{1}{2} \left(\hat{\mathbf{S}}^2 - \hat{\mathbf{s}}_1^2 - \hat{\mathbf{s}}_2^2\right)
	\label{eq:t1_result}
\end{equation}

Для $\hat{\mathbf{s}}^2$ мы можем посчитать собственное значение как $s \cdot (s + 1)$, где $s$ --- собственное значение оператора $\hat{\mathbf{s}}$. Зная, что спины равны $1/2$  мы сразу же получаем, что собственные числа $\hat{\mathbf{s}}_1^2$ и $\hat{\mathbf{s}}_2^2$ равны $3/4$. 

Теперь необходимо определить собственное число оператора $\hat{\mathbf{S}}^2$. Для этого сперва найдем собственное число оператора $\hat{\mathbf{S}} = \hat{\mathbf{s}}_1 + \hat{\mathbf{s}}_2$. Спины могут быть либо сонаправлены (тогда они сложатся и собственное число будет равно 1), либо противоположно направлены (тогда собственное число будет равно 0). На основании этого делаем вывод, что собственные числа $\hat{\mathbf{S}}^2$ по уже указанной формуле равны либо $0$, либо $2$.

На основании этого, с помощью формулы (\ref{eq:t1_result}) получаем:

\begin{equation*}
	\boxed{
	\left[ \begin{array}{lr}
	\hat{\mathbf{s}}_1 \cdot \hat{\mathbf{s}}_2 = -\dfrac{3}{4}\\
	\hat{\mathbf{s}}_1 \cdot \hat{\mathbf{s}}_2 = \dfrac{1}{4}\\
	\end{array}
	\right.}
\end{equation*}

\section*{Задача 2}
\subsection*{а)}

Если мы будем рассматривать спиноры относительно оси $\mathbf{n}$, то, очевидным образом, они будут следующими:

\begin{equation*}
	\alpha_{1\mathbf{n}} = 
	\begin{pmatrix}
		1\\
		0
	\end{pmatrix}
	\quad
	\alpha_{1\mathbf{n}} = 
	\begin{pmatrix}
		0\\
		1
	\end{pmatrix}
\end{equation*} 

При этом, спиноры относительно оси $\mathbf{z}$ и $\mathbf{n}$ связаны следующим образом:

\begin{equation}
	\alpha_{\mathbf{n}} = \hat{U} \alpha_\mathbf{z}
	\label{eq:main_2}
\end{equation}

где в силу условия мы можем утверждать, что $\hat{U} = \hat{U}_z \hat{U}_y$. В таком случае:

\begin{align*}
	\hat{U} = \hat{U}_y \hat{U}_z = 
	\begin{pmatrix}
		\cos\theta/2 & \sin\theta/2 \\
		-\sin\theta/2 & \cos\theta/2
	\end{pmatrix}
	\cdot
	\begin{pmatrix}
		\exp(i\phi/2) & 0 \\
		0 & \exp(-i\phi/2)
	\end{pmatrix}
	= \\
	=
	\begin{pmatrix}
		\exp(i\phi/2) \cos\theta/2 & \exp(-i \phi/2) \sin\theta/2 \\
		-\exp(i\phi/2) \sin\theta/2 & \exp(-i\phi/2) \cos\theta/2
	\end{pmatrix}
\end{align*}

Теперь необходимо найти обратную матрицу к этой. В силу простоты этих вычислений (обращение матрицы 2x2), приведу сразу ответ:

\begin{equation*}
	\hat{U}^{-1} = 
	\begin{pmatrix}
		\exp(-i\phi/2) \cos\theta/2 & -\exp(-i\phi/2) \sin\theta/2 \\
		\exp(i\phi/2) \sin\theta/2 & \exp(i\phi/2) \cos\theta/2
	\end{pmatrix}
\end{equation*}

Тогда по формуле $\ref{eq:main_2}$ с учетом ее домножения слева на $\hat{U}^{-1}$:

\begin{empheq}[box=\widefbox]{align*}
	\alpha_{1\mathbf{z}} &= \hat{U}^{-1} \alpha_{1\mathbf{n}} = 
	\begin{pmatrix}
		\exp(-i\phi/2) \cos\theta/2 \\
		\exp(i\phi/2) \sin\theta/2
	\end{pmatrix}\\
	\alpha_{2\mathbf{z}} &= \hat{U}^{-1} \alpha_{2\mathbf{n}} = 
	\begin{pmatrix}
	-\exp(-i\phi/2) \sin\theta/2 \\
	\exp(i\phi/2) \cos\theta/2
	\end{pmatrix}
\end{empheq}

\subsection*{б)}

Для начала сперва запишем выражения для вектора $\mathbf{n}$ через углы $\theta$ и $\phi$:

\begin{equation*}
	\mathbf{n} = 
	\begin{pmatrix}
		\sin\theta \cos\phi \\
		\sin\theta \sin\phi \\
		\cos\theta
	\end{pmatrix}
	\label{eq:n}
\end{equation*}

С учетом выражения для оператора спина через матрицы Паули мы в таком случае можем сказать, что матрица оператора проекции спина будет следующей:

\begin{align*}
	\frac{1}{2}\cdot \left[\sin\theta \cos\phi \cdot 
	\begin{pmatrix}
		0 & 1 \\
		1 & 0 
	\end{pmatrix}
	+ \sin\theta \sin\phi \cdot 
	\begin{pmatrix}
		0 & -i \\
		i & 0
	\end{pmatrix}
	+ \cos\theta \cdot
	\begin{pmatrix}
		1 & 0\\
		0 & -1
	\end{pmatrix}\right]
	= \\
	= \frac{1}{2} \cdot
	\begin{pmatrix}
		\cos\theta & \exp(-i \phi)\sin\theta \\
		\exp(i \phi)\sin\theta & -\cos\theta
	\end{pmatrix}
\end{align*}

Для того, чтобы найти интересующие нас спиноры, достаточно найти собственные векторы этой матрицы (в силу того, что проекции спина в проекции на необходимую нам ось составляют 1/2 и $-1/2$). Задача в целом тривиальная, поэтому сразу ответ:

\begin{empheq}[box=\widefbox]{align}
	\alpha_{1\mathbf{z}} &=
	\begin{pmatrix}
		\exp(-i \phi) \cdot \cos\theta/2 \\
		\sin\theta/2
	\end{pmatrix}
	= \exp(-i\phi/2) \cdot 
	\begin{pmatrix}
		\exp(-i\phi/2) \cos\theta/2 \\
		\exp(i\phi/2) \sin\theta/2
	\end{pmatrix}\\
	\alpha_{2\mathbf{z}} &= 
	\begin{pmatrix}
		-\exp(- i \phi) \cdot \sin\theta/2 \\
		\cos\theta/2
	\end{pmatrix}
	= \exp(-i\phi/2) \cdot
	\begin{pmatrix}
		-\exp(-i\phi/2) \sin\theta/2 \\
		\exp(i\phi/2) \cos\theta/2
	\end{pmatrix}
\end{empheq}

Удивительно, но способы дают один и тот же ответ (ну с точностью до фазового множителя).

\section*{Задача 3}
\subsection*{а)}

Сперва будем работать с случаем, когда проекция на ось $\mathbf{x}$ равна $\pm 1/2$. Для этого предположим, что спинор в базисе оси $\mathbf{z}$ выглядит так:

\begin{equation}
	\alpha_{1\mathbf{z}} = 
	\begin{pmatrix}
		a \\
		b
	\end{pmatrix}
	\label{eq:spinor}
\end{equation}

С учетом известных выражений для оператора конечных вращений вокруг оси $\mathbf{y}$ (а мы вращаем именно вокруг нее на угол $\pi/2$), получаем:

\begin{equation*}
	U_y = \frac{1}{\sqrt{2}}
	\begin{pmatrix}
		1 & 1\\
		-1 & 1
	\end{pmatrix}
\end{equation*}

Теперь применим этот оператор к спинору (\ref{eq:spinor}):

\begin{equation}
	\alpha_{1\mathbf{x}} = U_y \alpha_{1\mathbf{z}} = \frac{1}{\sqrt{2}} 
	\begin{pmatrix}
		a + b \\
		b - a
	\end{pmatrix}
	\label{eq:left}
\end{equation}

С другой стороны, если мы рассматриваем случай, когда проекция спина на ось $\mathbf{x}$ равна 1/2, спинор в базисе $\mathbf{x}$ записывается так:

\begin{equation}
	\alpha_{1\mathbf{x}} = 
	\begin{pmatrix}
		1\\
		0
	\end{pmatrix}
	\label{eq:right}
\end{equation}

Таким образом из \ref{eq:left} и \ref{eq:right}, получаем систему:

\begin{equation}
	\begin{cases}
		a + b = \sqrt{2}\\
		b - a = 0
	\end{cases}
	\qlrq 	\boxed{\alpha_{1\mathbf{z}} = 
	\begin{pmatrix}
	a \\
	b
	\end{pmatrix}
	= 
	\begin{pmatrix}
		1/\sqrt{2}\\
		1/\sqrt{2}
	\end{pmatrix}}
\end{equation}

По аналогии действуем в случае, когда проекция равна $-1/2$:

\begin{equation}
	\alpha_{2\mathbf{x}} = 
	\begin{pmatrix}
	1\\
	0
	\end{pmatrix}
	\qlrq \boxed{\alpha_{2\mathbf{z}} = 
	\begin{pmatrix}
	-1/\sqrt{2}\\
	1/\sqrt{2}
	\end{pmatrix}}
\end{equation}

Те же размышления верны и для работы с осью $\mathbf{y}$, однако тут оператор конечных вращений будет браться вокруг оси $\mathbf{x}$ и окажется равен:

\begin{equation*}
	U_x = \frac{1}{\sqrt{2}}
	\begin{pmatrix}
		1 & i\\
		i & 1
	\end{pmatrix}
\end{equation*}

В таком случае для проекции 1/2:

\begin{equation}
	\alpha_{1\mathbf{y}} = 
	\begin{pmatrix}
	1\\
	0
	\end{pmatrix}
	\qlrq \boxed{\alpha_{1\mathbf{z}} = 
	\begin{pmatrix}
	 1 / \sqrt{2}\\
	 -i / \sqrt{2}
	\end{pmatrix}}
\end{equation}

А для проекции $-1/2$:

\begin{equation}
	\alpha_{2\mathbf{y}} = 
	\begin{pmatrix}
	0\\
	1
	\end{pmatrix}
	\qlrq \boxed{\alpha_{2\mathbf{z}} = 
	\begin{pmatrix}
	-i / \sqrt{2}\\
	1 / \sqrt{2}
	\end{pmatrix}}
\end{equation}

\subsection*{б)}

В силу того, что мы работаем со спинорами, вероятность возможной проекции на ось $\mathbf{z}$  может быть получена как квадрат модуля той или иной компоненты спинора: для проекции $1/2$ берем верхнюю компоненту, для проекции $-1/2$ --- нижнюю в силу построения матрицы $\hat{s}$ (в базисе оси $\mathbf{z}$, разумеется). Тогда получаем, что в любом из состояний вероятности проекции $1/2$ и $-1/2$ равны между собой и равны $\boxed{w = 0.5}$ (см. спиноры выше).

%В случае проекции на ту или иную ось ($\mathbf{x}$ или $\mathbf{y}$) становится понятно, что средний вектор спина направлен ровно по этой же оси и равен, соответственно, $\pm 1/2$ в зависимости от проекции. Очевидно, что проекция этого вектора на ось $\mathbf{z}$ оказывается равной нулю. С другой стороны мы также можем сказать, что среднее значение вектора спина равно:
%
%\begin{equation*}
%	\bar{s}_{\mathbf{z}} = \frac{1}{2} \cdot (w_+ - w_-) = 0
%\end{equation*}
%
%Здесь $w_+$ и $w_-$ --- вероятности состояний со спином $1/2$ и $-1/2$ соответственно. В то же время также логично выполнение $w_+ + w_- = 1$. Таким образом, получаем, что в каждом из указанных состояний $\boxed{w_+ = w_- = 0.5}$ 
%
\section*{Задача 4}

\subsection*{а)}

Согласно формуле 55.5 из Ландау-Лифшица, имеем в общем случае:

\begin{align*}
	&(s_x)_{\sigma,\sigma-1} = (s_x)_{\sigma-1,\sigma} = \frac{1}{2} \sqrt{(s + \sigma)(s - \sigma + 1)}\\
	&(s_y)_{\sigma, \sigma-1} = (s_y)_{\sigma - 1, \sigma} = -\frac{i}{2}\sqrt{(s + \sigma) (s - \sigma + 1)}\\
	&(s_z)_{\sigma, \sigma} = \sigma
\end{align*}

В нашем случае мы имеем $s=1$ и, следовательно, $\sigma = 1, 0, -1$. В таком случае мы получаем следующие матрицы:

\begin{empheq}[box=\widefbox]{align}
	&\hat{s}_x = \frac{1}{\sqrt{2}}
	\begin{pmatrix}
		0 & 1 & 0 \\
		1 & 0 & 1  \\
		0 & 1 & 0
	\end{pmatrix}\\
	&\hat{s}_y = \frac{1}{\sqrt{2}}
	\begin{pmatrix}
		0 & -i & 0 \\
		i & 0 & -i \\
		0 & i & 0 
	\end{pmatrix}\\
	&\hat{s}_z =
	\begin{pmatrix}
		1 & 0 & 0 \\
		0 & 0 & 0 \\
		0 & 0 & -1
	\end{pmatrix}
\end{empheq}

\subsection*{б)}

По аналогии с пунктом \textbf{б} второй задачи находим матрицу оператора проекции спина. Она окажется равна:

\begin{equation}
	\frac{1}{2} \cdot 
	\begin{pmatrix}
		\cos\theta & \dfrac{1}{\sqrt{2}}\sin\theta \exp(-i\phi) & 0 \\
		\dfrac{1}{\sqrt{2}}\sin\theta \exp(i\phi) & 0 & \dfrac{1}{\sqrt{2}}\sin\theta \exp(-i\phi) \\
		0 & \dfrac{1}{\sqrt{2}}\sin\theta \exp(i\phi) & -\cos\theta
	\end{pmatrix}
\end{equation}

Ее собственные числа, как и следовало ожидать, равны $1, 0, -1$. Нас, опять же по аналогии со второй задачей, интересует собственный вектор, однако в этот раз вполне конкретный (собственное число должно быть равно 1). Делается это в целом тривиально, поэтому получаем:

\begin{equation}
	\boxed{
	\alpha_{1\mathbf{z}} = \frac{1}{2}
	\begin{pmatrix}
		\exp(-i\phi)(1 + \cos\theta)\\
		\sqrt{2} \sin\theta\\
		\exp(i\phi) (1-\cos\theta)
	\end{pmatrix}
}
\end{equation}

\subsection*{в)}

По полной аналогии с пунктом \textbf{б} третьей задачи получаем, что (не забываем про 1/2 перед вектором и вспоминаем, каким образом мы строили матрицы $\hat{s}_i$):

\begin{empheq}[box=\widefbox]{align*}
	\text{Вероятность проекции 1:}& \quad w_1 = \left|\frac{\exp(-i\phi)}{2}(1 + \cos\theta)\right|^2 = \frac{(1 + \cos\theta)^2}{4} = \cos^4 \frac{\theta}{2}\\
	\text{Вероятность проекции 0:}& \quad w_2 = \left|\frac{\sqrt{2}}{2} \sin\theta\right|^2 = \frac{\sin^2\theta}{2}\\
	\text{Вероятность проекции -1:}& \quad w_3 = \left|\frac{\exp(i\phi)}{2} (1-\cos\theta)\right|^2 = \sin^4\frac{\theta}{2}
\end{empheq}



\end{document}