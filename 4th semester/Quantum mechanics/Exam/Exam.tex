%
%Не забыть:
%--------------------------------------
%Вставить колонтитулы, поменять название на титульнике



%--------------------------------------

\documentclass[a4paper, 12pt]{article} 

%--------------------------------------
%Russian-specific packages
%--------------------------------------
%\usepackage[warn]{mathtext}
\usepackage[T2A]{fontenc}
\usepackage[utf8]{inputenc}
\usepackage[english,russian]{babel}
\usepackage[intlimits]{amsmath}
\usepackage{esint}
%--------------------------------------
%Hyphenation rules
%--------------------------------------
\usepackage{hyphenat}
\hyphenation{ма-те-ма-ти-ка вос-ста-нав-ли-вать}
%--------------------------------------
%Packages
%--------------------------------------
\usepackage{amsmath}
\usepackage{amssymb}
\usepackage{amsfonts}
\usepackage{amsthm}
\usepackage{latexsym}
\usepackage{mathtools}
\usepackage{etoolbox}%Булевые операторы
\usepackage{extsizes}%Выставление произвольного шрифта в \documentclass
\usepackage{geometry}%Разметка листа
\usepackage{indentfirst}
\usepackage{wrapfig}%Создание обтекаемых текстом объектов
\usepackage{fancyhdr}%Создание колонтитулов
\usepackage{setspace}%Настройка интерлиньяжа
\usepackage{lastpage}%Вывод номера последней страницы в документе, \lastpage
\usepackage{soul}%Изменение параметров начертания
\usepackage{hyperref}%Две строчки с настройкой гиперссылок внутри получаеммого
\usepackage[usenames,dvipsnames,svgnames,table,rgb]{xcolor}% pdf-документа
\usepackage{multicol}%Позволяет писать текст в несколько колонок
\usepackage{cite}%Работа с библиографией
\usepackage{subfigure}% Человеческая вставка нескольких картинок
\usepackage{tikz}%Рисование рисунков
\usepackage{float}% Возможность ставить H в положениях картинки
% Для картинок Моти
\usepackage{misccorr}
\usepackage{lscape}
\usepackage{cmap}
\usepackage{empheq}
\newcommand*\widefbox[1]{\fbox{\hspace{2em}#1\hspace{2em}}}



\usepackage{graphicx,xcolor}
\graphicspath{{Pictures/}}
\DeclareGraphicsExtensions{.pdf,.png,.jpg}

%----------------------------------------
%Список окружений
%----------------------------------------
\newenvironment {theor}[2]
{\smallskip \par \textbf{#1.} \textit{#2}  \par $\blacktriangleleft$}
{\flushright{$\blacktriangleright$} \medskip \par} %лемма/теорема с доказательством
\newenvironment {proofn}
{\par $\blacktriangleleft$}
{$\blacktriangleright$ \par} %доказательство
%----------------------------------------
%Список команд
%----------------------------------------
\newcommand{\grad}
{\mathop{\mathrm{grad}}\nolimits\,} %градиент

\newcommand{\diver}
{\mathop{\mathrm{div}}\nolimits\,} %дивергенция

\newcommand{\rot}
{\ensuremath{\mathrm{rot}}\,}

\newcommand{\Def}[1]
{\underline{\textbf{#1}}} %определение

\newcommand{\RN}[1]
{\MakeUppercase{\romannumeral #1}} %римские цифры

\newcommand {\theornp}[2]
{\textbf{#1.} \textit{ #2} \par} %Написание леммы/теоремы без доказательства

\newcommand{\qrq}
{\ensuremath{\quad \Rightarrow \quad}} %Человеческий знак следствия

\newcommand{\qlrq}
{\ensuremath{\quad \Leftrightarrow \quad}} %Человеческий знак равносильности

\renewcommand{\phi}{\varphi} %Нормальный знак фи

\newcommand{\me}
{\ensuremath{\mathbb{E}}}

\newcommand{\md}
{\ensuremath{\mathbb{D}}}

\newcommand{\bra}[1]
{\ensuremath{\left\langle#1\right|}}

\newcommand{\cat}[1]
{\ensuremath{\left|#1\right\rangle}}



%\renewcommand{\vec}{\overline}




%----------------------------------------
%Разметка листа
%----------------------------------------
\geometry{top = 3cm}
\geometry{bottom = 2cm}
\geometry{left = 1.5cm}
\geometry{right = 1.5cm}
%----------------------------------------
%Колонтитулы
%----------------------------------------
\pagestyle{fancy}%Создание колонтитулов
\fancyhead{}
%\fancyfoot{}
%\fancyhead[R]{\textsc{Уравнения Лондонов. Кинетическая индуктивность сверхпроводников.}}%Вставить колонтитул сюда
%----------------------------------------
%Интерлиньяж (расстояния между строчками)
%----------------------------------------
%\onehalfspacing -- интерлиньяж 1.5
%\doublespacing -- интерлиньяж 2
%----------------------------------------
%Настройка гиперссылок
%----------------------------------------
\hypersetup{				% Гиперссылки
	unicode=true,           % русские буквы в раздела PDF
	pdftitle={Заголовок},   % Заголовок
	pdfauthor={Автор},      % Автор
	pdfsubject={Тема},      % Тема
	pdfcreator={Создатель}, % Создатель
	pdfproducer={Производитель}, % Производитель
	pdfkeywords={keyword1} {key2} {key3}, % Ключевые слова
	colorlinks=true,       	% false: ссылки в рамках; true: цветные ссылки
	linkcolor=blue,          % внутренние ссылки
	citecolor=blue,        % на библиографию
	filecolor=magenta,      % на файлы
	urlcolor=red           % на URL
}
%----------------------------------------
%Работа с библиографией (как бич)
%----------------------------------------
\renewcommand{\refname}{Список литературы}%Изменение названия списка литературы для article
%\renewcommand{\bibname}{Список литературы}%Изменение названия списка литературы для book и report
%----------------------------------------
\begin{document}
	\begin{titlepage}
		\begin{center}
			$$$$
			$$$$
			$$$$
			$$$$
			{\Large{НАЦИОНАЛЬНЫЙ ИССЛЕДОВАТЕЛЬСКИЙ УНИВЕРСИТЕТ}}\\
			\vspace{0.1cm}
			{\Large{ВЫСШАЯ ШКОЛА ЭКОНОМИКИ}}\\
			\vspace{0.25cm}
			{\large{Факультет физики}}\\
			\vspace{5.5cm}
			{\Huge\textbf{{Экзамен. Квантовая механика}}}\\%Общее название
			\vspace{1cm}
			{\LARGE{Билет 5. Уравнение Шрёдингера и стационарное уравнение Шрёдингера.}}\\%Точное название
			\vspace{2cm}
			{Подготовку выполнил студент 2 курса}\\
			{Захаров Сергей Дмитриевич}
			\vfill
			\includegraphics[width = 0.2\textwidth]{HSElogo}\\
			\vfill
			Москва\\
			2020
		\end{center}
	\end{titlepage}

\section{Кто такой гамильтониан и откуда он взялся}

В классической механике для полного описания движения частицы достаточно задать координату и скорость в некоторый момент времени, чтобы описать дальнейшее движение с помощью уравнений. В квантовой механике у нас вместо координаты и скорости волновая функция, значит мы хоти знать, как меняется со временем она. То есть мы хотим, чтобы задание волновой функции в некоторый момент времени определяло (через уравнения) поведение частицы во все последующие моменты времени (естественно в том вероятностном смысле, который допускает квантовая механика). Математически это означает связь волновой функции и ее производной:

\begin{equation}
	i \hbar \frac{\partial \Psi}{\partial t} = \hat{H} \Psi
	\label{eq:shredinger}
\end{equation}

Про $\hat{H}$ мы можем пока сказать, что это какой-то линейный оператор.

Как мы уже знаем, интеграл квадрата модуля волновой функции от времени не зависит, значит:

\begin{equation*}
	\frac{\partial}{\partial t} \int |\Psi|^2 dx = \int \frac{\partial \Psi^*}{\partial t} \Psi dx + \int\Psi^* \frac{\partial \Psi}{\partial t} dx = 0
\end{equation*}

Заменяя здесь производные с помощью уравнения \ref{eq:shredinger}, получим:

\begin{equation}
	\frac{i}{\hbar}\int (\hat{H}^*\Psi^*)\Psi dx - \frac{i}{\hbar} \int \Psi^* (\hat{H}\Psi) dx \qrq \int \Psi^* \hat{H}^{*T} \Psi dx = \Psi^* \hat{H} \Psi dx
\end{equation}

$\Psi$ в общем случае произвольные, а значит $\hat{H}$ --- эрмитов.

\textbf{Эрмитов оператор} --- такой оператор, что $\hat{A}^\dagger = A$. $^\dagger$ обозначает комбинацию комплексного сопряжения и транспонирования.

Чему он соответствует? Ну, вспоминаем представление:

\begin{equation*}
	\Psi = |\Psi| e^{i\phi}, \quad \phi = \frac{S}{\hbar}
\end{equation*}

Учтем, что при переходе к классике (а мы хотим это сделать, чтобы понять, чему соответствует $\hat{H}$ в классике), при вычислении производной по времени основной вклад будет от дифференцирования быстрой фазы. Собственно, вычислим производную:

\begin{equation*}
	\frac{\partial \Psi}{\partial t} = \frac{i}{\hbar}\frac{\partial S}{\partial t} \Psi
\end{equation*}

Сравниваем теперь с \ref{eq:shredinger} и видим, что:

\begin{equation*}
	\hat{H} = -\frac{\partial S}{\partial t}
\end{equation*}

А это значит, что $H$ соответствует функции Гамильтона (полная энергия).

Уравнение \ref{eq:shredinger} --- \textbf{уравнение Шредингера}.

\section{Стационарное уравнение Шредингера и волновые функции во времени}

Уравнение на собственные значение гамильтониана (спойлер: из классической механики мы знаем, что это будет энергия) называется \textbf{стационарным уравнение Шредингера}:

\begin{equation}
	\hat{H} \cat{n} = E_n \cat{n}
	\label{eq:stationary}
\end{equation}

Пусть гамильтониан от времени не зависит, а $\cat{n}$ --- некоторая его собственная функция. Тогда из линейности ясно, что мы ее можем домножить на любой фазовый множитель. Во-вторых, из уравнения \ref{eq:shredinger} ясно, что зависимость от времени в таком случае должна иметь вид:

\begin{equation*}
	\exp\left(-i \frac{E_n}{\hbar} t\right) \cat{n}
\end{equation*}

То есть мы всю зависимость выкинули в фазовую часть. Поэтому собственные состояния гамильтониана называют стационарными — хотя волновые функции от времени зависят (только через фазовый множитель), физические величины — не зависят (фазовый множитель технически можно выбрать любой).

Имея это знание, мы можем сказать, как зависит от времени произвольная функция, которую мы задали при $t=0$, например. Разложим ее по собственных функциям гамильтониана:

\begin{equation*}
	\Psi(0) = \sum \limits_n c_n \cat{n}
\end{equation*}

Как зависят члены суммы от времени мы уже знаем, поэтому получаем просто:

\begin{equation*}
	\Psi(t) = \sum\limits_n c_n \exp\left(-i \frac{E_n}{\hbar} t\right) \cat{n}
\end{equation*}

Для нахождения коэффициентов $c_n$ нужно просто спроецировать:

\begin{equation*}
	c_n = \int \psi_n^*(x) \psi(x) dx
\end{equation*}

\newpage

\section{Вариационный принцип}

Чисто технически, стационарное уравнение Шредингера \ref{eq:stationary} можно представлять себе как результат действия вариационного принципа. Посмотрим как это получается:

\begin{equation*}
	\delta \int \psi^* (\hat{H} - E) \psi dx = 0
\end{equation*}

Функция $\psi$ комплексная. По этой причине нам нужно варьировать по двум независимым вещественным функциям. Долго не думаем, берем в качестве этих функций $\psi$ и $\psi^*$. Если мы проварьируем по $\psi^*$, то мы получим стационарное уравнение Шредингера и все. Если проварьируем по $\psi$, то получим то же самое, но комплексно сопряженное:

\begin{equation*}
	\int \psi^*(\hat{H} - E) \delta \psi dx = \int \delta \psi (\hat{H}^* - E) \psi^* dx = 0
\end{equation*}

Здесь мы пользуемся тем, что гамильтониан --- эрмитов оператор.

То есть то же самое. Удивительно.

\subsection{Запись через задачу об условном экстремуме}

Можно еще сказать, что наша задача сводится к нахождению следующего экстремума:

\begin{equation*}
	\delta \int \psi^* \hat{H} \psi dx = 0, \quad \text{при условии} \int\psi^* \psi dx = 1
\end{equation*}

В таком случае $E$ --- множитель Лагранжа. Минимальное значение, которое будет достигаться (при обозначенном условии) --- энергия основного состояние ($E_0$), функции --- волновые функции основного состояния.

\newpage

1



\end{document}