%
%Не забыть:
%--------------------------------------
%Вставить колонтитулы, поменять название на титульнике



%--------------------------------------

\documentclass[a4paper, 12pt]{article} 

%--------------------------------------
%Russian-specific packages
%--------------------------------------
%\usepackage[warn]{mathtext}
\usepackage[T2A]{fontenc}
\usepackage[utf8]{inputenc}
\usepackage[english,russian]{babel}
\usepackage[intlimits]{amsmath}
\usepackage{esint}
%--------------------------------------
%Hyphenation rules
%--------------------------------------
\usepackage{hyphenat}
\hyphenation{ма-те-ма-ти-ка вос-ста-нав-ли-вать}
%--------------------------------------
%Packages
%--------------------------------------
\usepackage{amsmath}
\usepackage{amssymb}
\usepackage{amsfonts}
\usepackage{amsthm}
\usepackage{latexsym}
\usepackage{mathtools}
\usepackage{etoolbox}%Булевые операторы
\usepackage{extsizes}%Выставление произвольного шрифта в \documentclass
\usepackage{geometry}%Разметка листа
\usepackage{indentfirst}
\usepackage{wrapfig}%Создание обтекаемых текстом объектов
\usepackage{fancyhdr}%Создание колонтитулов
\usepackage{setspace}%Настройка интерлиньяжа
\usepackage{lastpage}%Вывод номера последней страницы в документе, \lastpage
\usepackage{soul}%Изменение параметров начертания
\usepackage{hyperref}%Две строчки с настройкой гиперссылок внутри получаеммого
\usepackage[usenames,dvipsnames,svgnames,table,rgb]{xcolor}% pdf-документа
\usepackage{multicol}%Позволяет писать текст в несколько колонок
\usepackage{cite}%Работа с библиографией
\usepackage{subfigure}% Человеческая вставка нескольких картинок
\usepackage{tikz}%Рисование рисунков
\usepackage{float}% Возможность ставить H в положениях картинки
% Для картинок Моти
\usepackage{misccorr}
\usepackage{lscape}
\usepackage{cmap}

% Для Х И М И И

\usepackage{mhchem}



\usepackage{graphicx,xcolor}
\graphicspath{{Pictures/}}
\DeclareGraphicsExtensions{.pdf,.png,.jpg}

%----------------------------------------
%Список окружений
%----------------------------------------
\newenvironment {theor}[2]
{\smallskip \par \textbf{#1.} \textit{#2}  \par $\blacktriangleleft$}
{\flushright{$\blacktriangleright$} \medskip \par} %лемма/теорема с доказательством
\newenvironment {proofn}
{\par $\blacktriangleleft$}
{$\blacktriangleright$ \par} %доказательство
%----------------------------------------
%Список команд
%----------------------------------------
\newcommand{\grad}
{\mathop{\mathrm{grad}}\nolimits\,} %градиент

\newcommand{\diver}
{\mathop{\mathrm{div}}\nolimits\,} %дивергенция

\newcommand{\rot}
{\ensuremath{\mathrm{rot}}\,}

\newcommand{\Def}[1]
{\underline{\textbf{#1}}} %определение

\newcommand{\RN}[1]
{\MakeUppercase{\romannumeral #1}} %римские цифры

\newcommand {\theornp}[2]
{\textbf{#1.} \textit{ #2} \par} %Написание леммы/теоремы без доказательства

\newcommand{\qrq}
{\ensuremath{\quad \Rightarrow \quad}} %Человеческий знак следствия

\newcommand{\qlrq}
{\ensuremath{\quad \Leftrightarrow \quad}} %Человеческий знак равносильности

\renewcommand{\phi}{\varphi} %Нормальный знак фи

\newcommand{\me}
{\ensuremath{\mathbb{E}}}

\newcommand{\md}
{\ensuremath{\mathbb{D}}}



%\renewcommand{\vec}{\overline}




%----------------------------------------
%Разметка листа
%----------------------------------------
\geometry{top = 3cm}
\geometry{bottom = 2cm}
\geometry{left = 1.5cm}
\geometry{right = 1.5cm}
%----------------------------------------
%Колонтитулы
%----------------------------------------
\pagestyle{fancy}%Создание колонтитулов
\fancyhead{}
%\fancyfoot{}
%----------------------------------------
%Интерлиньяж (расстояния между строчками)
%----------------------------------------
%\onehalfspacing -- интерлиньяж 1.5
%\doublespacing -- интерлиньяж 2
%----------------------------------------
%Настройка гиперссылок
%----------------------------------------
\hypersetup{				% Гиперссылки
	unicode=true,           % русские буквы в раздела PDF
	pdftitle={Заголовок},   % Заголовок
	pdfauthor={Автор},      % Автор
	pdfsubject={Тема},      % Тема
	pdfcreator={Создатель}, % Создатель
	pdfproducer={Производитель}, % Производитель
	pdfkeywords={keyword1} {key2} {key3}, % Ключевые слова
	colorlinks=true,       	% false: ссылки в рамках; true: цветные ссылки
	linkcolor=blue,          % внутренние ссылки
	citecolor=blue,        % на библиографию
	filecolor=magenta,      % на файлы
	urlcolor=cyan           % на URL
}
%----------------------------------------
%Работа с библиографией (как бич)
%----------------------------------------
\renewcommand{\refname}{Список литературы}%Изменение названия списка литературы для article
%\renewcommand{\bibname}{Список литературы}%Изменение названия списка литературы для book и report
%----------------------------------------
\begin{document}
	\begin{titlepage}
		\begin{center}
			$$$$
			$$$$
			$$$$
			$$$$
			{\Large{НАЦИОНАЛЬНЫЙ ИССЛЕДОВАТЕЛЬСКИЙ УНИВЕРСИТЕТ}}\\
			\vspace{0.1cm}
			{\Large{ВЫСШАЯ ШКОЛА ЭКОНОМИКИ}}\\
			\vspace{0.25cm}
			{\large{Факультет физики}}\\
			\vspace{5.5cm}
			{\Huge\textbf{{Лабораторная работа}}}\\%Общее название
			\vspace{1cm}
			{\LARGE{<<Окислительно-восстановительные реакции; Синтез наночастиц \ce{Sr_{1-x}La_xF2}>>}}\\%Точное название
			\vspace{2cm}
			{Работу выполнил студент 3 курса}\\
			{Захаров Сергей Дмитриевич}
			\vfill
			\includegraphics[width = 0.2\textwidth]{HSElogo}\\
			\vfill
			Москва\\
			19 сентября 2020
		\end{center}
	\end{titlepage}

\tableofcontents

\newpage

\fancyhead[R]{\textsc{Окислительно-восстановительные реакции}}%Вставить колонтитул сюда

\section{Окислительно-восстановительные реакции}

\subsection{Опыт 1: Реакции с участием кислорода воздуха}

\subsubsection{Реактивы и оборудование}

\begin{itemize}
	\item Сухие соли: \ce{FeSO4*(NH4)2SO4*6H2O}, \ce{MnSO4}
	
	\item Раствор \ce{NaOH}
	
	\item Магниевая стружка
	
	\item Раствор фенолфталеина
	
	\item Пробирки
	\item Шпатель для реактивов
	\item Стеклянная палочка
	\item Пинцет
	\item Спиртовка
\end{itemize}


\subsubsection{Порядок выполнения}

\subsubsection*{Реакции гидроксидов металлов в промежуточной степени окисления с кислородом воздуха}

В две пробирки были помещены по одному микрошпателю сухих солей: в первую 

\noindent\ce{FeSO4*(NH4)2SO4*6H2O}, во вторую --- \ce{MnSO4}, после чего в каждую пробирку был прилит небольшой объем воды для полного растворения солей в нем. Затем в каждую пробирку был по каплям прилит раствор \ce{NaOH}.

В результате этого произошло следующее:

\begin{itemize}
	\item Раствор соли Мора (\ce{FeSO4*(NH4)2SO4*6H2O}), изначально имевший цвет морской волны, сменил цвет на лимонно-желтый. Спустя некоторое время раствор расслоился: сверху осталась бесцветная фракция, снизу -- осадок указанного цвета.
	
	\item Раствор \ce{MnSO4}, изначально имевший бледно-желтый цвет, постепенно набрал его, и стал оранжеватым. Спустя некоторое время он растворился так же, как и раствор соли Мора
\end{itemize}

Уравнение реакции с солью Мора:

\begin{equation}
	\text{аааААаА}
\end{equation}

Уравнение реакции с \ce{MnSO4}:

\begin{equation}
	\ce{MnSO4 + 2NaOH -> Mn(OH)2 + Na2SO4}
\end{equation}

%\begin{equation}
%	\ce{2MnSO4 + 4NaOH + O2 + 2H2O -> 2Mn(OH)4 + 2Na2SO4}
%\end{equation}

%\begin{equation}
%	\ce{4MnSO4 + 4NaOH + 3O2 -> 4Mn(OH)4 + 2H2O + 4Na2SO4}
%\end{equation}

\subsubsection*{Горение металлов на воздухе}

В вытяжном шкафу, с помощью спиртовки, было подожжено небольшое количество магния. Он быстро сгорел ярко-белой вспышкой, после чего продукт горения был помещен в пробирку и залит небольшим объемом воды, после чего был добавлен небольшой объем раствора фенолфталеина. Раствор окрасился в устойчивый розовый цвет, что свидетельствует о щелочной среде. Растворяется продукт реакции слабо.

С точки зрения химической реакции произошло следующее:

\begin{equation}
	\ce{2Mg + O2 -> MgO} \quad \text{--- горения магния в кислороде воздуха}
\end{equation}

\begin{equation}
	\ce{MgO + H2O -> Mg(OH)2} \quad \text{--- реакция оксида магния с водой}
\end{equation}

\subsection{Окислительные свойства дихромата калия}

\subsubsection{Реактивы и оборудование}

\begin{itemize}
	\item Сухие соли: \ce{FeSO4*(NH4)2SO4*6H2O}
	
	\item Растворы: \ce{K2Cr2O7}, \ce{HCl} (0.1M)
	
	\item Пробирки
	\item Шпатель для реактивов
	\item Стеклянная палочка
\end{itemize}

\subsubsection{Порядок выполнения работы}

В пробирку было внесено 2-3 капли раствора \ce{K2Cr2O7}, 7-8 капель 0.1M \ce{HCl}. После этого в пробирку был добавлен один микрошпатель \ce{FeSO4*(NH4)2SO4*6H2O}. Затем содержимое пробирки было тщательно перемешано до полного растворения соли.

В ходе протекания реакции окраска менялась следующим образом: сперва цвет раствора был желтым, после чего сменил цвет на цвет разбавленного медицинского йода, а затем побледнел, превратившись скорее в подобие суспензии, нежели в раствор.

Уравнение реакции:

\begin{align*}
	\ce{6FeSO4 * (NH4)2SO4 * 6H2O + K2Cr2O7 + 14 HCl -> }\\
	\ce{2 KCl + 2CrCl3 + 2 FeCl3 + 6(NH4)2SO4 + 43H2O + 2Fe2(SO4)3}
\end{align*}



%\begin{equation}
%	\ce{K2Cr2O7 + HCl -> 2CrCl3 + 2KCl + 3Cl2 + 7H2O}
%\end{equation}
%
%\begin{equation}
%	\ce{FeSO4 + 2HCl -> H2SO4 + FeCl2}
%\end{equation}

\subsection{Опыт 3: Окислительно-восстановительные свойства ионов металлов в различных степенях окисления}

\subsubsection{Реактивы и оборудование}

\begin{itemize}
	\item Сухие соли: \ce{FeCl3}, \ce{K4[Fe(CN)6]}, \ce{SnCl2}
	
	\item Пробирки
	\item Шпатель для реактивов
	\item Стеклянная палочка
\end{itemize}

\subsubsection{Порядок выполнения работы}

В пробирку было внесено 2 капли раствора \ce{FeCl3}, к которым была добавлена одна капля раствора \ce{K4[Fe(CN)6]} (желтой кровяной соли). Раствор принял темно-лазурный цвет. Уравнение реакции:

\begin{equation}
	\ce{FeCl3 + K4[Fe(CN)6] -> 3KCl + KFe[Fe(CN)6]}
\end{equation}

После этого было прилито несколько капель раствора \ce{SnCl2}. Растворе почернел,  Уравнение реакции:

Реакция восстановления \ce{FeCl3}:

\begin{equation}
	\ce{2 FeCl3 + SnCl2 -> 2FeCl2 + SnCl4}
\end{equation}

%\ce{FeSO4*(NH4)2SO4*6H2O + HCl + K2Cr2O7 -> FeCl3 + CrCl3 + KCl + H2O + H2SO4}  

\subsection{Опыт 4: Термическое разложение дихромата аммония}

\subsubsection{Реактивы и оборудование}

\begin{itemize}
	\item Сухая соль \ce{(NH4)2Cr2O7}
	
	\item Алюминиевая фольга
	
	\item Спички
\end{itemize}	

\subsubsection{Порядок выполнения работы}

Опыт демонстрационный. На алюминиевую фольгу было насыпано небольшое количество \ce{(NH4)2Cr2O7}, после чего он был подожжен с помощью спички. В результате получился "вулканчик" (Вулкан Беттгера) из дихромата аммония, который извергал из себя продукты реакции.

Уравнение реакции:

\begin{equation}
	\ce{(NH4)2Cr2O7 -> N2 + Cr2O3 + 4H2O}
\end{equation}

Черные продукты реакции, которые извергал вулкан, и есть \ce{Cr2O3}.

\subsection{Опыт 5: Влияние среды на окислительные свойства перманганата калия}

\subsubsection{Реактивы и оборудование}

\begin{itemize}
	\item Растворы: \ce{KMnO4} (0.01M), \ce{HCl} (0.1M), \ce{NaOH} (1M)
	
	\item Сухая соль: \ce{Na2SO3}
	
	\item Пробирки
	\item Шпатель для реактивов
	\item Стеклянная палочка
\end{itemize}

\subsubsection{Порядок выполнения работы}

В три пробирки было помещено 3-4 капли раствора \ce{KMnO4}. В первую пробирку было прилито 5-10 капель \ce{HCl}, во вторую --- 5-10 капель дистиллированной воды, в третью --- 5-10 капель \ce{NaOH}. После этого в каждую из пробирок было добавлено по 1 микрошпателю \ce{Na2SO3}.

В результате произошло следующее:

\begin{itemize}
	\item В 1-ой пробирке раствор обесцветился.
	
	\item Во 2-ой пробирке раствор стал бледно-желтоватым с коричневым осадком хлопьями.
	
	\item В 3-ей пробирке раствор сперва стал изумрудным, а спустя время изменил свой цвет коричневато-серый.
\end{itemize}

Уравнения реакция для каждой из пробирок следующие (сперва 1-ая, затем 2-ая и 3-я):

\begin{equation}
	\ce{5Na2SO3 + 2KMnO4 + 6HCl -> 5Na2SO4 + 2MnCl2 + 2KCl2 + 3H2O}
\end{equation}

\begin{equation}
	\ce{3Na2SO3 + 2KMnO4 + 4H2O -> 3Na2SO4 + 2MnO2 + 2KOH}
\end{equation}

\begin{equation}
	\ce{Na2SO3 + 2KMnO4 + 2KOH -> Na2SO4 + 2K2MnO4 + H2O}
\end{equation}

\subsection{Опыт 6: Окислительные и восстановительные свойства пероксида водорода}

\subsubsection{Реактивы и оборудование}

\begin{itemize}
	\item Растворы: \ce{KI} (0.05M), \ce{K2Cr2O7} (0.1M), \ce{HCl} (0.1M), \ce{H2O2} (3\%).
	
	\item Пробирки
	
	\item Пипетки
\end{itemize}

\subsubsection{Порядок выполнения работы}

\subsubsection*{a)}

В пробирку был налит 1 мл раствора \ce{KI}, который был подкислен 1-2 каплями \ce{HCl}. После этого в ту же пробирку был прилит небольшой объем \ce{H2O2}. В результате окраска сменилась с прозрачной на бурую.

Уравнение реакции:

\begin{equation}
	\ce{2 KI + 2 HCl + H2O2 -> I2 + 2 KCl + 2H2O}
\end{equation}

В данном случае окислителем выступает \ce{H2O2}, а восстановителем --- \ce{KI}.

\subsubsection*{б)}

В пробирку был налит 1 мл раствора \ce{K2Cr2O7}. который был подкислен 1-2 каплями \ce{HCl}. После этого в ту же пробирку был прилит небольшой объем \ce{H2O2}. В результате окраска сменилась с желтоватой на цвет кока-колы (раствор еще и шипель), после чего окраска стала оранжевой.

\begin{equation}
	\ce{K2Cr2O7 + 8 HCl + 3 H2O2 -> 2KCl + 2CrCl3 + 3O2 + 7H2O}
\end{equation}

В данном случае окислителем выступает \ce{K2Cr2O7}, а восстановителем --- \ce{H2O2}.

\section{Синтез наночастиц \ce{Sr_{1-x}La_xF2}}






\end{document}