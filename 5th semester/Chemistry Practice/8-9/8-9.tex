%
%Не забыть:
%--------------------------------------
%Вставить колонтитулы, поменять название на титульнике



%--------------------------------------

\documentclass[a4paper, 12pt]{article} 

%--------------------------------------
%Russian-specific packages
%--------------------------------------
%\usepackage[warn]{mathtext}
\usepackage[T2A]{fontenc}
\usepackage[utf8]{inputenc}
\usepackage[english,russian]{babel}
\usepackage[intlimits]{amsmath}
\usepackage{esint}
%--------------------------------------
%Hyphenation rules
%--------------------------------------
\usepackage{hyphenat}
\hyphenation{ма-те-ма-ти-ка вос-ста-нав-ли-вать}
%--------------------------------------
%Packages
%--------------------------------------
\usepackage{amsmath}
\usepackage{amssymb}
\usepackage{amsfonts}
\usepackage{amsthm}
\usepackage{latexsym}
\usepackage{mathtools}
\usepackage{etoolbox}%Булевые операторы
\usepackage{extsizes}%Выставление произвольного шрифта в \documentclass
\usepackage{geometry}%Разметка листа
\usepackage{indentfirst}
\usepackage{wrapfig}%Создание обтекаемых текстом объектов
\usepackage{fancyhdr}%Создание колонтитулов
\usepackage{setspace}%Настройка интерлиньяжа
\usepackage{lastpage}%Вывод номера последней страницы в документе, \lastpage
\usepackage{soul}%Изменение параметров начертания
\usepackage{hyperref}%Две строчки с настройкой гиперссылок внутри получаеммого
\usepackage[usenames,dvipsnames,svgnames,table,rgb]{xcolor}% pdf-документа
\usepackage{multicol}%Позволяет писать текст в несколько колонок
\usepackage{cite}%Работа с библиографией
\usepackage{subfigure}% Человеческая вставка нескольких картинок
\usepackage{tikz}%Рисование рисунков
\usepackage{float}% Возможность ставить H в положениях картинки
% Для картинок Моти
\usepackage{misccorr}
\usepackage{lscape}
\usepackage{cmap}

% Для Х И М И И

\usepackage{mhchem}



\usepackage{graphicx,xcolor}
\graphicspath{{Pictures/}}
\DeclareGraphicsExtensions{.pdf,.png,.jpg}

%----------------------------------------
%Список окружений
%----------------------------------------
\newenvironment {theor}[2]
{\smallskip \par \textbf{#1.} \textit{#2}  \par $\blacktriangleleft$}
{\flushright{$\blacktriangleright$} \medskip \par} %лемма/теорема с доказательством
\newenvironment {proofn}
{\par $\blacktriangleleft$}
{$\blacktriangleright$ \par} %доказательство
%----------------------------------------
%Список команд
%----------------------------------------
\newcommand{\grad}
{\mathop{\mathrm{grad}}\nolimits\,} %градиент

\newcommand{\diver}
{\mathop{\mathrm{div}}\nolimits\,} %дивергенция

\newcommand{\rot}
{\ensuremath{\mathrm{rot}}\,}

\newcommand{\Def}[1]
{\underline{\textbf{#1}}} %определение

\newcommand{\RN}[1]
{\MakeUppercase{\romannumeral #1}} %римские цифры

\newcommand {\theornp}[2]
{\textbf{#1.} \textit{ #2} \par} %Написание леммы/теоремы без доказательства

\newcommand{\qrq}
{\ensuremath{\quad \Rightarrow \quad}} %Человеческий знак следствия

\newcommand{\qlrq}
{\ensuremath{\quad \Leftrightarrow \quad}} %Человеческий знак равносильности

\renewcommand{\phi}{\varphi} %Нормальный знак фи

\newcommand{\me}
{\ensuremath{\mathbb{E}}}

\newcommand{\md}
{\ensuremath{\mathbb{D}}}



%\renewcommand{\vec}{\overline}




%----------------------------------------
%Разметка листа
%----------------------------------------
\geometry{top = 3cm}
\geometry{bottom = 2cm}
\geometry{left = 1.5cm}
\geometry{right = 1.5cm}
%----------------------------------------
%Колонтитулы
%----------------------------------------
\pagestyle{fancy}%Создание колонтитулов
\fancyhead{}
%\fancyfoot{}
%----------------------------------------
%Интерлиньяж (расстояния между строчками)
%----------------------------------------
%\onehalfspacing -- интерлиньяж 1.5
%\doublespacing -- интерлиньяж 2
%----------------------------------------
%Настройка гиперссылок
%----------------------------------------
\hypersetup{				% Гиперссылки
	unicode=true,           % русские буквы в раздела PDF
	pdftitle={Заголовок},   % Заголовок
	pdfauthor={Автор},      % Автор
	pdfsubject={Тема},      % Тема
	pdfcreator={Создатель}, % Создатель
	pdfproducer={Производитель}, % Производитель
	pdfkeywords={keyword1} {key2} {key3}, % Ключевые слова
	colorlinks=true,       	% false: ссылки в рамках; true: цветные ссылки
	linkcolor=blue,          % внутренние ссылки
	citecolor=blue,        % на библиографию
	filecolor=magenta,      % на файлы
	urlcolor=cyan           % на URL
}
%----------------------------------------
%Работа с библиографией (как бич)
%----------------------------------------
\renewcommand{\refname}{Список литературы}%Изменение названия списка литературы для article
%\renewcommand{\bibname}{Список литературы}%Изменение названия списка литературы для book и report
%----------------------------------------
\begin{document}
	\begin{titlepage}
		\begin{center}
			$$$$
			$$$$
			$$$$
			$$$$
			{\Large{НАЦИОНАЛЬНЫЙ ИССЛЕДОВАТЕЛЬСКИЙ УНИВЕРСИТЕТ}}\\
			\vspace{0.1cm}
			{\Large{ВЫСШАЯ ШКОЛА ЭКОНОМИКИ}}\\
			\vspace{0.25cm}
			{\large{Факультет физики}}\\
			\vspace{5.5cm}
			{\Huge\textbf{{Лабораторная работа}}}\\%Общее название
			\vspace{1cm}
			{\LARGE{<<Свойства переходных металлов и их соединений. Комплексные соединения переходных металлов>>}}\\%Точное название
			\vspace{2cm}
			{Работу выполнил студент 3 курса}\\
			{Захаров Сергей Дмитриевич}
			\vfill
			\includegraphics[width = 0.2\textwidth]{HSElogo}\\
			\vfill
			Москва\\
			3 октября 2020
		\end{center}
	\end{titlepage}

\tableofcontents

\newpage

\fancyhead[R]{\textsc{Свойства переходных металлов и их соединений}}%Вставить колонтитул сюда

\section{Свойства переходных металлов и их соединений}

\subsection{Опыт 1: Окислительно-восстановительные свойства соединений хрома III}

\subsubsection{Реактивы и оборудование}

\begin{itemize}
	\item Растворы: \ce{Cr(NO3)3}, \ce{NaOH} (1M), \ce{H2O2} (3\%), \ce{NaClO}
	
	\item Пробирки
	
	\item Держатель для пробирки
	
	\item Спиртовка 
\end{itemize}

\subsubsection{Порядок выполнения опыта}

В пробирку был налит 1 мл раствора \ce{Cr(NO3)3}, после чего в нее был прилит 1 мл 1M раствора \ce{NaOH} и \ce{H2O2} (3\%).

В ходе опыта окраска раствора поменялась с циановой (исходная) на изумрудную (\ce{NaOH}), после чего стала коричневой (при добавлении перекиси), а затем со временем стала желтой. 

После этого опыт был повторен, только теперь вместо перекиси использовался раствор \ce{NaClO}. В этом случае, для получения того же результата раствор пришлось нагреть над пламенем спиртовки.

\begin{equation}
	\ce{2 Cr(NO3)3 + 3 NaOH + 10 H2O2 -> 6 NaNO3 + 2 Na2CrO4 + 8 H2O} 
\end{equation}

\begin{equation}
	\ce{2 Cr(NO3)3 + 3 NaOH + 10 NaClO -> 6 NaNO3 + 2 Na2CrO4 + 5 H2O + 3NaCl} 
\end{equation}

Окислительная активность есть способность принимать электроны от других. Поэтому делаем вывод, что \ce{NaClO} более окислительно активен, чем \ce{H2O2}.

\subsection{Опыт 2: Равновесие ''хромат-дихромат'' и его зависимость от кислотности среды}

\subsubsection{Реактивы и оборудование}

\begin{itemize}
	\item Растворы: \ce{NaOH} (1M), \ce{H2SO4} (1M)
	
	\item Пробирки
\end{itemize}

\subsubsection{Порядок выполнения опыта}

К полученному в прошлом опыте раствору хромата натрия был добавлен небольшой объем 1M раствор \ce{H2SO4}. Вследствие этого желтый хромат стал коричневатого цвета с выделяющимися пузырьками.

\begin{equation}
	\ce{2Na2CrO4 + H2SO4 -> Na2Cr2O7 + Na2SO4 + H2O}
	%\ce{4NaNO3 + 2H2SO4 -> 2 Na2SO4 + O2 + 2 H2O + 4 NaO2}
\end{equation}

После этого к раствору был постепенно прилит 1M раствор \ce{NaOH}, из-за чего раствор сперва стал оранжевым, а затем и вовсе пожелтел.

\begin{equation}
	\ce{Cr2O7^2- + 2OH^- -> 2CrO4^{2-} + H2O}
\end{equation}

% В зависимости от pH раствора будет меняться равновесноя концентрация хромата и дихромата (мб цвет меняться будет?)

\subsection{Опыт 3: Окислительно-восстановительные свойства хрома и ванадия в высших степенях окисления (демонстрационный)}

\subsubsection{Реактивы и оборудование}

\begin{itemize}
	\item Раствор: \ce{K2Cr2O7}, \ce{Na3VO4}, \ce{HCl} (2M), \ce{H2SO4} (1M)
	\item \ce{Zn} (гранулы)
	\item Гексан
	
	\item Пробирки
	\item Стакан 100 мл
\end{itemize}

\subsubsection{Порядок выполнения опыта}

В пробирку было налито примерно 2 мл раствора \ce{K2Cr2O7}. После этого был добавлен 1 мл 2M \ce{HCl}, а затем несколько гранул цинка. В конце был налит слой органического растворителя (для невозможности реакции с кислородом воздуха). Раствор почернел.

%В пробирку было налито 1-2 мл раствора \ce{Na3VO4}, который был подкислен 2M \ce{HCl}. После этого в раствор был внесен металлический цинк.

% Нет реакции с выделением газа?

\subsection{Опыт 4: Разложение перманганата калия (демонстрационный)}

\subsubsection{Реактивы и оборудование}

\begin{itemize}
	\item Сухая соль: \ce{KMnO4}
	\item Раствор: \ce{NaOH} (1M)
	
	\item Пробирки
	\item Шпатель
	\item Спиртовка
	\item Лучина
\end{itemize}

\subsubsection{Порядок выполнения опыта}

В сухую пробирку были помещены кристаллы перманганата калия на кончике шпателя. Пробирка была нагрета. В пробирку была внесена тлеющая лучина, которая в момент внесения вновь загорелась от кислорода, выделяющегося в ходе реакции разложения перманганата с выделением кислорода:

\begin{equation}
	\ce{2KMnO4 -> K2MnO4 + MnO2 + O2}
\end{equation}

После этого (как только пробирка остыла), содержимое пробирки было растворено в 1M растворе \ce{NaOH}.

Реакция после добавления \ce{NaOH}: % а ее нет))

\begin{equation}
	\ce{}
\end{equation}


% ВООБЩЕ ВСЕ ПЕРЕПРОВЕРИТЬ
\subsection{Опыт 5: Химические свойства железа и меди}

\subsubsection{Реактивы и оборудование}

\begin{itemize}
	\item Растворы (конц.): \ce{HCl}, \ce{HNO3}, \ce{H2SO4}
	\item Растворы (разб.): \ce{H2SO4}

	\item \ce{Fe} (порошок)
	
	\item \ce{CuSO4}
	
	\item \ce{Cu} (проволока)
	
	\item Пробирки
	
	\item Шпатель
	
	\item Стеклянная палочка
\end{itemize}

\subsubsection{Порядок выполнения опыта}

\subsubsection*{Fe}

Описание реакции железа с:

\begin{itemize}
	\item \ce{HCl\text{(конц.)}}: раствор шипит и пенится, пробирка нагревается, образуются бесцветные кристаллы.
	
	\begin{equation}
		\ce{Fe + 2HCl(\text{конц.}) -> FeCl2 + H2}
	\end{equation}
	
	\item \ce{HNO3\text{(конц.)}}: раствор шипит, железо растворяется.
	
	\begin{equation}
		\ce{Fe + 6HNO3(\text{ конц.}) -> Fe(NO3)3 + 3NO2  + 3H2O}
	\end{equation}
	
	\item \ce{H2SO4\text{(разб.)}}: раствор шипит.
	
	\begin{equation}
		\ce{+}
	\end{equation}
	
	\item \ce{H2SO4\text{(конц.)}}: раствор едва шипит, реакция не идет (если и идет, то крайне слабо).
	
	\begin{equation}
		\ce{2Fe + 6H2SO4(\text{конц.}) -> Fe2(SO4)3 + 3SO2 + 6H2O}
	\end{equation}

\end{itemize}

После этого порошок меди был добавлен в раствор сульфата меди \ce{CuSO4}. На крупицах железа начал образовываться рыжий медный налет:

\begin{equation}
	\ce{Fe + CuSO4 -> FeSO4 + Cu}
\end{equation}


\subsubsection*{\ce{Cu}}

Описание реакции меди с:

\begin{itemize}
	\item \ce{HCl\text{(конц.)}}: ничего не происходит.
	
	% По идее нет реакции?
	%\begin{equation}
	%	\ce{2 Cu + 4 HCl(\text{конц.}) -> 2H[CuCl2] + H2}
	%\end{equation}
	
	\item \ce{HNO3\text{(конц.)}}: медь быстро растворяется, раствор зеленеет.
	
	\begin{equation}
		\ce{Cu + 4HNO3(\text{конц.}) -> Cu(NO3)2 + 2NO2 + 2H2O}
	\end{equation}

	\item \ce{H2SO4\text{(разб.)}}: медь растворяется, но медленнее.
	
	\begin{equation}
		\ce{Cu + H2SO4(\text{разб.}) -> CuSO4 + H2}
	\end{equation}
	
	\item \ce{H2SO4\text{(конц.)}}: медь растворяется, чуть быстрее.
	
	\begin{equation}
		\ce{Cu + 2H2SO4(\text{конц.}) -> CuSO4 + SO2 + H2O}
	\end{equation}
\end{itemize}


\subsection{Опыт 6: Взаимодействие цинка с растворами кислот и щелочей}

\subsubsection{Реактивы и оборудование}

\begin{itemize}
	\item Растворы: \ce{HCl} (2M), \ce{NaOH} (1M)
	
	\item \ce{Zn} (гранулы)
	
	\item Пробирки
	\item Стеклянная палочка
	\item Спиртовка
\end{itemize}

\subsubsection{Порядок выполнения опыта}

В первую пробирку была помещена гранула цинка и прилита 2M \ce{HCl}.

\begin{equation}
	\ce{Zn + 2HCl -> ZnCl2 + H2}
\end{equation}

Во вторую пробирку была также помещена гранула цинка, а добавлен был раствор \ce{NaOH}.

\begin{equation}
	\ce{Zn + 2NaOH -> Na2ZnO2 + H2}
\end{equation}

В обоих случаях было видно выделение газа (во второй пробирке процесс пошел не сразу).

Если цинк будет взаимодействовать с чем-то, содержащим водород, то цинк вытесняет его, т.е. обладает выраженными восстановительными свойствами.

\newpage

\section{Комплексные соединения переходных металлов}

\fancyhead[R]{\textsc{Комплексные соединения переходных металлов}}%Вставить колонтитул сюда

\subsection{Опыт 1: Гидроксокомплексы металлов и их свойства}

\subsubsection{Реактивы и оборудование}

\begin{itemize}
	\item Растворы: \ce{ZnCl2}, \ce{Cr(NO3)3}, \ce{NaOH} (1M)
	
	\item Пробирки
\end{itemize}

\subsubsection{Порядок выполнения опыта}

В пробирку был налит небольшой объем \ce{ZnCl2}. После этого по каплям был добавлен раствор 1M \ce{NaOH}. Сперва (при недостатке \ce{NOH}) образовывался белый осадок хлопьями. Спустя время, когда появился избыток \ce{NaOH}, осадок растворился.

После этого опыт был повторен с \ce{Cr(NO3)3}, где произошло аналогичное (сперва раствор слегка позеленел).

В случае недостатка \ce{NaOH}:

\begin{align}
	\ce{ZnCl2 + 2NaOH &-> Zn(OH)2 + 2 NaCl}\\
	\ce{Cr(NO3)3 + 3 NaOH &-> Cr(OH)3 + 3NaNO3}
\end{align}

В случае избытка \ce{NaOH}:

\begin{align}
	\ce{ZnCl2 + 4NaOH &-> Na2[Zn(OH)4] + 2NaCl}\\
	\ce{Cr(NO3)3 + 6 NaOH &-> Na3[Cr(OH)6] + 3NaNO3}
\end{align}

Осадок растворяется в щелочах, а также, как сказано в указании, в кислотах. По этой причине относим \ce{Zn(OH)2} и \ce{Cr(OH)3} к амфотерным.



\subsection{Опыт 2: Получение катионных комплексов}

\subsubsection{Реактивы и оборудование}

\begin{itemize}
	\item Растворы: \ce{NH3}, \ce{Ni(NO3)2}, \ce{CuSO4}, \ce{CoCl3}, \ce{NaOH} (1M)
	
	\item Пробирки
\end{itemize}

\subsubsection{Порядок выполнения опыта}

В пробирку было внесено несколько капель раствора \ce{Ni(NO3)2}, а также раствор 1M \ce{NaOH} до момента образования бело-зеленоватого осадка (\ce{Ni(OH)2}). После этого к осадку было добавлено несколько капель раствора \ce{NH3}. При этом раствор стал сине-сиреневым, а осадок растворился.

Аналогичные опыты были проведены с \ce{CuSO4} и \ce{CoCl3}. В случае меди осадок был синий (а после добавления \ce{NH3} раствор стал сине-фиолетовым), в случае кобальта --- сперва лазурным, после циановым (после добавления \ce{NH3} раствор сначала не поменял цвет, а затем посерел; осадок также растворился).

\begin{align}
	\ce{Ni(NO3)2 + 2NaOH &-> Ni(OH)2 + 2NaNO3} %\\
	%\ce{CuSO4 + 2NaOH &-> Cu(OH)2 + Na2SO4} \\
	%\ce{CoCl3 + 3NaOH &-> Co(OH)3 + 3 NaCl} 
\end{align}

\begin{align}
	\ce{Ni(OH)2 + 6 NH3 &-> [Ni(NH3)6](OH)2} %\\
	%\ce{Cu(OH)2 + 2 NH3 &-> [Cu(NH3)2](OH)2} \\
	%\ce{Co(OH)3 + 6 NH3 & -> [Co(NH3)6](OH)3}
\end{align}

\begin{equation}
	[Ni(NH3)6](OH)2 -> [Ni(NH3)3]^2+ + 2OH-
\end{equation}

Комплексное основание сильнее гидроксида, т.к. сила определяется радиусом катиона (у комплексного основания он очевидно больше).

\subsection{Опыт 3: Образование комплексных соединений в реакциях обмена}

\subsubsection{Реактивы и оборудование}

\begin{itemize}
	\item Растворы: \ce{CuSO4}, \ce{K4[Fe(CN)6]}
	
	\item Пробирки
\end{itemize}

\subsubsection{Порядок выполнения опыта}

В пробирке были смешаны несколько капель 0,1M раствора \ce{CuSO4} и \ce{K4[Fe(CN)6]}. В результате выпал хлопьевидный осадок коричневого цвета.

\begin{align*}
	\ce{2CuSO4 + K4[Fe(CN)6] &-> 2K2SO4 + Cu2[Fe(CN)6]}\\
	\ce{Cu^2+ + [Fe(CN)6]^4- &-> Cu2[Fe(Cn)6]}
\end{align*}


\subsection{Опыт 4: Получение двойного комплексного соединения}


\subsubsection{Реактивы и оборудование}

\begin{itemize}
	\item Растворы: \ce{Ni(NO3)2}, \ce{K4[Fe(CN)6]}, \ce{NH3}
\end{itemize}

\subsubsection{Порядок выполнения опыта}

В пробирку были внесены 3-4 капели \ce{K4[Fe(CN)6]}. К ним были добавлены 5-6 капель раствора \ce{Ni(NO3)2}, в результате чего выпал бело-лаймовый осадок. После этого в пробирку был добавлен раствор \ce{NH3}. Спустя некоторое время в пробирке появился лилово-сиреневый осадок.

\begin{align}
	\ce{K4Fe(CN)6 + 2Ni(NO3)2 &-> Ni2[Fe(CN)6] + 4KNO3} \\
	\ce{Ni2(Fe(CN)6) + 12NH3 &-> [Ni(NH3)6]2[Fe(CN)6]}
\end{align}

\subsection{Опыт 5: Окислительно-восстановительные реакции с участием комплексного иона}

\subsubsection{Реактивы и оборудование}

\begin{itemize}
	\item Растворы: \ce{KMnO4}, \ce{K4[Fe(CN)6]}, \ce{HCl} (1M)
	
	\item Пробирки
\end{itemize}

\subsubsection{Порядок выполнения опыта}

В пробирку были внесены 4-5 капли раствора \ce{KMnO4}. Раствор был подкислен несколькими каплями \ce{HCl} (1M). После этого в раствор по каплям был влит раствор \ce{K4[Fe(CN)6]}, в результате чего раствор покраснел.

\begin{equation}
	\ce{5K4[Fe(CN)6] + 8HCl + KMnO4 -> 5K3[Fe(CN)6] + MnCl2 + 6KCl + 4H2O}
\end{equation}

\begin{align*}
	\ce{Fe^{2+} - e &-> Fe^{+3}} \quad &\text{--- восстановитель}\\
	\ce{Mn^{+7} + 5 e &-> Mn^{+2} \quad &\text{--- окислитель}}
\end{align*}

\subsection{Опыт 6: Исследование устойчивости комплексных ионов}

\subsubsection{Реактивы и оборудование}

\begin{itemize}
	\item Растворы: \ce{AgNO3}, \ce{NaCl}, \ce{NH3}, \ce{KI}
	\item \ce{Zn} (Гранулы)
	
	\item Пробирки
\end{itemize}

\subsubsection{Порядок выполнения опыта}

Сперва необходимо было получить раствор \ce{ [Ag(NH3)2]Cl}. Для этого в пробирке сперва были смешаны 1 мл раствора \ce{NaCl} и 1-2 капли раствора \ce{AgNO3}, при этом в пробирке выпал беловатый осадок. После этого осадок был растворен раствором \ce{NH3}, благодаря чему и был получен \ce{[Ag(NH3)2]Cl}. Полученный раствор был разделен надвое.

Опишем строение \ce{[Ag(NH3)2]Cl}. \ce{Ag} --- комплексообразователь, \ce{NH3} --- лиганд (их два, они лиганды), \ce{Cl} --- ион на внешней сфере.

В первую часть раствора был добавлен раствор \ce{KI}. При этом выпал светло-светло-желто-лимонный осадок.

Во вторую часть раствора была внесена гранула цинка. В ходе реакции она почернела.

\begin{align}
	\ce{NaCl + AgNO3 &-> NaNO3 + AgCl}\\
	\ce{AgCl + 2NH3 &-> [Ag(NH3)2]Cl}\\
	\ce{[Ag(NH3)2]Cl + KI &-> AgI + KCl + 2NH3} \\
	\ce{2[Ag(NH3)2]Cl + Zn &-> Ag + ZnCl2 + 4NH3} % есть сомнения в этой реакции
\end{align}

% Надо добить этих несчастных

ПР$_{\ce{AgCl}} = [\ce{Ag^+}]\cdot[\ce{Cl^-}] \approx 1.77 \cdot 10^{-10}$

ПР$_{\ce{AgI}} = [\ce{Ag^+}]\cdot[\ce{I^-}] \approx 8.3 \cdot 10^{-17}$

\end{document}