%
%Не забыть:
%--------------------------------------
%Вставить колонтитулы, поменять название на титульнике



%--------------------------------------

\documentclass[a4paper, 12pt]{article} 

%--------------------------------------
%Russian-specific packages
%--------------------------------------
%\usepackage[warn]{mathtext}
\usepackage[T2A]{fontenc}
\usepackage[utf8]{inputenc}
\usepackage[english,russian]{babel}
\usepackage[intlimits]{amsmath}
\usepackage{esint}
%--------------------------------------
%Hyphenation rules
%--------------------------------------
\usepackage{hyphenat}
\hyphenation{ма-те-ма-ти-ка вос-ста-нав-ли-вать}
%--------------------------------------
%Packages
%--------------------------------------
\usepackage{amsmath}
\usepackage{amssymb}
\usepackage{amsfonts}
\usepackage{amsthm}
\usepackage{latexsym}
\usepackage{mathtools}
\usepackage{etoolbox}%Булевые операторы
\usepackage{extsizes}%Выставление произвольного шрифта в \documentclass
\usepackage{geometry}%Разметка листа
\usepackage{indentfirst}
\usepackage{wrapfig}%Создание обтекаемых текстом объектов
\usepackage{fancyhdr}%Создание колонтитулов
\usepackage{setspace}%Настройка интерлиньяжа
\usepackage{lastpage}%Вывод номера последней страницы в документе, \lastpage
\usepackage{soul}%Изменение параметров начертания
\usepackage{hyperref}%Две строчки с настройкой гиперссылок внутри получаеммого
\usepackage[usenames,dvipsnames,svgnames,table,rgb]{xcolor}% pdf-документа
\usepackage{multicol}%Позволяет писать текст в несколько колонок
\usepackage{cite}%Работа с библиографией
\usepackage{subfigure}% Человеческая вставка нескольких картинок
\usepackage{tikz}%Рисование рисунков
\usepackage{float}% Возможность ставить H в положениях картинки
% Для картинок Моти
\usepackage{misccorr}
\usepackage{lscape}
\usepackage{cmap}

% Для Х И М И И

\usepackage{mhchem}



\usepackage{graphicx,xcolor}
\graphicspath{{Pictures/}}
\DeclareGraphicsExtensions{.pdf,.png,.jpg}

%----------------------------------------
%Список окружений
%----------------------------------------
\newenvironment {theor}[2]
{\smallskip \par \textbf{#1.} \textit{#2}  \par $\blacktriangleleft$}
{\flushright{$\blacktriangleright$} \medskip \par} %лемма/теорема с доказательством
\newenvironment {proofn}
{\par $\blacktriangleleft$}
{$\blacktriangleright$ \par} %доказательство
%----------------------------------------
%Список команд
%----------------------------------------
\newcommand{\grad}
{\mathop{\mathrm{grad}}\nolimits\,} %градиент

\newcommand{\diver}
{\mathop{\mathrm{div}}\nolimits\,} %дивергенция

\newcommand{\rot}
{\ensuremath{\mathrm{rot}}\,}

\newcommand{\Def}[1]
{\underline{\textbf{#1}}} %определение

\newcommand{\RN}[1]
{\MakeUppercase{\romannumeral #1}} %римские цифры

\newcommand {\theornp}[2]
{\textbf{#1.} \textit{ #2} \par} %Написание леммы/теоремы без доказательства

\newcommand{\qrq}
{\ensuremath{\quad \Rightarrow \quad}} %Человеческий знак следствия

\newcommand{\qlrq}
{\ensuremath{\quad \Leftrightarrow \quad}} %Человеческий знак равносильности

\renewcommand{\phi}{\varphi} %Нормальный знак фи

\newcommand{\me}
{\ensuremath{\mathbb{E}}}

\newcommand{\md}
{\ensuremath{\mathbb{D}}}



%\renewcommand{\vec}{\overline}




%----------------------------------------
%Разметка листа
%----------------------------------------
\geometry{top = 3cm}
\geometry{bottom = 2cm}
\geometry{left = 1.5cm}
\geometry{right = 1.5cm}
%----------------------------------------
%Колонтитулы
%----------------------------------------
\pagestyle{fancy}%Создание колонтитулов
\fancyhead{}
%\fancyfoot{}
\fancyhead[R]{\textsc{Кислотно-основное титрование}}%Вставить колонтитул сюда
%----------------------------------------
%Интерлиньяж (расстояния между строчками)
%----------------------------------------
%\onehalfspacing -- интерлиньяж 1.5
%\doublespacing -- интерлиньяж 2
%----------------------------------------
%Настройка гиперссылок
%----------------------------------------
\hypersetup{				% Гиперссылки
	unicode=true,           % русские буквы в раздела PDF
	pdftitle={Заголовок},   % Заголовок
	pdfauthor={Автор},      % Автор
	pdfsubject={Тема},      % Тема
	pdfcreator={Создатель}, % Создатель
	pdfproducer={Производитель}, % Производитель
	pdfkeywords={keyword1} {key2} {key3}, % Ключевые слова
	colorlinks=true,       	% false: ссылки в рамках; true: цветные ссылки
	linkcolor=blue,          % внутренние ссылки
	citecolor=blue,        % на библиографию
	filecolor=magenta,      % на файлы
	urlcolor=cyan           % на URL
}
%----------------------------------------
%Работа с библиографией (как бич)
%----------------------------------------
\renewcommand{\refname}{Список литературы}%Изменение названия списка литературы для article
%\renewcommand{\bibname}{Список литературы}%Изменение названия списка литературы для book и report
%----------------------------------------
\begin{document}
	\begin{titlepage}
		\begin{center}
			$$$$
			$$$$
			$$$$
			$$$$
			{\Large{НАЦИОНАЛЬНЫЙ ИССЛЕДОВАТЕЛЬСКИЙ УНИВЕРСИТЕТ}}\\
			\vspace{0.1cm}
			{\Large{ВЫСШАЯ ШКОЛА ЭКОНОМИКИ}}\\
			\vspace{0.25cm}
			{\large{Факультет физики}}\\
			\vspace{5.5cm}
			{\Huge\textbf{{Лабораторная работа}}}\\%Общее название
			\vspace{1cm}
			{\LARGE{<<Кислотно-основное титрование>>}}\\%Точное название
			\vspace{2cm}
			{Работу выполнил студент 3 курса}\\
			{Захаров Сергей Дмитриевич}
			\vfill
			\includegraphics[width = 0.2\textwidth]{HSElogo}\\
			\vfill
			Москва\\
			5 сентября 2020
		\end{center}
	\end{titlepage}

\tableofcontents

\newpage

\section{Метод выполнения работы}

В ходе выполнение работы предполагалось провести титрование раствора соляной кислоты HCl раствором гидроксида натрия NaOH с целью определения реальной концентрации раствора NaOH.

Для этого сперва была приготовлена навеска гидроксида натрия для получение 100 мл раствора предполагаемой концентрацией 0.1M. Для расчета массы воспользуемся определением молярной концентрации:

\begin{equation}
		C_m = \frac{m}{M V} \qrq m = C_m M V
		\label{eq:NaOH_mass}
\end{equation}

Здесь $m$ --- масса растворенного вещества, $M$ --- его молярная масса, $V$ --- объем. С учетом того, что 0.1M означает 0.1 моль/л, примем следующие значения:

\begin{itemize}
	\item $M = 40$ г/моль 
	
	\item $V = 0.1$ л (необходимый объем раствора)
	
	\item $C_m = 0.1 \text{M}$
\end{itemize}

Таким образом, по формуле \ref{eq:NaOH_mass} получаем, что необходимая масса NaOH составляет $m_t = 0.4$~г. Данная масса была отмерена с помощью аналитических весов.

Раствор NaOH был получен следующим образом: сперва в мерную колбу на 100 мл была помещена навеска NaOH, после чего к ней была добавлен небольшой объем воды (порядка 40 мл). Затем, после тщательного перемешивания, была долита оставшаяся вода до отметки 100 мл.

Небольшим объемом получившегося раствора щелочи была промыта бюретка, после чего она была заполнена до нулевой отметки этим же раствором.

После этого в колбу с широким горлом было отмерено 10 мл 0.1 M раствора соляной кислоты, а также небольшой (2-3 капли) объем фенолфталеина. Затем, постоянно перемешивая раствор с помощью магнитной мешалки, в него постепенно, по капле, вносился раствор щелочи до получение устойчивой светло-розовой окраски. Необходимый для этого объем раствора щелочи был записан для последующей обработки. Кроме того, была также измерена реальная pH полученного раствора.

Последний этап был повторен 3 раза для получение более надежного результата.

\section{Полученные данные}

В результате серии экспериментов был получен следующий набор данных:

\begin{center}
	
	\begin{tabular}{|c|c|c|}
		\hline
		№ & Объем раствора NaOH, мл & pH \\
		\hline
		1 & 10.3 & 9.66  \\
		\hline
		2 & 10 & 9.23 \\
		\hline
		3 & 10.2 & 9.6 \\
		\hline
	\end{tabular}

\end{center}

Таким образом, принимаем за необходимый объем среднее значение объема раствора и получаем $V_{nec} \approx 10.2$ мл.

\section{Обработка полученных результатов}

\subsection{Определение концентрации раствора щелочи}

При добавлении к соляной кислоте в колбе щелочи происходит реакция нейтрализации, которая описывается следующей формулой:

\begin{equation}
	\ce{HCl + NaOH -> NaCl + H2O}
	\label{eq:react}
\end{equation}

Рассчитаем количество вещества \ce{HCl} в исходном растворе соляной кислоты:

\begin{equation}
		\nu_{HCl} = V C_{m HCl} = 0.01 \cdot 0.1 = 0.001 \text{ моль}
\end{equation}

Согласно уравнению реакции \ref{eq:react}, количество вещества \ce{NaOH} должно быть таким же, т.е.:

\begin{equation}
	\nu_{NaOH} = \nu_{HCl} = 0.001 \text{ моль}
\end{equation}

Таким образом, реальная концентрация раствора, которую мы получаем с помощью метода титрования, оказывается равной:

\begin{equation}
		\boxed{
		C_{mr} = \frac{\nu_{NaOH}}{V_{nec}} = \frac{0.001}{0.0102} \approx 0.98 \text{ М}
	}
\end{equation}

В целом, полученная концентрация не сильно отличается от той, которой мы пытались добиться.

\bigskip

\subsection{Определение pH раствора}

Мы смешивали сильную кислоту с сильным основанием. В ходе этой реакции мы получаем соль, которая не подвержена гидролизу. Следовательно, было бы логично ожидать pH в районе 7, что и является индикацией нейтральной среды.

\section{Выводы}

\begin{enumerate}
	\item Полученная в ходе эксперимента концентрация приготовленного нами раствора щелочи составляет $\boxed{C_{mr} \approx 0.098 \text{M}}$. Она несильно отличается от той, который мы пытались добиться (0.1 М). Отличие скорее всего вызвано несовершенством техники измерения объема израсходованной щелочи: при измерениях всегда есть погрешность, на которую в нашем случае накладывается еще и погрешность из-за наличия у раствора мениска.
	
	\item pH, полученная в ходе эксперимента (в среднем она составляет 9.5), отличается от ожидаемой (pH нейтральной среды 7). Скорее всего, это объясняется методикой выполнения эксперимента: при использовании фенолфталеина мы переставали добавлять щелочь при получении светло-розового окраса, что, строго говоря, является индикатором щелочной среды и свидетельствует о том, что реакция нейтрализации соляной кислоты завершилась, и в оставшемся растворе есть избыток щелочи. В таком случае, pH несколько больше нейтральной ожидаема (она и свидетельствует о щелочной среде).
\end{enumerate}


	
\end{document}