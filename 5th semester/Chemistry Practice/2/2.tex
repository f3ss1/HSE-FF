%
%Не забыть:
%--------------------------------------
%Вставить колонтитулы, поменять название на титульнике



%--------------------------------------

\documentclass[a4paper, 12pt]{article} 

%--------------------------------------
%Russian-specific packages
%--------------------------------------
%\usepackage[warn]{mathtext}
\usepackage[T2A]{fontenc}
\usepackage[utf8]{inputenc}
\usepackage[english,russian]{babel}
\usepackage[intlimits]{amsmath}
\usepackage{esint}
%--------------------------------------
%Hyphenation rules
%--------------------------------------
\usepackage{hyphenat}
\hyphenation{ма-те-ма-ти-ка вос-ста-нав-ли-вать}
%--------------------------------------
%Packages
%--------------------------------------
\usepackage{amsmath}
\usepackage{amssymb}
\usepackage{amsfonts}
\usepackage{amsthm}
\usepackage{latexsym}
\usepackage{mathtools}
\usepackage{etoolbox}%Булевые операторы
\usepackage{extsizes}%Выставление произвольного шрифта в \documentclass
\usepackage{geometry}%Разметка листа
\usepackage{indentfirst}
\usepackage{wrapfig}%Создание обтекаемых текстом объектов
\usepackage{fancyhdr}%Создание колонтитулов
\usepackage{setspace}%Настройка интерлиньяжа
\usepackage{lastpage}%Вывод номера последней страницы в документе, \lastpage
\usepackage{soul}%Изменение параметров начертания
\usepackage{hyperref}%Две строчки с настройкой гиперссылок внутри получаеммого
\usepackage[usenames,dvipsnames,svgnames,table,rgb]{xcolor}% pdf-документа
\usepackage{multicol}%Позволяет писать текст в несколько колонок
\usepackage{cite}%Работа с библиографией
\usepackage{subfigure}% Человеческая вставка нескольких картинок
\usepackage{tikz}%Рисование рисунков
\usepackage{float}% Возможность ставить H в положениях картинки
% Для картинок Моти
\usepackage{misccorr}
\usepackage{lscape}
\usepackage{cmap}

% Для Х И М И И

\usepackage{mhchem}



\usepackage{graphicx,xcolor}
\graphicspath{{Pictures/}}
\DeclareGraphicsExtensions{.pdf,.png,.jpg}

%----------------------------------------
%Список окружений
%----------------------------------------
\newenvironment {theor}[2]
{\smallskip \par \textbf{#1.} \textit{#2}  \par $\blacktriangleleft$}
{\flushright{$\blacktriangleright$} \medskip \par} %лемма/теорема с доказательством
\newenvironment {proofn}
{\par $\blacktriangleleft$}
{$\blacktriangleright$ \par} %доказательство
%----------------------------------------
%Список команд
%----------------------------------------
\newcommand{\grad}
{\mathop{\mathrm{grad}}\nolimits\,} %градиент

\newcommand{\diver}
{\mathop{\mathrm{div}}\nolimits\,} %дивергенция

\newcommand{\rot}
{\ensuremath{\mathrm{rot}}\,}

\newcommand{\Def}[1]
{\underline{\textbf{#1}}} %определение

\newcommand{\RN}[1]
{\MakeUppercase{\romannumeral #1}} %римские цифры

\newcommand {\theornp}[2]
{\textbf{#1.} \textit{ #2} \par} %Написание леммы/теоремы без доказательства

\newcommand{\qrq}
{\ensuremath{\quad \Rightarrow \quad}} %Человеческий знак следствия

\newcommand{\qlrq}
{\ensuremath{\quad \Leftrightarrow \quad}} %Человеческий знак равносильности

\renewcommand{\phi}{\varphi} %Нормальный знак фи

\newcommand{\me}
{\ensuremath{\mathbb{E}}}

\newcommand{\md}
{\ensuremath{\mathbb{D}}}



%\renewcommand{\vec}{\overline}




%----------------------------------------
%Разметка листа
%----------------------------------------
\geometry{top = 3cm}
\geometry{bottom = 2cm}
\geometry{left = 1.5cm}
\geometry{right = 1.5cm}
%----------------------------------------
%Колонтитулы
%----------------------------------------
\pagestyle{fancy}%Создание колонтитулов
\fancyhead{}
%\fancyfoot{}
\fancyhead[R]{\textsc{Кислотно-основное равновесие в растворах. Гидролиз. Буферные растворы}}%Вставить колонтитул сюда
%----------------------------------------
%Интерлиньяж (расстояния между строчками)
%----------------------------------------
%\onehalfspacing -- интерлиньяж 1.5
%\doublespacing -- интерлиньяж 2
%----------------------------------------
%Настройка гиперссылок
%----------------------------------------
\hypersetup{				% Гиперссылки
	unicode=true,           % русские буквы в раздела PDF
	pdftitle={Заголовок},   % Заголовок
	pdfauthor={Автор},      % Автор
	pdfsubject={Тема},      % Тема
	pdfcreator={Создатель}, % Создатель
	pdfproducer={Производитель}, % Производитель
	pdfkeywords={keyword1} {key2} {key3}, % Ключевые слова
	colorlinks=true,       	% false: ссылки в рамках; true: цветные ссылки
	linkcolor=blue,          % внутренние ссылки
	citecolor=blue,        % на библиографию
	filecolor=magenta,      % на файлы
	urlcolor=cyan           % на URL
}
%----------------------------------------
%Работа с библиографией (как бич)
%----------------------------------------
\renewcommand{\refname}{Список литературы}%Изменение названия списка литературы для article
%\renewcommand{\bibname}{Список литературы}%Изменение названия списка литературы для book и report
%----------------------------------------
\begin{document}
	\begin{titlepage}
		\begin{center}
			$$$$
			$$$$
			$$$$
			$$$$
			{\Large{НАЦИОНАЛЬНЫЙ ИССЛЕДОВАТЕЛЬСКИЙ УНИВЕРСИТЕТ}}\\
			\vspace{0.1cm}
			{\Large{ВЫСШАЯ ШКОЛА ЭКОНОМИКИ}}\\
			\vspace{0.25cm}
			{\large{Факультет физики}}\\
			\vspace{5.5cm}
			{\Huge\textbf{{Лабораторная работа}}}\\%Общее название
			\vspace{1cm}
			{\LARGE{<<Кислотно-основное титрование>>}}\\%Точное название
			\vspace{2cm}
			{Работу выполнил студент 3 курса}\\
			{Захаров Сергей Дмитриевич}
			\vfill
			\includegraphics[width = 0.2\textwidth]{HSElogo}\\
			\vfill
			Москва\\
			12 сентября 2020
		\end{center}
	\end{titlepage}

\tableofcontents

\newpage

\section{Опыт 1: Гидролиз солей}

\subsection{Реактивы и оборудование:}

\begin{itemize}
	\item Сухие соли: \ce{CH3COONa}, \ce{MgCl2}, \ce{Na2CO3}, \ce{(NH4)2CO3}, \ce{NaCl}, \ce{CH3COONH4}, \ce{Na2SO3}, \ce{ZnCl2}
	
	\item Раствор универсального индикатора
	
	\item Пробирки
	
	\item Шпатель для реактивов
	
	\item Стеклянная палочка
\end{itemize}

\subsection{Порядок выполнения опыта}

В 8 пробирок были добавлены по одному микрошпателю указанных сухих солей, после чего они были разбавлены одинаковым небольшим количеством дистиллированной воды. К полученным растворам был также добавлен в небольшом объеме (2-3 капли). Все растворы были тщательно перемешаны стеклянной палочкой.

В результате были получены следующие значения для pH:


\begin{center}
\begin{tabular}{|c|c|c|c|c|c|c|c|c|}
	\hline
	В-во & \ce{CH3COONa} & \ce{MgCl2} & \ce{Na2CO3} & \ce{(NH4)2CO3} & \ce{NaCl} & \ce{CH3COONH4} & \ce{Na2SO3} & \ce{ZnCl2} \\
	\hline
	pH & 8 & 7 & 10 & 9.5 & 7 & 8 & 10 & 4.5 \\
	\hline
\end{tabular}
\end{center}

\subsection{Дополнительное задание}

\begin{itemize}
	\item \ce{CH3COONa} --- сильное основание и слабая кислота, гидролиз по аниону:
	
	\ce{CH3COO- + HOH <=> CH3COOH + OH-} 
	
	\ce{CH3COONa + HOH <=> CH3COOH + NaOH}
	
	\item \ce{MgCl2} --- среднее основание и сильная кислота, гидролиз в целом не идет (но если бы шел, то был бы по катиону)
	
	%\textbf{1 ступень}
	%
	%\ce{Mg2+ + HOH <=> MgOH+ + H+}
	%
	%\ce{MgCl2 + HOH <=> MgOHCl + HCl}
	%
	%\textbf{2 ступень}
	%
	%\ce{MgOH+ + HOH <=> Mg(OH)2 + H+}
	%
	%\ce{MgOHCl + HOH <=> Mg(OH)2 + HCl}
	
	\item \ce{Na2CO3} --- сильное основание и слабая кислота, гидролиз по аниону
	
	\textbf{1 ступень}
	
	\ce{CO3^2- + HOH <=> HCO3- + OH-}
	
	\ce{Na2CO3 + HOH <=> NaHCO3 + NaOH}
	
	\textbf{2 ступень}
	
	\ce{HCO3- + HOH <=> H2CO3 + OH-}
	
	\ce{NaHCO3 + HOH <=> H2CO3 + NaOH}
	
	\item \ce{(NH4)2CO3} --- слабое основание и слабая кислота, гидролиз и по аниону, и по катиону:
	
	\ce{NH4+ + CO3^2- + HOH <=> HCO3- + NH4OH}
	
	\ce{(NH4)2CO3 + HOH <=> NH4CO3 + NH4OH}
	
	\item \ce{NaCl} --- сильное основание и сильная кислота, гидролиз не идет.
	
	\item \ce{CH3COONH4} --- слабое основание и слабая кислота, гидролиз и по аниону, и по катиону:
	
	\ce{NH4+ + COOCH3- + HOH <=> NH4OH + CH3COOH}
	
	\ce{CH3COONH4 + HOH <=> NH4OH + CH3COOH}
	
	\item \ce{Na2SO3} --- сильное основание и слабая кислота, гидролиз по аниону
	
	\textbf{1 ступень}
	
	\ce{SO3^2- + HOH <=> HSO3- + OH-}
	
	\ce{Na2SO3 + HOH <=> NaHSO3 + NaOH}
	
	\textbf{2 ступень}
	
	\ce{HSO3- + HOH <=> H2SO3 + OH-}
	
	\ce{NaHSO3 + HOH <=> H2SO3 + NaOH}
	
	\item \ce{ZnCl2} --- слабое основание и сильная кислота, гидролиз по катиону
	
	\textbf{1 ступень}
	
	\ce{Zn^2+ + HOH <=> ZnOH+ + H+}
	
	\ce{ZnCl2 + HOH <=> ZnOHCl + HCl}
	
	\textbf{2 ступень}
	
	\ce{ZnOH+ + HOH <=> Zn(OH)2 + H+}
	
	\ce{ZnOHCl + HOH <=> Zn(OH)2 + HCl}
	
\end{itemize}




\section{Опыт 2: Факторы, влияющие на степень гидролиза}

\subsection{Реактивы и оборудование}

\begin{itemize}
	\item Сухие соли: \ce{CH3COONa}, \ce{MgCl2}, \ce{Na2CO3}, \ce{NaHCO3}, \ce{Na2SO3}, \ce{ZnCl2}
	
	\item Раствор универсального индикатора
	
	\item Индикаторная бумага
	
	\item Пробирки
	
	\item Шпатель для реактивов
	
	\item Стеклянная палочка
	
	\item Спиртовка
\end{itemize}

\subsection{Порядок выполнения}

\subsubsection*{Влияние силы кислоты и основания, образующих соль, на степень ее гидролиза}

В одну пробирку был внесен \ce{Na2SO3}, во вторую --- \ce{Na2CO3}. К обеим солям было прилито одно и то же небольшое количество воды и несколько капель универсального индикатора, после чего они были размешаны с помощью стеклянной палочки.

Полученные результаты приведены в таблице ниже:

\begin{center}
	\begin{tabular}{|c|c|c|}
		\hline
		В-во & \ce{Na2SO3} & \ce{Na2CO3} \\
		\hline
		pH & 10 & 11 \\
		\hline
	\end{tabular}
\end{center} 

\ce{H2CO3} более сильная, чем \ce{H2SO3}, поэтому степень гидролиза будет выше у \ce{Na2CO3}.

\begin{itemize}
	\item \ce{Na2SO3} --- сильное основание и слабая кислота, гидролиз по аниону
	
	\textbf{1 ступень}
	
	\ce{SO3^2- + HOH <=> HSO3- + OH-}
	
	\ce{Na2SO3 + HOH <=> NaHSO3 + NaOH}
	
	\textbf{2 ступень}
	
	\ce{HSO3- + HOH <=> H2SO3 + OH-}
	
	\ce{NaHSO3 + HOH <=> H2SO3 + NaOH}
	
	\item \ce{Na2CO3} --- сильное основание и слабая кислота, гидролиз по аниону
	
	\textbf{1 ступень}
	
	\ce{CO3^2- + HOH <=> HCO3- + OH-}
	
	\ce{Na2CO3 + HOH <=> NaHCO3 + NaOH}
	
	\textbf{2 ступень}
	
	\ce{HCO3- + HOH <=> H2CO3 + OH-}
	
	\ce{NaHCO3 + HOH <=> H2CO3 + NaOH}
	
\end{itemize}

То же самое было проделано для \ce{ZnCl2} и \ce{MgCl2}. Результаты:

% Здесь все сломалось

\begin{center}
	\begin{center}
		\begin{tabular}{|c|c|c|}
			\hline
			В-во & \ce{ZnCl2} & \ce{MgCl2} \\
			\hline
			pH & 5 & 8 \\
			\hline
		\end{tabular}
	\end{center} 
\end{center}

\begin{itemize}
	\item \ce{ZnCl2} --- слабое основание и сильная кислота, гидролиз по катиону
	
	\textbf{1 ступень}
	
	\ce{Zn^2+ + HOH <=> ZnOH+ + H+}
	
	\ce{ZnCl2 + HOH <=> ZnOHCl + HCl}
	
	\textbf{2 ступень}
	
	\ce{ZnOH+ + HOH <=> Zn(OH)2 + H+}
	
	\ce{ZnOHCl + HOH <=> Zn(OH)2 + HCl}
	
	\item \ce{MgCl2} --- слабое основание и сильная кислота, гидролиз по катиону:
	
	\textbf{1 ступень}
	
	\ce{Mg2+ + HOH <=> MgOH+ + H+}
	
	\ce{MgCl2 + HOH <=> MgOHCl + HCl}
	
	\textbf{2 ступень}
	
	\ce{MgOH+ + HOH <=> Mg(OH)2 + H+}
	
	\ce{MgOHCl + HOH <=> Mg(OH)2 + HCl}
\end{itemize}



\subsubsection*{Влияние температуры на степень гидролиза}

В пробирку был внесен \ce{CH3COONa}, к которому был прилит небольшой объем воды и несколько капель фенолфталеина, после чего раствор был перемешан. Раствор при этом оставался прозрачным. После этого пробирка с раствором была постепенно нагрета на спиртовой горелке, в ходе чего было отмечено изменение оттенка раствора с бесцветного на нежно-розовый, что свидетельствует о появлении в пробирке щелочной среды. Это неудивительно: гидролиз -- эндотермическая среда, поэтому повышение температуры смещает равновесие в сторону продуктов.

\subsubsection*{Гидролиз средних и кислых солей}

В одну пробирку был внесен \ce{Na2CO3}, во вторую --- \ce{NaHCO3}. К обеим солям было прилито одно и то же небольшое количество воды и несколько капель универсального индикатора, после чего они были размешаны с помощью стеклянной палочки.

Результаты pH полученных растворов приведены ниже:

\begin{center}
	\begin{tabular}{|c|c|c|}
		\hline
		В-во & \ce{Na2CO3} & \ce{NaHCO3} \\
		\hline
		pH & 9.5 & 7.5 \\
		\hline
	\end{tabular}
\end{center}

\begin{itemize}
	\item \ce{Na2CO3} --- сильное основание и слабая кислота, гидролиз по аниону
	
	\textbf{1 ступень}
	
	\ce{CO3^2- + HOH <=> HCO3- + OH-}
	
	\ce{Na2CO3 + HOH <=> NaHCO3 + NaOH}
	
	\textbf{2 ступень}
	
	\ce{HCO3- + HOH <=> H2CO3 (H2O + CO2) + OH-}
	
	\ce{NaHCO3 + HOH <=> H2CO3 (H2O + CO2) + NaOH}
	
	%ниже хз
	
	\item \ce{NaHCO3} --- сильное основание и слабая кислота, гидролиз по аниону
	
	\ce{HCO3- + H2O <=> H2CO3 (H2O + CO2) + OH-}
	
	\ce{NaHCO3 + H2O <=> H2CO3 (H2O + CO2) + NaOH}
\end{itemize}



\section{Буферные растворы}

\subsection{Реактивы и оборудование}

\begin{itemize}
	\item Сухие соли: \ce{NaH2PO4*2H2O}, \ce{NaOH}
	
	\item Растворы: \ce{HCl} 0.1M, \ce{NaOH} 0.1M
	
	\item Раствор универсального индикатора
	
	\item Индикаторная бумага
	
	\item Мерная колба на 100 мл
	
	\item Весы
	
	\item Шпатель для реактивов
	
	\item Стеклянная палочка
	
	\item Два стаканчика на 100 мл
\end{itemize}

\subsubsection*{Расчет навесок}

\ce{NaH2PO4 + NaOH -> Na2HPO4 + H2O}

По условию $\nu(\ce{NaH2PO4}) = 2\nu(\ce{Na2HPO4})$. Положим индекс 1 для \ce{NaH2PO4}, индекс 2 --- для \ce{Na2HPO4}. Тогда 

Подготовленные навески \ce{NaH2PO4*2H2O}, \ce{NaOH} были внесены в мерную колбу, после чего она была залита водой до отметки, а полученный раствор --- тщательно перемешан. pH полученного буферного раствора оказалась близка к нейтральной.

После этого полученный раствор был разделен поровну между двумя стаканчиками на 100 мл. В стаканчики было добавлено 2-3 капли универсального индикатора. Затем в один из стаканчиков по капле прибавлялся раствор \ce{HCl}, в другой --- \ce{NaOH}

Наблюдения следующие: до добавления определенного объема раствора кислоты или щелочи окраска раствора менялась слабо, после чего резко поменялась. Пороговые значения приведены ниже:

\begin{center}
	\begin{tabular}{|c|c|c|}
		\hline
		В-во & \ce{HCl} & \ce{NaOH} \\
		\hline
		Объем, мл & 2.5 & 2.4 \\
		\hline
	\end{tabular}
\end{center}

Запишем формулу диссоциации:

\ce{H2PO4- <=> H+ + HPO4^2-}

Посчитаем константу:

\begin{equation*}
	K_A = \frac{[\ce{HPO4^2-}][\ce{H+}]}{[\ce{HPO4^-}]} \approx 6.2\cdot10^{-8} \qrq \text{p}K_A = -\lg K_A = 7.2
\end{equation*}

Таким образом, pH изначального раствора оказывается равной:

\begin{equation*}
	\text{pH} = \text{p}K_A + \lg \frac{[\ce{Na2HPO4}]}{[\ce{NaH2PO4}]} = 7.2
\end{equation*}

Реакция с основанием:

\ce{NaH2PO4 + NaOH -> Na2HPO4 + H2O}

В таком случае pH:

\begin{equation*}
	\text{pH} = \text{p}K_A + \lg \frac{[\ce{Na2HPO4}] + \Delta }{[\ce{NaH2PO4}] - \Delta} = 7.7
\end{equation*}

Реакция с кислотой:

\ce{Na2HPO4 + HCl -> NaH2PO4 + NaCl}

В таком случае pH:

\begin{equation*}
	\text{pH} = \text{p}K_A + \lg \frac{[\ce{Na2HPO4}] - \Delta }{[\ce{NaH2PO4}] + \Delta} = 6.65
\end{equation*}

\subsubsection*{Вывод константы гидролиза}

Выведем на примере \ce{CH3COONa}:

\ce{CH3COO- + HOH <=> CH3COOH + OH-}

\ce{CH3COONa + HOH <=> CH3COOH + NaOH}

\begin{equation*}
	\frac{[\ce{OH-}][\ce{CH3COOH}]}{[\ce{CH3COO-}][\ce{H2O}]} = K \quad K_\Gamma = K[\ce{H2O}] 
\end{equation*}

Введем $K_{\ce{H2O}} = [\ce{H+}][\ce{OH-}]$,тогда:

\begin{equation*}
	\frac{K_{\ce{H2O}}[\ce{CH3COOH}]}{[\ce{H+}][\ce{CH3COO-}]} = \frac{K_{\ce{H2O}}}{K_{\ce{CH3COOH}}} = K_\Gamma = 5.9 \cdot 10^{-10}
\end{equation*}




\end{document}