%
%Не забыть:
%--------------------------------------
%Вставить колонтитулы, поменять название на титульнике



%--------------------------------------

\documentclass[a4paper, 12pt]{article} 

%--------------------------------------
%Russian-specific packages
%--------------------------------------
%\usepackage[warn]{mathtext}
\usepackage[T2A]{fontenc}
\usepackage[utf8]{inputenc}
\usepackage[english,russian]{babel}
\usepackage[intlimits]{amsmath}
\usepackage{esint}
%--------------------------------------
%Hyphenation rules
%--------------------------------------
\usepackage{hyphenat}
\hyphenation{ма-те-ма-ти-ка вос-ста-нав-ли-вать}
%--------------------------------------
%Packages
%--------------------------------------
\usepackage{amsmath}
\usepackage{amssymb}
\usepackage{amsfonts}
\usepackage{amsthm}
\usepackage{latexsym}
\usepackage{mathtools}
\usepackage{etoolbox}%Булевые операторы
\usepackage{extsizes}%Выставление произвольного шрифта в \documentclass
\usepackage{geometry}%Разметка листа
\usepackage{indentfirst}
\usepackage{wrapfig}%Создание обтекаемых текстом объектов
\usepackage{fancyhdr}%Создание колонтитулов
\usepackage{setspace}%Настройка интерлиньяжа
\usepackage{lastpage}%Вывод номера последней страницы в документе, \lastpage
\usepackage{soul}%Изменение параметров начертания
\usepackage{hyperref}%Две строчки с настройкой гиперссылок внутри получаеммого
\usepackage[usenames,dvipsnames,svgnames,table,rgb]{xcolor}% pdf-документа
\usepackage{multicol}%Позволяет писать текст в несколько колонок
\usepackage{cite}%Работа с библиографией
\usepackage{subfigure}% Человеческая вставка нескольких картинок
\usepackage{tikz}%Рисование рисунков
\usepackage{float}% Возможность ставить H в положениях картинки
\usepackage{wasysym} % Астрономические знаки
% Для картинок Моти
\usepackage{misccorr}
\usepackage{lscape}
\usepackage{cmap}



\usepackage{graphicx,xcolor}
\graphicspath{{Pictures/}}
\DeclareGraphicsExtensions{.pdf,.png,.jpg}

%----------------------------------------
%Список окружений
%----------------------------------------
\newenvironment {theor}[2]
{\smallskip \par \textbf{#1.} \textit{#2}  \par $\blacktriangleleft$}
{\flushright{$\blacktriangleright$} \medskip \par} %лемма/теорема с доказательством
\newenvironment {proofn}
{\par $\blacktriangleleft$}
{$\blacktriangleright$ \par} %доказательство
%----------------------------------------
%Список команд
%----------------------------------------
\newcommand{\grad}
{\mathop{\mathrm{grad}}\nolimits\,} %градиент

\newcommand{\diver}
{\mathop{\mathrm{div}}\nolimits\,} %дивергенция

\newcommand{\rot}
{\ensuremath{\mathrm{rot}}\,}

\newcommand{\Def}[1]
{\underline{\textbf{#1}}} %определение

\newcommand{\RN}[1]
{\MakeUppercase{\romannumeral #1}} %римские цифры

\newcommand {\theornp}[2]
{\textbf{#1.} \textit{ #2} \par} %Написание леммы/теоремы без доказательства

\newcommand{\qrq}
{\ensuremath{\quad \Rightarrow \quad}} %Человеческий знак следствия

\newcommand{\qlrq}
{\ensuremath{\quad \Leftrightarrow \quad}} %Человеческий знак равносильности

\renewcommand{\phi}{\varphi} %Нормальный знак фи

\newcommand{\me}
{\ensuremath{\mathbb{E}}}

\newcommand{\md}
{\ensuremath{\mathbb{D}}}

\newcommand{\bra}[1]
{\ensuremath{\left\langle#1\right|}}

\newcommand{\cat}[1]
{\ensuremath{\left|#1\right\rangle}}

\newcommand{\const}{\text{const.}}



%\renewcommand{\vec}{\overline}




%----------------------------------------
%Разметка листа
%----------------------------------------
\geometry{top = 3cm}
\geometry{bottom = 2cm}
\geometry{left = 1.5cm}
\geometry{right = 1.5cm}
%----------------------------------------
%Колонтитулы
%----------------------------------------
\pagestyle{fancy}%Создание колонтитулов
\fancyhead{}
%\fancyfoot{}
%\fancyhead[R]{\textsc{Уравнения Лондонов. Кинетическая индуктивность сверхпроводников.}}%Вставить колонтитул сюда
%----------------------------------------
%Интерлиньяж (расстояния между строчками)
%----------------------------------------
%\onehalfspacing -- интерлиньяж 1.5
%\doublespacing -- интерлиньяж 2
%----------------------------------------
%Настройка гиперссылок
%----------------------------------------
\hypersetup{				% Гиперссылки
	unicode=true,           % русские буквы в раздела PDF
	pdftitle={Заголовок},   % Заголовок
	pdfauthor={Автор},      % Автор
	pdfsubject={Тема},      % Тема
	pdfcreator={Создатель}, % Создатель
	pdfproducer={Производитель}, % Производитель
	pdfkeywords={keyword1} {key2} {key3}, % Ключевые слова
	colorlinks=true,       	% false: ссылки в рамках; true: цветные ссылки
	linkcolor=blue,          % внутренние ссылки
	citecolor=blue,        % на библиографию
	filecolor=magenta,      % на файлы
	urlcolor=red           % на URL
}
%----------------------------------------
%Работа с библиографией (как бич)
%----------------------------------------
\renewcommand{\refname}{Список литературы}%Изменение названия списка литературы для article
%\renewcommand{\bibname}{Список литературы}%Изменение названия списка литературы для book и report
%----------------------------------------
\begin{document}
	\begin{titlepage}
		\begin{center}
			$$$$
			$$$$
			$$$$
			$$$$
			{\Large{НАЦИОНАЛЬНЫЙ ИССЛЕДОВАТЕЛЬСКИЙ УНИВЕРСИТЕТ}}\\
			\vspace{0.1cm}
			{\Large{ВЫСШАЯ ШКОЛА ЭКОНОМИКИ}}\\
			\vspace{0.25cm}
			{\large{Факультет физики}}\\
			\vspace{5.5cm}
			{\Huge\textbf{{Домашнее задание}}}\\%Общее название
			\vspace{1cm}
			{\LARGE{Введение в астрофизику, неделя 5}}\\%Точное название
			\vspace{2cm}
			{Задание выполнил студент 3 курса}\\
			{Захаров Сергей Дмитриевич}
			\vfill
			\includegraphics[width = 0.2\textwidth]{HSElogo}\\
			\vfill
			Москва\\
			2020
		\end{center}
	\end{titlepage}

\section*{Задача 1}

Считая, что условием остановки магнитным полем потока вещества является равенство магнитного давления (на радиусе остановки) и давления, связанного с движением падающего вещества, определите, при каком темпе аккреции можно пренебречь влиянием магнитного поля, если на поверхности нейтронной звезды оно равно $10^8$~Гс. Какой будет при этом аккреционная светимость?

\subsubsection*{Решение}

А его тут нет.

\section*{Задача 2}

Идет аккреция на белый карлик. Темп аккреции $10^{20}$ грамм в секунду. Определите светимость источника и температуру, задав типичные параметры белого карлика. В каком диапазоне можно наблюдать излучение такого объекта?

\subsubsection*{Решение}

Скажем, что вещество ''падает'' с бесконечности, т.е. $E = 0$. В таком случае, при падении на белый карлик, оно будет передавать ему энергию:

\begin{equation}
	E = \frac{G M m}{R}
\end{equation}

Здесь $G$ --- гравитационная постоянная, $M$ --- масса белого карлика, $R$ --- радиус белого карлика, $m$ --- масса падающего вещества.

Согласно семинарским предпосылкам, скажем, что излучается примерно половина этой энергии, т.е. для мощности (а в нашем контексте для светимости) мы получим:

\begin{equation}
	L = \frac{1}{2} \frac{G M \dot{m}}{R}
\end{equation}

Здесь $\dot{m}$ --- темп аккреции.

В качестве параметров белого карлика возьмем параметры, например, Сириуса~B: $M \approx M_{\odot}$, $R \approx 0.0084 R_{\odot}$. Тогда получим:

\begin{equation}
	\boxed{L = \frac{1}{2} \frac{G M \dot{m}}{R} \approx 1.1 \cdot 10^{37} \text{ эрг/с}}
\end{equation}

Это светимость \textbf{чисто} от аккреции, без учета светимости самого карлика.

Дальше вообще непонятно что делать, если честно, поэтому мы предположим довольно тупые вещи в духе того что излучение от аккреции идет как будто бы от черного тела (потому что я искренне не знаю как иначе), и что площадь излучающей поверхности такая же, как у звезды (тоже на первый взгляд не самая тривиальная вещь).

В таком случае, по формуле Стефана-Больцмана, получим:

\begin{equation}
	L = 4\pi R^2 \sigma T^4 \qrq T = \sqrt[\uproot{2}4]{\frac{L}{4 \pi R^2 \sigma}} 
\end{equation}

Теперь, воспользовавшись законом смещения Вина:

\begin{equation}
	\lambda_{\text{max}} = \frac{b}{T} = \boxed{b \cdot \sqrt[\uproot{2}4]{\frac{4\pi R^2 \sigma}{L}} = \lambda_{\text{max}} \approx 6.3 \text{ pм}}
\end{equation}

Получаем, что максимум приходится на что-то пограничное между гамма-излучением и рентгеновским излучением.


\end{document}