%
%Не забыть:
%--------------------------------------
%Вставить колонтитулы, поменять название на титульнике



%--------------------------------------

\documentclass[a4paper, 12pt]{article} 

%--------------------------------------
%Russian-specific packages
%--------------------------------------
%\usepackage[warn]{mathtext}
\usepackage[T2A]{fontenc}
\usepackage[utf8]{inputenc}
\usepackage[english,russian]{babel}
\usepackage[intlimits]{amsmath}
\usepackage{esint}
%--------------------------------------
%Hyphenation rules
%--------------------------------------
\usepackage{hyphenat}
\hyphenation{ма-те-ма-ти-ка вос-ста-нав-ли-вать}
%--------------------------------------
%Packages
%--------------------------------------
\usepackage{amsmath}
\usepackage{amssymb}
\usepackage{amsfonts}
\usepackage{amsthm}
\usepackage{latexsym}
\usepackage{mathtools}
\usepackage{etoolbox}%Булевые операторы
\usepackage{extsizes}%Выставление произвольного шрифта в \documentclass
\usepackage{geometry}%Разметка листа
\usepackage{indentfirst}
\usepackage{wrapfig}%Создание обтекаемых текстом объектов
\usepackage{fancyhdr}%Создание колонтитулов
\usepackage{setspace}%Настройка интерлиньяжа
\usepackage{lastpage}%Вывод номера последней страницы в документе, \lastpage
\usepackage{soul}%Изменение параметров начертания
\usepackage{hyperref}%Две строчки с настройкой гиперссылок внутри получаеммого
\usepackage[usenames,dvipsnames,svgnames,table,rgb]{xcolor}% pdf-документа
\usepackage{multicol}%Позволяет писать текст в несколько колонок
\usepackage{cite}%Работа с библиографией
\usepackage{subfigure}% Человеческая вставка нескольких картинок
\usepackage{tikz}%Рисование рисунков
\usepackage{float}% Возможность ставить H в положениях картинки
\usepackage{wasysym} % Астрономические знаки
% Для картинок Моти
\usepackage{misccorr}
\usepackage{lscape}
\usepackage{cmap}



\usepackage{graphicx,xcolor}
\graphicspath{{Pictures/}}
\DeclareGraphicsExtensions{.pdf,.png,.jpg}

%----------------------------------------
%Список окружений
%----------------------------------------
\newenvironment {theor}[2]
{\smallskip \par \textbf{#1.} \textit{#2}  \par $\blacktriangleleft$}
{\flushright{$\blacktriangleright$} \medskip \par} %лемма/теорема с доказательством
\newenvironment {proofn}
{\par $\blacktriangleleft$}
{$\blacktriangleright$ \par} %доказательство
%----------------------------------------
%Список команд
%----------------------------------------
\newcommand{\grad}
{\mathop{\mathrm{grad}}\nolimits\,} %градиент

\newcommand{\diver}
{\mathop{\mathrm{div}}\nolimits\,} %дивергенция

\newcommand{\rot}
{\ensuremath{\mathrm{rot}}\,}

\newcommand{\Def}[1]
{\underline{\textbf{#1}}} %определение

\newcommand{\RN}[1]
{\MakeUppercase{\romannumeral #1}} %римские цифры

\newcommand {\theornp}[2]
{\textbf{#1.} \textit{ #2} \par} %Написание леммы/теоремы без доказательства

\newcommand{\qrq}
{\ensuremath{\quad \Rightarrow \quad}} %Человеческий знак следствия

\newcommand{\qlrq}
{\ensuremath{\quad \Leftrightarrow \quad}} %Человеческий знак равносильности

\renewcommand{\phi}{\varphi} %Нормальный знак фи

\newcommand{\me}
{\ensuremath{\mathbb{E}}}

\newcommand{\md}
{\ensuremath{\mathbb{D}}}

\newcommand{\bra}[1]
{\ensuremath{\left\langle#1\right|}}

\newcommand{\cat}[1]
{\ensuremath{\left|#1\right\rangle}}

\newcommand{\const}{\text{const.}}



%\renewcommand{\vec}{\overline}




%----------------------------------------
%Разметка листа
%----------------------------------------
\geometry{top = 3cm}
\geometry{bottom = 2cm}
\geometry{left = 1.5cm}
\geometry{right = 1.5cm}
%----------------------------------------
%Колонтитулы
%----------------------------------------
\pagestyle{fancy}%Создание колонтитулов
\fancyhead{}
%\fancyfoot{}
%\fancyhead[R]{\textsc{Уравнения Лондонов. Кинетическая индуктивность сверхпроводников.}}%Вставить колонтитул сюда
%----------------------------------------
%Интерлиньяж (расстояния между строчками)
%----------------------------------------
%\onehalfspacing -- интерлиньяж 1.5
%\doublespacing -- интерлиньяж 2
%----------------------------------------
%Настройка гиперссылок
%----------------------------------------
\hypersetup{				% Гиперссылки
	unicode=true,           % русские буквы в раздела PDF
	pdftitle={Заголовок},   % Заголовок
	pdfauthor={Автор},      % Автор
	pdfsubject={Тема},      % Тема
	pdfcreator={Создатель}, % Создатель
	pdfproducer={Производитель}, % Производитель
	pdfkeywords={keyword1} {key2} {key3}, % Ключевые слова
	colorlinks=true,       	% false: ссылки в рамках; true: цветные ссылки
	linkcolor=blue,          % внутренние ссылки
	citecolor=blue,        % на библиографию
	filecolor=magenta,      % на файлы
	urlcolor=red           % на URL
}
%----------------------------------------
%Работа с библиографией (как бич)
%----------------------------------------
\renewcommand{\refname}{Список литературы}%Изменение названия списка литературы для article
%\renewcommand{\bibname}{Список литературы}%Изменение названия списка литературы для book и report
%----------------------------------------
\begin{document}
	\begin{titlepage}
		\begin{center}
			$$$$
			$$$$
			$$$$
			$$$$
			{\Large{НАЦИОНАЛЬНЫЙ ИССЛЕДОВАТЕЛЬСКИЙ УНИВЕРСИТЕТ}}\\
			\vspace{0.1cm}
			{\Large{ВЫСШАЯ ШКОЛА ЭКОНОМИКИ}}\\
			\vspace{0.25cm}
			{\large{Факультет физики}}\\
			\vspace{5.5cm}
			{\Huge\textbf{{Домашнее задание}}}\\%Общее название
			\vspace{1cm}
			{\LARGE{Введение в астрофизику, неделя 4}}\\%Точное название
			\vspace{2cm}
			{Задание выполнил студент 3 курса}\\
			{Захаров Сергей Дмитриевич}
			\vfill
			\includegraphics[width = 0.2\textwidth]{HSElogo}\\
			\vfill
			Москва\\
			2020
		\end{center}
	\end{titlepage}

\section*{Задача 1}

На звезде с радиусом 500 000 км и температурой поверхности 5000 К возникло пятно с диаметром 30 000 км и температурой 4000 К. На сколько упадет светимость звезды, если пятно прямо на луче зрения?

\subsubsection*{Решение}

Воспользуемся определением светимости, чтобы посчитать $L$ до того, как возникло пятно:

\begin{equation}
	L = S \cdot \sigma T^4 \qrq L_1 = 4 \pi R^2 \sigma T^4
\end{equation}

Возникшее пятно заставляет нас разбить светимость на две: на светимость всей звезды за вычетом одного ее небольшого кусочка и на светимость, собственно, пятна:

\begin{equation}
	S_{\text{spot}} = \pi\frac{D^2}{4} \qrq L_2 = \pi\left(4R^2 - \frac{D^2}{4}\right) \sigma T^4 + \pi \frac{D^2}{4}\sigma T^4_{\text{spot}}
\end{equation}

Наконец, найдем изменение светимости как разность $L_1 - L_2$:

\begin{equation}
	\boxed{\Delta L = \pi\frac{D^2}{4}\sigma \left(T^4 - T^4_{\text{spot}}\right) \approx 1.47 \cdot 10^{29} \text{ эрг}}
\end{equation}

\section*{Задача 2}

На звезде происходит вспышка с полным энерговыделением $10^{34}$ эрг и длительностью 30 минут. Считая светимость постоянной во время вспышки, определить, на сколько звездных величин возрастает блеск звезды, если ее масса равна 0.5 масс Солнца (вся энергия вспышки перешла в оптическое излучение).

\subsubsection*{Решение}

Предположим, что зависимость светимости от массы следующая:

\begin{equation}
	\frac{L}{L_{\odot}} = \left(\frac{M}{M_{\odot}}\right)^4 \qrq L = L_{\odot} \cdot \left(\frac{M}{M_{\odot}}\right)^4
\end{equation}

Для звезды в 0.5 масс Солнца показатель 4 выглядит вменяемым.

Предполагаем, что вся энергия вспышки пошла в светимость, тогда:

\begin{equation}
	m_1 - m_0 = -2.5 \lg \left(\frac{L + E/t}{L}\right) = \boxed{-2.5 \lg \left(1 + \frac{E/t}{L_{\odot}} \cdot \left(\frac{M_{\odot}}{M}\right)^4\right) = \Delta m \approx -0.025^\text{m}}
\end{equation}

\section*{Задача 3}

Две звезды имеют светимости по 10 000 светимостей Солнца. Одна из них имеет 8 видимую звездную величину, а вторая - 13-ую. Сравнить параллаксы этих звезд (поглощением света в межзвездной среде пренебречь) и сделать выводы о потенциальной наблюдаемости таких параллаксов с помощью современного оборудования.

\subsubsection*{Решение}

Запишем общую формулу для работы со звездными величинами:

\begin{equation}
	m_1 - m_2 = -2.5 \lg \left(\frac{E_1}{E_2}\right)
\end{equation}

С учетом того, что светимости звезд одинаковые, а освещенность обратно пропорциональна квадрату расстояния до объекта излучения, получим:

\begin{equation}
	m_1 - m_2 = 5 \lg\left(\frac{r_1}{r_2}\right) \qrq r_2 = r_1 \cdot 10^{(m_2 - m_1) / 5}
\end{equation}

Для того, чтобы найти $r_1$, сравним звезду с Солнцем, учтя теперь, что их светимости отличаются, а освещенность прямо пропорциональна светимости:

\begin{equation}
	m_\odot - m_1 = 2.5 \lg \left(\frac{r_{\odot}^2}{r_1^2} \cdot \frac{L_1}{L_{\odot}}\right) \qrq r_1 = r_{\odot} \cdot 10^{(m_1 - m_{\odot}) / 5} \cdot \sqrt{\frac{L_1}{L_{\odot}}}
\end{equation}

Здесь $r_{\odot} = 1 \text{ а.е.} \approx 4.84\cdot 10^{-6} \text{ пк}$ --- расстояние от Земли до Солнца; $m_\odot \approx -26.74$ --- видимая звездная величина Солнца с Земли, $L_{\odot}$ --- светимость Солнца.

Подставив числа и учтя, что параллакс в секундах равен 1 / расстояние  до объекта в парсеках:

\begin{align*}
	r_1 &= 4293.84 \text{ пк} \qrq \boxed{\pi_1 \approx 0.00023\text{ as}}\\
	r_2 &= 42938.4 \text{ пк} \qrq \boxed{\pi_2 \approx 0.000023\text{ as}}
\end{align*}

На основании данных о точности измерения параллакса GAIA (а это порядка $25 \mu\text{as}$), делаем вывод, что первый параллакс явно наблюдаемый, а второй находится на границе, и уже не факт, что сможет наблюдаться.


\end{document}