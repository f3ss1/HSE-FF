%
%Не забыть:
%--------------------------------------
%Вставить колонтитулы, поменять название на титульнике



%--------------------------------------

\documentclass[a4paper, 12pt]{article} 

%--------------------------------------
%Russian-specific packages
%--------------------------------------
%\usepackage[warn]{mathtext}
\usepackage[T2A]{fontenc}
\usepackage[utf8]{inputenc}
\usepackage[english,russian]{babel}
\usepackage[intlimits]{amsmath}
\usepackage{esint}
%--------------------------------------
%Hyphenation rules
%--------------------------------------
\usepackage{hyphenat}
\hyphenation{ма-те-ма-ти-ка вос-ста-нав-ли-вать}
%--------------------------------------
%Packages
%--------------------------------------
\usepackage{amsmath}
\usepackage{amssymb}
\usepackage{amsfonts}
\usepackage{amsthm}
\usepackage{latexsym}
\usepackage{mathtools}
\usepackage{etoolbox}%Булевые операторы
\usepackage{extsizes}%Выставление произвольного шрифта в \documentclass
\usepackage{geometry}%Разметка листа
\usepackage{indentfirst}
\usepackage{wrapfig}%Создание обтекаемых текстом объектов
\usepackage{fancyhdr}%Создание колонтитулов
\usepackage{setspace}%Настройка интерлиньяжа
\usepackage{lastpage}%Вывод номера последней страницы в документе, \lastpage
\usepackage{soul}%Изменение параметров начертания
\usepackage{hyperref}%Две строчки с настройкой гиперссылок внутри получаеммого
\usepackage[usenames,dvipsnames,svgnames,table,rgb]{xcolor}% pdf-документа
\usepackage{multicol}%Позволяет писать текст в несколько колонок
\usepackage{cite}%Работа с библиографией
\usepackage{subfigure}% Человеческая вставка нескольких картинок
\usepackage{tikz}%Рисование рисунков
\usepackage{float}% Возможность ставить H в положениях картинки
% Для картинок Моти
\usepackage{misccorr}
\usepackage{lscape}
\usepackage{cmap}



\usepackage{graphicx,xcolor}
\graphicspath{{Pictures/}}
\DeclareGraphicsExtensions{.pdf,.png,.jpg}

%----------------------------------------
%Список окружений
%----------------------------------------
\newenvironment {theor}[2]
{\smallskip \par \textbf{#1.} \textit{#2}  \par $\blacktriangleleft$}
{\flushright{$\blacktriangleright$} \medskip \par} %лемма/теорема с доказательством
\newenvironment {proofn}
{\par $\blacktriangleleft$}
{$\blacktriangleright$ \par} %доказательство
%----------------------------------------
%Список команд
%----------------------------------------
\newcommand{\grad}
{\mathop{\mathrm{grad}}\nolimits\,} %градиент

\newcommand{\diver}
{\mathop{\mathrm{div}}\nolimits\,} %дивергенция

\newcommand{\rot}
{\ensuremath{\mathrm{rot}}\,}

\newcommand{\Def}[1]
{\underline{\textbf{#1}}} %определение

\newcommand{\RN}[1]
{\MakeUppercase{\romannumeral #1}} %римские цифры

\newcommand {\theornp}[2]
{\textbf{#1.} \textit{ #2} \par} %Написание леммы/теоремы без доказательства

\newcommand{\qrq}
{\ensuremath{\quad \Rightarrow \quad}} %Человеческий знак следствия

\newcommand{\qlrq}
{\ensuremath{\quad \Leftrightarrow \quad}} %Человеческий знак равносильности

\renewcommand{\phi}{\varphi} %Нормальный знак фи

\newcommand{\me}
{\ensuremath{\mathbb{E}}}

\newcommand{\md}
{\ensuremath{\mathbb{D}}}

\newcommand{\bra}[1]
{\ensuremath{\left\langle#1\right|}}

\newcommand{\cat}[1]
{\ensuremath{\left|#1\right\rangle}}



%\renewcommand{\vec}{\overline}




%----------------------------------------
%Разметка листа
%----------------------------------------
\geometry{top = 3cm}
\geometry{bottom = 2cm}
\geometry{left = 1.5cm}
\geometry{right = 1.5cm}
%----------------------------------------
%Колонтитулы
%----------------------------------------
\pagestyle{fancy}%Создание колонтитулов
\fancyhead{}
%\fancyfoot{}
%\fancyhead[R]{\textsc{Уравнения Лондонов. Кинетическая индуктивность сверхпроводников.}}%Вставить колонтитул сюда
%----------------------------------------
%Интерлиньяж (расстояния между строчками)
%----------------------------------------
%\onehalfspacing -- интерлиньяж 1.5
%\doublespacing -- интерлиньяж 2
%----------------------------------------
%Настройка гиперссылок
%----------------------------------------
\hypersetup{				% Гиперссылки
	unicode=true,           % русские буквы в раздела PDF
	pdftitle={Заголовок},   % Заголовок
	pdfauthor={Автор},      % Автор
	pdfsubject={Тема},      % Тема
	pdfcreator={Создатель}, % Создатель
	pdfproducer={Производитель}, % Производитель
	pdfkeywords={keyword1} {key2} {key3}, % Ключевые слова
	colorlinks=true,       	% false: ссылки в рамках; true: цветные ссылки
	linkcolor=blue,          % внутренние ссылки
	citecolor=blue,        % на библиографию
	filecolor=magenta,      % на файлы
	urlcolor=red           % на URL
}
%----------------------------------------
%Работа с библиографией (как бич)
%----------------------------------------
\renewcommand{\refname}{Список литературы}%Изменение названия списка литературы для article
%\renewcommand{\bibname}{Список литературы}%Изменение названия списка литературы для book и report
%----------------------------------------
\begin{document}
	\begin{titlepage}
		\begin{center}
			$$$$
			$$$$
			$$$$
			$$$$
			{\Large{НАЦИОНАЛЬНЫЙ ИССЛЕДОВАТЕЛЬСКИЙ УНИВЕРСИТЕТ}}\\
			\vspace{0.1cm}
			{\Large{ВЫСШАЯ ШКОЛА ЭКОНОМИКИ}}\\
			\vspace{0.25cm}
			{\large{Факультет физики}}\\
			\vspace{5.5cm}
			{\Huge\textbf{{Домашнее задание}}}\\%Общее название
			\vspace{1cm}
			{\LARGE{Введение в астрофизику, неделя 1}}\\%Точное название
			\vspace{2cm}
			{Задание выполнил студент 3 курса}\\
			{Захаров Сергей Дмитриевич}
			\vfill
			\includegraphics[width = 0.2\textwidth]{HSElogo}\\
			\vfill
			Москва\\
			2020
		\end{center}
	\end{titlepage}

\section*{Задача 1}

Квазар со светимостью $10^{12}$ светимостей Солнца находится на расстоянии 300 Мпк. Каким должен быть диаметр телескопа, чтобы можно было увидеть этот источник, считая, что невооруженным глазом (диаметром 8мм) мы видим звезду 6-й величины?

\subsubsection*{Решение}

Сперва запишем общую формулу для работы со звездными величинами:

\begin{equation}
	m_1 - m_2 = -2.5 \lg \left(\frac{E_1}{E_2}\right)
	\label{eq:mag_main}
\end{equation}

Затем определим абсолютную звездную величину квазара. Для этого в формуле для звездной величины \ref{eq:mag_main} заменим отношение освещенностей на отношение светимостей (это возможно, поскольку абсолютная звездна величина по определению измеряется на фиксированном расстоянии 10 пк). Отсюда получим (принимая, что абсолютная звездная величина солнца примерно составляет $4.83^\text{m}$) :

\begin{equation}
	M_q - M_{\odot} = -2.5 \lg\left(\frac{L_q}{L_{\odot}}\right) \qrq M_q = M_{\odot} -2.5 \lg\left(\frac{L_q}{L_{\odot}}\right) \approx -25.17^\text{m}
\end{equation}

После этого, зная расстояние до квазара, вычислим его видимую звездную величину, заменив отношение освещенностей в формуле \ref{eq:mag_main} на отношение величин, обратных квадратам расстояния (поскольку освещенность обратно пропорциональна квадрату расстояния до источника):

\begin{equation}
	m_q = M_q - 2.5 \lg\left(\frac{r^2}{r^2_{10\text{пк}}}\right) = M_q - 5^\text{m} + 5\lg r \approx 12.22^\text{m}
\end{equation}

Для определения диаметра телескопа учтем, что освещенность пропорциональна площади собирающей поверхности (а та, в свою очередь, квадрату диаметра зрачка или объектива):

\begin{equation}
	m_{\text{eye}} - m_{q} = -2.5 \lg \left(\frac{D^2}{d_{\text{eye}}^2}\right) \qrq D = d_{\text{eye}} \cdot 10^{(m_q - m_{\text{eye}}) / 5} \approx \boxed{140 \text{ мм}}
\end{equation}

Сделаем отдельную ремарку про знаки в последней формуле. Поскольку мы смотрим в телескоп глазом, логично, что на выходе из телескопа мы должны получать предельную звездную величину, которую глазом можно увидеть, т.е. указанную $m_{\text{eye}} \approx 6^{\text{m}}$. Ее мы сравниваем с той величиной, которую имеем при наблюдении невооруженным глазом, т.е. с $m_q$.

\newpage

\section*{Задача 2}

Рентгеновский источник испускает излучение на энергии 8 кэВ. Детектор площадью 1 квадратный метр за 10 000 секунд зарегистрировал 100 отсчетов. Определите светимость источника, если расстояние до него 30 кпк.

\subsubsection*{Решение}

Сразу переведем энергию одного фотона в СГС:

\begin{equation}
	E = 8\text{ кэВ} = 1.28 \cdot 10^{-8} \text{ эрг}
\end{equation}

Запишем поток, который получает детектор:

\begin{equation}
	\Phi = \frac{N E}{t S}
\end{equation}

Здесь $N$ --- число фотонов, полученное за время наблюдения, $E$ --- энергия одного фотона, $t$ --- время наблюдения, $S$ --- площадь детектора.

С другой стороны, поток можно расписать через светимость:

\begin{equation}
	\Phi = \frac{L}{4 \pi r^2}
\end{equation}

Здесь $L$ --- светимость объекта, $r$ --- расстояние до него.

Приравняв, получим:

\begin{equation}
	L = 4 \pi r^2 \frac{N E}{t S} \approx \boxed{1.4 \cdot 10^{33} \; \frac{\text{эрг}}{\text{с}}}
\end{equation}






\end{document}