%
%Не забыть:
%--------------------------------------
%Вставить колонтитулы, поменять название на титульнике



%--------------------------------------

\documentclass[a4paper, 12pt]{article} 

%--------------------------------------
%Russian-specific packages
%--------------------------------------
%\usepackage[warn]{mathtext}
\usepackage[T2A]{fontenc}
\usepackage[utf8]{inputenc}
\usepackage[english,russian]{babel}
\usepackage[intlimits]{amsmath}
\usepackage{esint}
%--------------------------------------
%Hyphenation rules
%--------------------------------------
\usepackage{hyphenat}
\hyphenation{ма-те-ма-ти-ка вос-ста-нав-ли-вать}
%--------------------------------------
%Packages
%--------------------------------------
\usepackage{amsmath}
\usepackage{amssymb}
\usepackage{amsfonts}
\usepackage{amsthm}
\usepackage{latexsym}
\usepackage{mathtools}
\usepackage{etoolbox}%Булевые операторы
\usepackage{extsizes}%Выставление произвольного шрифта в \documentclass
\usepackage{geometry}%Разметка листа
\usepackage{indentfirst}
\usepackage{wrapfig}%Создание обтекаемых текстом объектов
\usepackage{fancyhdr}%Создание колонтитулов
\usepackage{setspace}%Настройка интерлиньяжа
\usepackage{lastpage}%Вывод номера последней страницы в документе, \lastpage
\usepackage{soul}%Изменение параметров начертания
\usepackage{hyperref}%Две строчки с настройкой гиперссылок внутри получаеммого
\usepackage[usenames,dvipsnames,svgnames,table,rgb]{xcolor}% pdf-документа
\usepackage{multicol}%Позволяет писать текст в несколько колонок
\usepackage{cite}%Работа с библиографией
\usepackage{subfigure}% Человеческая вставка нескольких картинок
\usepackage{tikz}%Рисование рисунков
\usepackage{float}% Возможность ставить H в положениях картинки
\usepackage{wasysym} % Астрономические знаки
% Для картинок Моти
\usepackage{misccorr}
\usepackage{lscape}
\usepackage{cmap}



\usepackage{graphicx,xcolor}
\graphicspath{{Pictures/}}
\DeclareGraphicsExtensions{.pdf,.png,.jpg}

%----------------------------------------
%Список окружений
%----------------------------------------
\newenvironment {theor}[2]
{\smallskip \par \textbf{#1.} \textit{#2}  \par $\blacktriangleleft$}
{\flushright{$\blacktriangleright$} \medskip \par} %лемма/теорема с доказательством
\newenvironment {proofn}
{\par $\blacktriangleleft$}
{$\blacktriangleright$ \par} %доказательство
%----------------------------------------
%Список команд
%----------------------------------------
\newcommand{\grad}
{\mathop{\mathrm{grad}}\nolimits\,} %градиент

\newcommand{\diver}
{\mathop{\mathrm{div}}\nolimits\,} %дивергенция

\newcommand{\rot}
{\ensuremath{\mathrm{rot}}\,}

\newcommand{\Def}[1]
{\underline{\textbf{#1}}} %определение

\newcommand{\RN}[1]
{\MakeUppercase{\romannumeral #1}} %римские цифры

\newcommand {\theornp}[2]
{\textbf{#1.} \textit{ #2} \par} %Написание леммы/теоремы без доказательства

\newcommand{\qrq}
{\ensuremath{\quad \Rightarrow \quad}} %Человеческий знак следствия

\newcommand{\qlrq}
{\ensuremath{\quad \Leftrightarrow \quad}} %Человеческий знак равносильности

\renewcommand{\phi}{\varphi} %Нормальный знак фи

\newcommand{\me}
{\ensuremath{\mathbb{E}}}

\newcommand{\md}
{\ensuremath{\mathbb{D}}}

\newcommand{\bra}[1]
{\ensuremath{\left\langle#1\right|}}

\newcommand{\cat}[1]
{\ensuremath{\left|#1\right\rangle}}

\newcommand{\const}{\text{const.}}



%\renewcommand{\vec}{\overline}




%----------------------------------------
%Разметка листа
%----------------------------------------
\geometry{top = 3cm}
\geometry{bottom = 2cm}
\geometry{left = 1.5cm}
\geometry{right = 1.5cm}
%----------------------------------------
%Колонтитулы
%----------------------------------------
\pagestyle{fancy}%Создание колонтитулов
\fancyhead{}
%\fancyfoot{}
%\fancyhead[R]{\textsc{Уравнения Лондонов. Кинетическая индуктивность сверхпроводников.}}%Вставить колонтитул сюда
%----------------------------------------
%Интерлиньяж (расстояния между строчками)
%----------------------------------------
%\onehalfspacing -- интерлиньяж 1.5
%\doublespacing -- интерлиньяж 2
%----------------------------------------
%Настройка гиперссылок
%----------------------------------------
\hypersetup{				% Гиперссылки
	unicode=true,           % русские буквы в раздела PDF
	pdftitle={Заголовок},   % Заголовок
	pdfauthor={Автор},      % Автор
	pdfsubject={Тема},      % Тема
	pdfcreator={Создатель}, % Создатель
	pdfproducer={Производитель}, % Производитель
	pdfkeywords={keyword1} {key2} {key3}, % Ключевые слова
	colorlinks=true,       	% false: ссылки в рамках; true: цветные ссылки
	linkcolor=blue,          % внутренние ссылки
	citecolor=blue,        % на библиографию
	filecolor=magenta,      % на файлы
	urlcolor=red           % на URL
}
%----------------------------------------
%Работа с библиографией (как бич)
%----------------------------------------
\renewcommand{\refname}{Список литературы}%Изменение названия списка литературы для article
%\renewcommand{\bibname}{Список литературы}%Изменение названия списка литературы для book и report
%----------------------------------------
\begin{document}
	\begin{titlepage}
		\begin{center}
			$$$$
			$$$$
			$$$$
			$$$$
			{\Large{НАЦИОНАЛЬНЫЙ ИССЛЕДОВАТЕЛЬСКИЙ УНИВЕРСИТЕТ}}\\
			\vspace{0.1cm}
			{\Large{ВЫСШАЯ ШКОЛА ЭКОНОМИКИ}}\\
			\vspace{0.25cm}
			{\large{Факультет физики}}\\
			\vspace{5.5cm}
			{\Huge\textbf{{Домашнее задание}}}\\%Общее название
			\vspace{1cm}
			{\LARGE{Введение в астрофизику, неделя 2}}\\%Точное название
			\vspace{2cm}
			{Задание выполнил студент 3 курса}\\
			{Захаров Сергей Дмитриевич}
			\vfill
			\includegraphics[width = 0.2\textwidth]{HSElogo}\\
			\vfill
			Москва\\
			2020
		\end{center}
	\end{titlepage}

\section*{Задача 1}

Орбитальная скорость Земли 30 км/с. Считая, что Юпитер в 5.2 раз дальше, оцените его орбитальную скорость. 

\subsubsection*{Решение}

Орбиты планет будем считать круговыми (эксцентриситеты реальных говорят, что можно). В таком случае, чтобы оценить скорость Юпитера, необходимо найти период его обращения. Воспользуемся \RN{3} законом Кеплера:

\begin{equation}
	\frac{T_{\earth}^2}{T_{\jupiter}^2} = \frac{a_{\earth}^3}{a_{\jupiter}^3} \qrq T_{\jupiter} = T_{\earth} \left(\frac{a_{\jupiter}}{a_{\earth}}\right)^{\frac{3}{2}}
\end{equation}

С учетом формы орбиты, получим выражение для скорости:

\begin{equation}
	v = \frac{2 \pi a}{T} \qrq v_{\jupiter} = v_{\earth} \frac{a_{\jupiter}}{a_{\earth}} \frac{T_{\earth}}{T_{\jupiter}} = \boxed{v_{\jupiter} = v_{\earth} \sqrt{\frac{a_{\earth}}{a_{\jupiter}}} \approx 13.16 \text{ км/с}}
\end{equation}

\section*{Задача 2}

Две системы. В первой вокруг звезды с массой Солнца вращается легкая планета. Во второй на такой же по размеру орбите вращается вторая звезда с массой Солнца. Как будут отличаться орбитальные периоды тел в двух системах?

\subsubsection*{Решение}

В случае легкой планеты будем пренебрегать ее массой. С учетом того, что во втором случае <<на орбите>> находится вторая звезда с такой же массой, а фразу <<на такой же по размеру орбите>> будем все же понимать так, что расстояние между звездами во втором случае равно расстоянию от звезды до планеты в первом, реальная полуось будем равна $a_2 = 0.5 a_1$, где $a_1$ --- полуось орбиты легкой планеты в силу того, что звезды будут вращаться вокруг общего центра масс, который в силу симметрии задачи находится на середине отрезка, соединяющего их.

Согласно обобщенному закону Кеплера (в случае легкой планеты пренебрежем ее массой):

\begin{equation}
	\frac{T_1^2}{T_2^2} \cdot \frac{M}{M + M} = \frac{a_1^3}{a_2^3} = 2^3 \qrq \boxed{\frac{T_1}{T_2} = \sqrt{2^4} = 4}
\end{equation}

\section*{Задача 3} 

В результате миграции планета приблизилась к звезде вдвое. Во сколько раз изменился орбитальный период?

\subsubsection*{Решение}

Согласно \RN{3} закону Кеплера:

\begin{equation}
	\frac{a_1^3}{a_2^3} = \frac{T_1^2}{T_2^2}
\end{equation}

Если планета приблизилась вдвое, то $a_1 = 2 a_2$. В таком случае получаем, что:

\begin{equation}
	\boxed{\frac{T_2}{T_1} = \sqrt{\frac{a_2^3}{a_1^3}} \approx 0.354}
\end{equation}


\end{document}