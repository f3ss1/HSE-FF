\thispagestyle{empty}

\begin{center}
	\textit{Федеральное государственное автономное учреждение \\
		высшего профессионального образования}
	\vspace{0.5ex}
	
	\textbf{НАЦИОНАЛЬНЫЙ ИССЛЕДОВАТЕЛЬСКИЙ УНИВЕРСИТЕТ \\ <<ВЫСШАЯ ШКОЛА ЭКОНОМИКИ>>}\\
	\vspace{0.5ex}
	\textbf{Факультет физики}\\
	\vspace{0.5ex}
	\textbf{Бакалавриат}\\
	\vspace{0.5ex}
	\small{\textit{Образовательная программа «Физика» 03.03.02}}
\end{center}
\vspace{7ex}

\begin{center}
	\vspace{2ex}
	{\Large\textbf{О\,Т\,Ч\,Е\,Т}}
	
	{\large\textbf{по учебной практике}} \\
	\vspace{1ex}	
	{по теме:}
	
	\Large{\textbf{\textit{<<Зародышеобразование и рост двумерных пленок йодида никеля
			на поверхности  Ni(110)>>}}}
	

\end{center}
\begin{flushright}
\vspace{30ex}
	\noindent
	Выполнил студент группы БФЗ183\\
	\textit{Захаров Сергей Дмитриевич}\\
\vspace{2ex}
\underline{\hspace{3cm}}\\
  
\end{flushright}
\begin{flushleft}
\vspace{1ex}
	\noindent
	\textbf{Проверил:}\\
	к.ф.-м.н.,
доцент Базовой кафедры\\
квантовых технологий\\
при Институте общей физики\\
им. А.М. Прохорова РАН\\
факультета физики НИУ ВШЭ\\

\textit{Комаров Никита Сергеевич}\\
\vspace{2ex}
\underline{\hspace{3cm}}\\
\vspace{2ex}
\today
\end{flushleft}

\begin{center}
    	\vfill
	Москва \\2021 г.
\end{center}


\newpage